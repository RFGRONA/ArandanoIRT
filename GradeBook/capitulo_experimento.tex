% Capítulo V: Estudio Experimental en Plantas de Arándano Biloxi
\chapter{ESTUDIO EXPERIMENTAL}

\section{Introducción}
Se presenta el contexto del cultivo de arándano biloxi y la relevancia de detectar tempranamente Botrytis cinerea. Se definen los objetivos específicos del estudio experimental.

\section{Objetivos}
\begin{itemize}
    \item Obtener el patógeno \textit{Botrytis cinerea} en condiciones controladas.
    \item Infectar y registrar plantas de arándano biloxi con el hongo mediante datos cuantitativos y cualitativos.
    \item Evaluar la efectividad de la detección temprana mediante termografía con los datos parciales.
    \item Comparar los resultados entre plantas infectadas y de control.
\end{itemize}

\section{Preparación del Hongo \textit{Botrytis cinerea}}
Descripción del protocolo para el manejo, cultivo y preparación del hongo, detallando medidas de seguridad y procedimientos para garantizar la viabilidad del patógeno.

\section{Diseño Experimental}
\begin{itemize}
    \item \textbf{Procedimiento de Infección:} Descripción de cómo se realizará la inoculación en las plantas.
    \item \textbf{Planificación de la Toma de Datos:} Definición del cronograma, frecuencia y condiciones bajo las cuales se efectuarán las mediciones.
    \item \textbf{Técnicas Utilizadas:} Detalle de las metodologías (incluyendo termografía) y herramientas empleadas para la captura de datos.
\end{itemize}

\section{Recolección y Análisis de Datos}
Metodología para la recopilación de datos experimentales, métodos estadísticos aplicados para el análisis comparativo entre plantas infectadas y de control, y presentación de resultados preliminares.

\section{Resultados}
Presentación detallada de los hallazgos experimentales, análisis crítico de los resultados y comparación con los objetivos planteados en el estudio. Se discutirán las implicaciones de la detección temprana y su potencial para mejorar el manejo del cultivo.