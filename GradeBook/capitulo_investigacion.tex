% Capítulo I: Informe de Investigación
\doublespacing
\chapter{INFORME DE INVESTIGACIÓN}

\section{Estado del Arte}
% --- INICIA EL ESTADO DEL ARTE ---
\label{sec:estado_del_arte}

La termografía infrarroja (IRT) es una técnica basada en el estudio de la radiación emitida por los cuerpos, lo que permite generar una imagen representativa de su temperatura superficial. En España, investigadores del Instituto de Investigación y Formación Agraria y Pesquera (IFAPA) señalan que esta técnica es robusta, rápida y versátil para el monitoreo agrícola \cite{GarciaTejero2015}. Su aplicabilidad se fundamenta en que la temperatura foliar es un valioso indicador del estado fisiológico de la planta, ya que responde a diversos factores de estrés \cite{Pineda2021}.

Una de sus aplicaciones se da en el ámbito agrícola, donde se destaca frente a otros métodos de medición de temperatura. Esto se evidencia en el estudio ``Open-source time-lapse thermal imaging camera for canopy temperature monitoring'', realizado por investigadores de la Universidad Estatal de Michigan \cite{Dong2024}. En este trabajo, los autores analizan diversos métodos para monitorear la temperatura de la superficie foliar, cuyos resultados se sintetizan en la Tabla \ref{tab:comparativo_metodos}.

\FloatBarrier
\begin{table}[h!]
    \centering
    \caption{Comparativo entre métodos de medición de temperatura.}
    \label{tab:comparativo_metodos}
    \renewcommand{\arraystretch}{1.1}
    \small
    \begin{tabular}{|>{\raggedright\arraybackslash}p{2.5cm}|>{\raggedright\arraybackslash}p{4.5cm}|>{\raggedright\arraybackslash}p{4.0cm}|>{\raggedright\arraybackslash}p{4.0cm}|}
        \hline
        \textbf{Método} & \textbf{Descripción} & \textbf{Ventajas} & \textbf{Desventajas} \\
        \hline
        Termómetro Infrarrojo & Utiliza un haz infrarrojo para medir la radiación/energía reflejada de una superficie objetivo. & 
        \begin{itemize}[topsep=0pt, itemsep=0pt, partopsep=0pt, parsep=0pt, leftmargin=1.2em]
            \item Método sin contacto
            \item Amplia disponibilidad en el mercado
            \item Preciso con certificado de calibración
        \end{itemize} & 
        \begin{itemize}[topsep=0pt, itemsep=0pt, partopsep=0pt, parsep=0pt, leftmargin=1.2em]
            \item Problemas con la emisividad
            \item Errores aumentan con la distancia
            \item Limitado a un solo punto en la superficie de la hoja
        \end{itemize} \\
        \hline
        Sonda de Temperatura Táctil & El sensor de temperatura se sujeta a la superficie de la hoja y mide la resistencia térmica usando una fuente de corriente constante y tres cables. &
        \begin{itemize}[topsep=0pt, itemsep=0pt, partopsep=0pt, parsep=0pt, leftmargin=1.2em]
            \item Bajo costo
            \item Alta precisión para mediciones puntuales
        \end{itemize} &
        \begin{itemize}[topsep=0pt, itemsep=0pt, partopsep=0pt, parsep=0pt, leftmargin=1.2em]
            \item Puede alterar el microambiente natural de la planta
            \item Problemas con hojas pequeñas, jóvenes o muy delgadas
            \item Fluctuaciones de temperatura en el cable del sensor
        \end{itemize} \\
        \hline
        Cámara de Imagen Térmica & Utilizada ampliamente en agricultura para monitorear la salud de las plantas, programación de riego, detección de enfermedades, estimación de rendimiento, etc. &
        \begin{itemize}[topsep=0pt, itemsep=0pt, partopsep=0pt, parsep=0pt, leftmargin=1.2em]
            \item Observación continua de imágenes térmicas
            \item Información útil sobre el estrés hídrico y la salud de las plantas
        \end{itemize} &
        \begin{itemize}[topsep=0pt, itemsep=0pt, partopsep=0pt, parsep=0pt, leftmargin=1.2em]
            \item Cámaras térmicas continúas limitadas y costosas
            \item Método más complejo y caro en comparación con otros métodos
        \end{itemize} \\
        \hline
    \end{tabular}
\end{table}
\FloatBarrier

De acuerdo con la tabla anterior, se destacan las ventajas de la cámara de imagen térmica, que permite monitorear la temperatura de las plantas de manera no invasiva. Por lo anterior se propone utilizar esta tecnología para el proyecto. No obstante, debido a los altos costos de las cámaras térmicas, se busca emplear hardware de bajo costo, con la expectativa de obtener resultados que permitan cumplir los objetivos de esta investigación.

% --- CÓDIGO PARA LA FIGURA 1 ---
\begin{figure}[h!]
    \centering
    \caption{Imagen termográfica de una planta con cámara MLX90640.}
    \includegraphics[width=0.7\textwidth]{img/termo.png} 
    \label{fig:termografia_mlx}
\end{figure}

En la agricultura, la IRT se ha consolidado como una herramienta potente para la gestión hídrica. La relación es directa: el estrés por falta de agua provoca un cierre estomático que reduce la transpiración y, en consecuencia, eleva la temperatura de la planta \cite{GarciaTejero2015}. A pesar de su potencial, uno de los principales desafíos para su adopción ha sido el alto costo de los equipos comerciales. Sin embargo, investigaciones recientes en Estados Unidos, lideradas por la Universidad Estatal de Michigan, han demostrado la viabilidad de utilizar hardware de bajo costo, como cámaras térmicas compactas acopladas a microcomputadoras, para monitorear la temperatura del dosel con precisión suficiente para aplicaciones agrícolas, incluyendo el arándano \cite{Dong2024}.

En el contexto latinoamericano, también se ha validado su eficacia. Un estudio conjunto entre el Instituto Federal de Espírito Santo en Brasil y la Universidad de Georgia en Estados Unidos, demostró que la termografía diferencia eficazmente entre plantas de cítricos con distintos niveles de riego en invernadero \cite{Vieira2021}. Asimismo, en Chile, investigadores de la Universidad de Concepción validaron el uso de índices térmicos como el CWSI (Índice de Estrés Hídrico del Cultivo) como un indicador fiable del estado hídrico en frutales \cite{Quezada2020}. Esta tecnología es particularmente relevante para el arándano, un cultivo cuyo sistema radicular superficial lo hace muy susceptible a las fluctuaciones de agua, según lo documentado por el Instituto de Investigaciones Agropecuarias de Chile \cite{Morales2017}. Una gestión hídrica inadecuada en este cultivo impacta negativamente el rendimiento y la calidad de la fruta, haciendo fundamental contar con métodos de detección de estrés.

Este tema ha cobrado relevancia en Colombia, como lo demuestran los avances a nivel de software para el análisis de imágenes infrarrojas realizados por la Universidad del Valle, en su estudio sobre el uso de la termografía en América Latina \cite{Aux2022}. Paralelamente, la industria del arándano en el país ha crecido notablemente, con un aumento promedio anual del 9.1\% desde el año 2000 \cite{Aimeth2018}. Este crecimiento se ha acelerado desde 2019, siendo Boyacá uno de los departamentos líderes en la producción \cite{Blanco2023}. La importancia económica del cultivo es subrayada en un estudio de la Universidad de Cundinamarca, donde se destaca su gran potencial de exportación \cite{Augusto2023}. En este contexto, la Universidad Pedagógica y Tecnológica de Colombia menciona que la variedad Biloxi es la principal sembrada en el Altiplano Cundiboyacense debido a su resistencia, buen calibre y producción \cite{QuintanaReina2020}.

En resumen, los estudios internacionales y regionales demuestran que la termografía es una tecnología con gran potencial para monitorear el estado hídrico de las plantas. Su aplicación en el arándano Biloxi ofrece una herramienta útil para detectar el estrés hídrico y optimizar el riego. Sin embargo, la aplicación de esta tecnología sigue estando limitada debido a los altos costos del hardware especializado. La combinación del crecimiento de la industria del arándano en Colombia y los avances en hardware de bajo costo ofrece una oportunidad clara para desarrollar soluciones tecnológicas que mejoren la producción y protección del cultivo en el país.

\section{Línea de Investigación}
\label{subsec:just_linea_inv}

La adscripción de este proyecto a la línea de investigación \textbf{Aprendizaje, conocimiento, tecnologías, comunicación y digitalización} se justifica plenamente por la naturaleza y los objetivos de la solución propuesta. El sistema de detección de estrés hídrico no es un fin en sí mismo, sino una herramienta integral que aborda sistemáticamente cada uno de los pilares de esta línea:

\begin{description}
    \item [Tecnologías y Digitalización:] 
    El núcleo del proyecto es la \emph{aplicación de tecnología} (hardware de bajo costo como sensores IRT y microcontroladores) y el desarrollo de software para la \emph{digitalización} de un proceso agrícola. Se captura un fenómeno físico y análogo (la temperatura foliar de la planta) y se transforma en datos digitales estructurados y analizables, lo cual es la esencia de la agricultura de precisión y la Industria 4.0.

    \item [Conocimiento:] 
    El sistema no se limita a recolectar datos; su función principal es procesar esa información digital para generar \emph{conocimiento} nuevo y accionable. Transforma lecturas térmicas brutas en un indicador claro y comprensible (un reporte de estado hídrico) que representa el estado fisiológico de la planta. Este es un conocimiento al que el agricultor no podría acceder por simple observación.

    \item [Comunicación:] 
    El resultado de este procesamiento se convierte en un acto de \emph{comunicación} fundamental. El sistema actúa como un puente que ``traduce'' el estado fisiológico del cultivo y \emph{comunica} eficazmente un reporte al agricultor. Se supera así la barrera de la detección visual tardía, estableciendo un nuevo canal de comunicación directo entre la planta y el productor.

    \item [Aprendizaje:] 
    Finalmente, la iniciativa cierra el ciclo del \emph{aprendizaje}. Al recibir esta información (comunicación) derivada del conocimiento (procesamiento), el agricultor \emph{aprende} en tiempo real sobre las necesidades hídricas de su cultivo. Este aprendizaje validado por datos fomenta una cultura de gestión eficiente, optimiza el uso de recursos y permite tomar decisiones informadas para mejorar la productividad.
\end{description}

En conjunto, el proyecto es un ejemplo claro de cómo la tecnología y la digitalización facilitan la generación de conocimiento, mejoran la comunicación y promueven un aprendizaje práctico y continuo en el sector agrícola.

\section{Planteamiento del Problema y Pregunta de Investigación}
\label{sec:planteamiento_problema}

La producción de arándanos en Colombia está en una fase de expansión significativa, impulsada por una alta demanda en los mercados internacionales. La variedad Biloxi se ha destacado como una de las más cultivadas en el altiplano cundiboyacense debido a su adaptabilidad y calidad de fruto \cite{QuintanaReina2020}. La relevancia económica de este cultivo se refleja en el notable crecimiento de las exportaciones. En 2024, estas sumaron US\$3,29 millones, lo que representó una cifra histórica que se elevó 85\,\% frente a 2023 y puso fin a tres años consecutivos de caídas \cite{DANE2025}.

A pesar de su potencial, el arándano es un cultivo particularmente sensible a las condiciones hídricas. Su sistema radicular se caracteriza por ser fibroso, superficial y carente de pelos absorbentes, lo que limita su capacidad para explorar el perfil del suelo en busca de agua \cite{Morales2017}. Esta característica lo hace extremadamente vulnerable al estrés hídrico, tanto por déficit como por exceso, lo que puede provocar una reducción drástica en el crecimiento de la planta, el calibre de los frutos y, en última instancia, el rendimiento total del cultivo. Sin un manejo preciso del riego, los productores enfrentan un riesgo constante de pérdidas económicas significativas.

Fisiológicamente, las plantas responden al déficit de agua cerrando sus estomas para reducir la pérdida de agua por transpiración. La transpiración es un mecanismo de enfriamiento natural; por lo tanto, su disminución provoca un aumento medible en la temperatura de la superficie de las hojas \cite{GarciaTejero2015}. Este cambio térmico es un indicador temprano y fiable del estrés hídrico, a menudo detectable antes de que los síntomas visuales, como la marchitez, sean evidentes \cite{Pineda2021}.

La termografía infrarroja (IRT) es una técnica no invasiva que permite detectar estas variaciones de temperatura y, por consiguiente, monitorear el estado hídrico de los cultivos para optimizar el riego \cite{Dong2024, Vieira2021}. Su aplicación en la agricultura de precisión es una herramienta valiosa para una gestión más sostenible y automatizada. Sin embargo, su adopción se ha visto limitada por el alto costo de los equipos termográficos, lo que los hace inaccesibles para muchos agricultores e investigadores \cite{GarciaTejero2015}.

Considerando la vulnerabilidad del arándano Biloxi al estrés hídrico y la barrera económica de las tecnologías de monitoreo existentes, surge la necesidad de explorar alternativas más asequibles. Esto conduce a la siguiente pregunta de investigación: ¿Cómo se puede desarrollar un sistema que utilice hardware de bajo costo para la detección aproximada del estrés hídrico en plantas de arándano Biloxi mediante termografía?


\section{Objetivo General y Objetivos Específicos}
\subsection{Objetivo General}
Desarrollar un sistema basado en termografía infrarroja (IRT) y hardware de bajo costo para la detección del estrés hídrico en plantas de arándano Biloxi.

\subsection{Objetivos Específicos}
\begin{enumerate}
    \item Identificar y documentar los requisitos funcionales y no funcionales del sistema.
    \item Modelar la arquitectura del sistema mediante la creación de diagramas UML.
    \item Integrar el hardware del módulo termográfico para la recolección de datos.
    \item Desarrollar el software que recolecte los datos del módulo termográfico para su posterior procesamiento.
    \item Evaluar la generación de un reporte de estado hídrico que diferencie entre plantas de arándano Biloxi con riego óptimo y con déficit, a partir de los datos recopilados por el sistema.
\end{enumerate}

\section{ Alcance e Impacto del Proyecto }
\label{sec:impacto}

El presente proyecto busca ser un apoyo a la producción de arándanos a través de la implementación de tecnologías de precisión \emph{asequibles}, como la termografía de bajo costo, para monitorear y gestionar la salud de los cultivos. Esto podría permitir a los agricultores minimizar las pérdidas en las cosechas y, con ello, contribuir a ``satisfacer los desafíos de seguridad alimentaria local, regional y global del siglo XXI'' \cite{Vargas2021}.

La agricultura de precisión, apoyada en tecnologías como la termografía, es fundamental para una producción más sostenible, ya que permite optimizar el uso de recursos críticos como el agua \cite{Pineda2021}. La capacidad de monitorear el estado hídrico de las plantas en tiempo real posibilita una gestión del riego más eficiente, aplicando agua solo cuando y donde es realmente necesario \cite{GarciaTejero2015}. Este enfoque no solo promueve la conservación del recurso hídrico, sino que también contribuye a que los agricultores sean más resilientes y competitivos \cite{Dong2024}.

Este proyecto está alineado con los Objetivos de Desarrollo Sostenible (ODS) de las Naciones Unidas \cite{UNDPsf}. Específicamente, responde al \textbf{ODS 9 (Industria, Innovación e Infraestructura)}, al buscar promover la inversión y la innovación tecnológica en el sector agrícola. También contribuye directamente al \textbf{ODS 12 (Producción y Consumo Responsables)}, al fomentar una gestión eficiente de los recursos naturales mediante la implementación de prácticas agrícolas sostenibles.

En el plano local, la implementación de este proyecto puede tener un impacto significativo en la seguridad alimentaria y la economía regional. Al reducir las pérdidas económicas en la producción, se fortalecen las cadenas de suministro y se mejora la competitividad de un cultivo de alta importancia para el país \cite{Aimeth2018, Augusto2023}. Además, el acceso a tecnologías de precisión de bajo costo ayuda a reducir la brecha digital, democratizando herramientas avanzadas que de otro modo serían inaccesibles para pequeños y medianos productores, lo cual se alinea nuevamente con las metas del ODS 9.



\section{Metodología}
\label{sec:metodologia}

Para el desarrollo de la investigación se empleará una metodología mixta, que integra enfoques cuantitativos y cualitativos para obtener una comprensión más profunda del fenómeno estudiado. Este método permite combinar la recolección y análisis de datos numéricos con la interpretación de observaciones cualitativas, fortaleciendo así la validez de los resultados al compensar las limitaciones inherentes de cada enfoque por separado \cite{HamuiSutton2013}.

En el contexto de este proyecto, el componente cuantitativo se centrará en la recolección y análisis de los datos de temperatura superficial de la canopia, obtenidos a través del módulo de cámara térmica. El componente cualitativo consistirá en el registro y la descripción de los indicadores visuales del estado de la planta, como la turgencia de las hojas y el vigor general, bajo diferentes regímenes de riego. La integración de ambos tipos de datos permitirá establecer una relación más robusta entre las lecturas termográficas y el nivel de estrés hídrico aplicado, comprendiendo los factores que puedan influir en la efectividad del sistema.

Para la construcción del sistema, se utilizarán elementos del marco de trabajo ágil Scrum. Este enfoque se caracteriza por su flexibilidad, adaptabilidad y enfoque en la entrega de valor de manera incremental a lo largo del proyecto \cite{SCRUMstudy2022}. Se organizará el trabajo en ciclos cortos o Sprints, gestionando las tareas a través de un Product Backlog para asegurar un desarrollo organizado y eficiente del prototipo.


\section{Marcos de Referencia}
\subsection{Marco Teórico}
Texto del marco teórico...

\subsection{Marco Legal}
El desarrollo del presente proyecto de investigación y del sistema tecnológico asociado se enmarca en la normativa colombiana vigente, abordando tres ejes fundamentales: la propiedad intelectual, la protección de datos personales y la seguridad digital.

\subsubsection{Propiedad Intelectual y Derechos de Autor}
La titularidad y gestión de los derechos de autor generados en este proyecto se rigen por el \textbf{Acuerdo No. 04 de 2018}, por medio del cual se adopta el Estatuto de Propiedad Intelectual de la Universidad de Cundinamarca \cite{UDEC2018}. Con base en este estatuto, se establecen las siguientes claridades:
\begin{itemize}
    \item \textbf{Titularidad de los Estudiantes:} De acuerdo con el Artículo 16 del estatuto, la titularidad de los derechos de autor, tanto morales como patrimoniales, sobre el software desarrollado y el documento de tesis pertenece a los estudiantes autores, al ser una producción intelectual realizada en el marco de su trabajo de grado.
    \item \textbf{Reconocimiento de Coautoría:} En cumplimiento del mismo Artículo 16, se reconoce el derecho moral de la directora del proyecto a ser incluida como coautora de la obra, en virtud de su rol de orientación y asesoría en la investigación.
    \item \textbf{Licencia de Uso Académico para la Universidad:} Si bien la titularidad es de los estudiantes, se concede a la Universidad de Cundinamarca una licencia de uso no exclusiva y con fines académicos. Esto autoriza a la institución a incluir el trabajo de grado en sus bibliotecas y repositorios institucionales para fines de consulta, preservación e investigación, conforme a lo estipulado en los Artículos 14 y 24 del estatuto.
\end{itemize}

\subsubsection{Protección de Datos Personales}
El sistema contempla la gestión de usuarios, por lo que su diseño y futura implementación se alinean con la \textbf{Ley Estatutaria 1581 de 2012}, por la cual se dictan disposiciones generales para la protección de datos personales \cite{Congreso2012}. El cumplimiento de esta ley se aborda desde dos perspectivas:
\begin{itemize}
    \item \textbf{Fase Académica Actual:} En la etapa de desarrollo, el cumplimiento se centra en la adopción de buenas prácticas y en el diseño del sistema. Se garantiza el respeto a los principios de la ley, como la finalidad (los datos se usan solo para la autenticación), la seguridad de la información y el ejercicio de los derechos del titular a través de las funcionalidades que permiten al usuario gestionar sus propios datos.
    \item \textbf{Proyección a Futuro Uso Comercial:} Se reconoce que una eventual transición del sistema a un entorno comercial implicaría la asunción de obligaciones legales más estrictas, como el registro de la base de datos en el Registro Nacional de Bases de Datos (RNBD), la publicación de una política de tratamiento de datos y la implementación de canales formales para atender las solicitudes de los usuarios.
\end{itemize}

\subsubsection{Seguridad Digital y Delitos Informáticos}
El desarrollo del software no solo se enfoca en la funcionalidad, sino también en la seguridad, considerando el marco de la \textbf{Ley 1273 de 2009}, que modifica el Código Penal en materia de delitos informáticos \cite{Congreso2009}. Las medidas de seguridad implementadas, como los controles de acceso y la autenticación de usuarios, tienen un doble propósito: proteger la integridad de los datos personales y prevenir de manera proactiva la comisión de conductas delictivas, como el acceso abusivo a un sistema informático. Este enfoque asegura que el proyecto se desarrolle de manera responsable y en conformidad con el marco legal que protege la información y los sistemas tecnológicos en el país.
