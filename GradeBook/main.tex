\documentclass[12pt,openright,oneside]{book}

% Paquetes necesarios
\usepackage[spanish]{babel}        % Idioma español
\usepackage[utf8]{inputenc}        % Codificación UTF-8
\usepackage[T1]{fontenc}           % Codificación de salida
\usepackage{newtxtext,newtxmath}   % Fuente Times New Roman
\usepackage{setspace}              % Interlineado
\usepackage{geometry}              % Márgenes
\usepackage{titlesec}              % Formato de títulos
\usepackage{fancyhdr}              % Encabezados y pies de página
\usepackage{graphicx}              % Imágenes
\usepackage{tocloft}               % Formato del índice
\usepackage{hyperref}              % Hipervínculos
\usepackage{apacite}               % Citas y referencias APA
\usepackage{enumerate}             % Listas enumeradas
\usepackage{caption}               % Personalización de leyendas
\usepackage{subcaption}            % Subfiguras
\usepackage{longtable}             % Tablas largas
\usepackage{listings}              % Código fuente
\usepackage{pdfpages}              % Incluir PDF (para anexos)
\usepackage{color}                 % Colores
\usepackage{xcolor}                % Colores
\usepackage{float}                 % Posicionamiento de figuras
\usepackage{chngcntr}              % Cambiar numeración de figuras y tablas
\usepackage{array}                 % Tablas
\usepackage[table]{xcolor}         % Colores en tablas
\usepackage[T1]{fontenc}           % Codificación de salida
\usepackage{booktabs}              % Tablas con líneas horizontales
\usepackage{siunitx}               % Para alinear números en tablas
\usepackage{enumitem}              % Listas personalizadas
\usepackage{placeins}              % Control de flotantes
\usepackage{textcomp}              % Símbolos adicionales
\usepackage{amsmath}               % Símbolos matemáticos
\usepackage{gensymb}               % Símbolos generales

% Atributos de paquetes
\counterwithout{figure}{chapter}
\counterwithout{table}{chapter}
\definecolor{tablecolor}{HTML}{CECECE}

% Configuración de márgenes
\geometry{
    left=3cm,
    right=2cm,
    top=4cm,
    bottom=3cm,
}

% Configuración de hipervinculos
\hypersetup{
    colorlinks=true,   % Colorea los enlaces en lugar de usar recuadros
    linkcolor=black,   % Color para enlaces internos (índice, referencias cruzadas)
    citecolor=black,   % Color para enlaces de citas bibliográficas 
    urlcolor=blue      % Color para URLs externas
}

% Configuración para números en tablas
\sisetup{
    output-decimal-marker={,}, % Coma decimal
    group-separator={.},       % Punto para miles
    group-minimum-digits={4}   % Agrupar a partir de 4 dígitos
}

% Configuración de código fuente
% Configuración del paquete listings
\lstset{
    basicstyle=\ttfamily\small, % Estilo básico del código
    keywordstyle=\color{blue}\bfseries, % Estilo de las palabras clave
    commentstyle=\color{gray}\itshape, % Estilo de los comentarios
    stringstyle=\color{red}, % Estilo de las cadenas de texto
    numbers=left, % Números de línea a la izquierda
    numberstyle=\tiny\color{gray}, % Estilo de los números de línea
    stepnumber=1, % Numerar cada línea
    numbersep=5pt, % Separación entre el número y el código
    frame=single, % Marco alrededor del código
    framexleftmargin=15pt, % Margen izquierdo del marco
    xleftmargin=15pt, % Margen izquierdo del código
    captionpos=b, % Posición de la leyenda (caption) abajo
    breaklines=true, % Permitir saltos de línea
    breakatwhitespace=false, % Romper líneas en espacios en blanco
    showspaces=false, % No mostrar espacios en blanco
    showtabs=false, % No mostrar tabulaciones
    tabsize=2, % Tamaño de tabulación
    columns=fullflexible
}

% Interlineado 1.5 para todo el documento
\onehalfspacing

% Numeración de capítulos en números romanos y secciones en arábigos
\titleformat{\chapter}{\normalfont\huge\bfseries}{\Roman{chapter}.}{1em}{}
\titleformat{\section}{\normalfont\Large\bfseries}{\thesection.}{1em}{}

% Ajustes para el índice
\renewcommand{\contentsname}{ÍNDICE GENERAL}
\setlength{\cftbeforechapskip}{0pt}
\setlength{\cftbeforesecskip}{0pt}

% Encabezados y pies de página
\pagestyle{fancy}
\fancyhf{}
\renewcommand{\headrulewidth}{0pt}
\fancyfoot[C]{\thepage}

% En español, que diga "Figura" en lugar de "Figure"
\renewcommand{\figurename}{Figura}

% Configuración de tablas y figuras
\captionsetup[table]{font=normalsize,skip=0pt,singlelinecheck=false}
\addto\captionsspanish{\renewcommand{\tablename}{Tabla}}

% Ajustar espacio antes/después de la leyenda
\setlength{\abovecaptionskip}{8pt}   % espacio entre imagen y leyenda si la leyenda va debajo
\setlength{\belowcaptionskip}{0pt}  % espacio después de la leyenda

% Configuración de leyendas para figuras 
\captionsetup[figure]{%
  position=above,                       % Coloca la leyenda (número y título) encima de la imagen
  labelfont=bf,                         % "Figura X." en negrita
  labelsep=newline,                     % Salto de línea tras "Figura X."
  textfont=it,                          % Título en cursiva
  font={normalsize,stretch=1.5},         % Interlineado de 1.5 para cumplir APA
  skip=10pt,                            % Espacio entre la leyenda y la imagen (ajusta este valor según lo requieras)
  justification=justified,
  singlelinecheck=false
}

% Configuración de leyendas para tablas
\DeclareCaptionLabelFormat{apatable}{\textbf{Tabla #2}} % Define formato: "Tabla" + Número en negrita
\captionsetup[table]{
  labelformat=apatable,        % Usa el formato definido arriba
  labelsep=newline,            % Salto de línea tras "Tabla X"
  textfont=it,                 % Título en cursiva
  font={normalsize,stretch=1.5}, % Tamaño normal, interlineado 1.5
  justification=justified,     % Justifica el texto del título
  singlelinecheck=false,       % Aplica justificación incluso a títulos cortos
  skip=10pt                    % Espacio entre la leyenda y la tabla
}

% Configuración de citas y referencias APA
\bibliographystyle{apacite}
\renewcommand{\APACrefYearMonthDay}[3]{\APACrefYear{#1}}

\begin{document}
% Configuración de numeración de páginas para las secciones preliminares
\pagenumbering{Roman}
\setcounter{page}{1}

% Portada
\begin{titlepage}
    \begin{center}
        \vspace*{2cm}
        {\Large\textbf{SISTEMA PARA LA DETECCIÓN DE ESTRÉS HÍDRICO EN EL ARÁNDANO BILOXI MEDIANTE TERMOGRAFÍA DE BAJO COSTO}}\\[4cm]
        {\large\textbf{AUTOR(ES)}}\\
        Juan Esteban Fuentes Rojas\\
        Gabriel Esteban Martinez Roldan\\[6cm]
        \textbf{UNIVERSIDAD DE CUNDINAMARCA}\\
        Facultad de Ingeniería\\
        Programa de Ingeniería de Sistemas y Computación\\
        Facatativá, Noviembre 2025
    \end{center}
\end{titlepage}

% Contraportada
\newpage
\thispagestyle{empty}
\begin{center}
    \vspace*{2cm}
    {\Large\textbf{SISTEMA PARA LA DETECCIÓN DE ESTRÉS HÍDRICO EN EL ARÁNDANO BILOXI MEDIANTE TERMOGRAFÍA DE BAJO COSTO}}\\[3cm]
    {\large\textbf{AUTOR(ES)}}\\
    Directora: Ing. Gina Maribel Valenzuela Sabogal\\
    Juan Esteban Fuentes Rojas\\
    Gabriel Esteban Martinez Roldan\\[3cm]
    \textbf{GRUPO DE INVESTIGACIÓN DE SISTEMAS Y TECNOLOGÍA DE FACATATIVÁ (GISTFA)}\\[3cm]
    \textbf{UNIVERSIDAD DE CUNDINAMARCA}\\
    Facultad de Ingeniería\\
    Programa de Ingeniería de Sistemas\\
    Facatativá, Noviembre 2025
\end{center}

% Dedicatoria (opcional)
\chapter*{Dedicatoria}
\addcontentsline{toc}{chapter}{Dedicatoria}
Texto de la dedicatoria...

% Agradecimientos (opcional)
\chapter*{Agradecimientos}
\addcontentsline{toc}{chapter}{Agradecimientos}
Texto de los agradecimientos...

% Compromiso de Autor (Anexo final, incluir si es necesario)

% Resumen y Palabras Clave
\chapter*{Resumen}
\addcontentsline{toc}{chapter}{Resumen}

% Interlineado sencillo para el resumen
\begin{spacing}{1}
La gestión eficiente del agua es un desafío creciente en la agricultura, especialmente para cultivos sensibles como el arándano Biloxi. Las tecnologías de monitoreo avanzado, como la termografía infrarroja (IRT), pueden ayudar a optimizar el riego al detectar el estrés hídrico de forma temprana midiendo la temperatura de las plantas. Sin embargo, el alto costo de los equipos comerciales limita su uso, particularmente para pequeños y medianos productores.

Este trabajo presenta el desarrollo completo de un sistema alternativo de bajo costo para monitorear el estrés hídrico en arándanos. Se construyó un prototipo utilizando componentes electrónicos asequibles, incluyendo un sensor térmico de matriz, sensores ambientales y un microcontrolador, junto con una aplicación web para visualizar los datos.

Para asegurar la confianza en el sistema, primero se evaluó la precisión y fiabilidad de sus componentes de hardware en condiciones controladas. Se confirmó que, con piezas nuevas y bien seleccionadas, el prototipo ofrece mediciones consistentes. No obstante, se observó que el uso prolongado puede afectar la precisión de algunos sensores de bajo costo, destacando la importancia de revisiones periódicas.

Posteriormente, se probó el sistema monitorizando una planta de arándano durante 55 días, simulando un periodo de sequía seguido de riego normal. El sistema demostró ser capaz de detectar los cambios en la temperatura de la planta asociados a la falta de agua y su posterior recuperación. Los indicadores térmicos calculados reflejaron claramente el aumento del estrés durante la sequía y la mejora tras volver a regar.

Por lo tanto, este proyecto demuestra que es viable construir herramientas de monitoreo termográfico funcionales y asequibles. Se presenta una solución tecnológica completa y validada que abre la puerta a una agricultura de precisión más accesible, ayudando a los agricultores a tomar mejores decisiones sobre el riego y a fomentar un uso más sostenible del agua.
\end{spacing}

\vspace{1cm}

\textbf{Palabras clave:} Termografía infrarroja, Estrés hídrico, Arándano Biloxi, Agricultura de precisión, Tecnología de bajo costo, Monitoreo de cultivos, Internet de las Cosas (IoT), Gestión del agua.

\newpage % Nueva página para el abstract y keywords

% Abstract and Keywords
\chapter*{Abstract}
\addcontentsline{toc}{chapter}{Abstract}
% Interlineado sencillo para el abstract
\begin{spacing}{1}
Efficient water management is a growing challenge in agriculture, especially for sensitive crops like Biloxi blueberries. Advanced monitoring technologies, such as infrared thermography (IRT), can help optimize irrigation by detecting water stress early through plant temperature measurements. However, the high cost of commercial equipment limits its use, particularly for small and medium-sized producers.

This work presents the complete development of an alternative low-cost system for monitoring water stress in blueberries. A prototype was built using affordable electronic components, including a thermal array sensor, environmental sensors, and a microcontroller, along with a web application to visualize the data.

To ensure confidence in the system, the accuracy and reliability of its hardware components were first evaluated under controlled conditions. It was confirmed that, with new and well-selected parts, the prototype provides consistent measurements. However, it was observed that prolonged use can affect the accuracy of some low-cost sensors, highlighting the importance of periodic checks.

Subsequently, the system was tested by monitoring a blueberry plant for 55 days, simulating a period of drought followed by normal irrigation. The system proved capable of detecting changes in plant temperature associated with water deficit and subsequent recovery. Calculated thermal indicators clearly reflected increased stress during drought and improvement after re-watering.

In conclusion, this project demonstrates that it is feasible to build functional and affordable thermographic monitoring tools. A complete and validated technological solution is presented, opening the door to more accessible precision agriculture, helping farmers make better irrigation decisions and promoting more sustainable water use.
\end{spacing}

\vspace{1cm}

\textbf{Keywords:} Infrared thermography, Water stress, Biloxi blueberry, Precision agriculture, Low-cost technology, Crop monitoring, Internet of Things (IoT), Water management.

% --- FIN RESUMEN Y ABSTRACT ---
% --- Aquí comenzaría tu \tableofcontents o el Capítulo 1 ---

\newpage  % Nueva página para el índice general

% Índice General
\tableofcontents
\newpage

% Lista de Tablas
\renewcommand{\listtablename}{Lista de Tablas}
\listoftables
\addcontentsline{toc}{chapter}{Lista de Tablas}
\newpage


% Lista de Figuras
\renewcommand{\listfigurename}{Lista de Figuras}
\listoffigures
\addcontentsline{toc}{chapter}{Lista de Figuras}
\newpage

% Introducción
\chapter*{Introducción}
\addcontentsline{toc}{chapter}{Introducción}
La agricultura contemporánea enfrenta el desafío dual de incrementar la producción de alimentos para una población creciente y, simultáneamente, adaptarse a los efectos del cambio climático, que intensifican la presión sobre recursos vitales como el agua \cite{Laveglia2024}. La gestión hídrica eficiente se ha convertido, por tanto, en una prioridad global, impulsando la adopción de prácticas agrícolas más sostenibles y tecnificadas. En este contexto, la agricultura de precisión emerge como una estrategia fundamental, empleando tecnologías para medir y responder a la variabilidad espacial y temporal de los cultivos, optimizando así el uso de insumos y mejorando la toma de decisiones agronómicas \cite{Dong2024, Vargas2021}.

Entre los cultivos de alto valor que demandan una gestión hídrica cuidadosa se encuentra el arándano (\textit{Vaccinium corymbosum} L.), cuya popularidad ha impulsado su expansión en diversas regiones, incluyendo el altiplano cundiboyacense colombiano \cite{Aimeth2018}. La variedad Biloxi, adaptada a climas con bajos requerimientos de frío, es particularmente relevante en esta zona \cite{QuintanaReina2020, Balbontin2023}. Sin embargo, el arándano posee un sistema radicular superficial y poco eficiente, lo que lo hace extremadamente vulnerable al estrés hídrico, tanto por déficit como por exceso, comprometiendo significativamente el crecimiento de la planta, el calibre de los frutos y el rendimiento final de la cosecha \cite{Morales2017, INIA2017, Almutairi2021, Salgado2018}. La detección temprana y precisa del estrés hídrico es, por lo tanto, crucial para optimizar el riego y asegurar la sostenibilidad económica del cultivo \cite{Rinza2021}.

La termografía infrarroja (IRT) se ha consolidado como una técnica no invasiva y eficaz para este propósito \cite{GarciaTejero2015, ErazoAux2022}. Se basa en el principio fisiológico de que las plantas, al experimentar déficit hídrico, cierran sus estomas para conservar agua, reduciendo la transpiración y provocando un aumento medible en la temperatura de la superficie foliar \cite{Jones2004, Pineda2021, Cho2024, Pradawet2023}. Índices derivados de la IRT, como el Índice de Estrés Hídrico del Cultivo (CWSI), permiten cuantificar este nivel de estrés \cite{Idso1981, Quezada2020}. No obstante, la aplicación generalizada de la IRT en la agricultura ha estado históricamente limitada por el elevado costo de las cámaras termográficas comerciales, creando una brecha tecnológica significativa, especialmente para pequeños y medianos productores \cite{Dong2024, Yun2023}.

Avances recientes en la tecnología de sensores, particularmente la aparición de matrices térmicas de bajo costo como el MLX90640 \cite{Melexis2021}, han abierto nuevas posibilidades para desarrollar herramientas de monitoreo más asequibles \cite{Dong2024}. Sin embargo, la fiabilidad de estos componentes de bajo costo en aplicaciones agrícolas exigentes requiere una validación metrológica rigurosa antes de su implementación práctica \cite{Acorsi2020, Jiao2022, Yun2023}. Factores como el desgaste por uso y la sensibilidad a condiciones ambientales pueden impactar su precisión y estabilidad a largo plazo.

Este trabajo aborda precisamente esta problemática, presentando el diseño, desarrollo, validación y prueba de concepto de un \textbf{sistema integral de bajo costo para la detección del estrés hídrico en arándano Biloxi mediante termografía infrarroja}. El sistema combina hardware asequible (microcontrolador ESP32-S3, sensor térmico MLX90640, sensores ambientales MEMS) con software embebido y una aplicación web para la adquisición, procesamiento (cálculo de $\Delta T$ y CWSI) y visualización de datos.

El proyecto se estructura en cinco capítulos principales. El \textbf{Capítulo I} establece el marco de la investigación, presentando el estado del arte, el planteamiento del problema, los objetivos, el alcance y la metodología general. El \textbf{Capítulo II} detalla exhaustivamente la documentación del software desarrollado, incluyendo la arquitectura (MVC, Onion), el stack tecnológico, la especificación de requerimientos, el diseño detallado mediante diagramas UML (Casos de Uso, Secuencia, Actividad, Clases, Despliegue) y una estimación de recursos basada en Puntos de Casos de Uso. El \textbf{Capítulo III} se enfoca en la documentación del hardware, describiendo los componentes seleccionados y la metodología empleada para su caracterización y validación metrológica, abordando aspectos clave como la precisión y el impacto del desgaste. El \textbf{Capítulo IV} presenta el estudio experimental realizado como prueba de concepto (PoC), detallando el diseño metodológico (caso único n=1), el procedimiento de estrés hídrico inducido y recuperación, la recolección de datos y los resultados obtenidos en términos de respuesta térmica ($\Delta T$, CWSI) de la planta monitoreada. Finalmente, el \textbf{Capítulo V} consolida los resultados y conclusiones finales, discutiendo la consecución de los objetivos, las fortalezas y limitaciones del sistema desarrollado, su impacto potencial y las recomendaciones para trabajos futuros.

Mediante la integración de la validación instrumental del hardware y la demostración funcional en un escenario experimental relevante, este trabajo busca no solo responder a la pregunta de investigación sobre la viabilidad de la termografía de bajo costo, sino también aportar una base técnica sólida para el desarrollo futuro de herramientas accesibles que impulsen una agricultura más precisa, eficiente y sostenible.

\cleardoublepage
\pagenumbering{arabic} % Reinicia numeración a arábiga para el cuerpo principal
\setcounter{page}{1} % La página siguiente será la 1

% Cuerpo del Documento
\mainmatter
\setcounter{page}{1}

% Capítulo I: Informe de Investigación
\doublespacing
\chapter{INFORME DE INVESTIGACIÓN}

\section{Estado del Arte}
% --- INICIA EL ESTADO DEL ARTE ---
\label{sec:estado_del_arte}

La termografía infrarroja (IRT) es una técnica basada en el estudio de la radiación emitida por los cuerpos, lo que permite generar una imagen representativa de su temperatura superficial. En España, investigadores del Instituto de Investigación y Formación Agraria y Pesquera (IFAPA) señalan que esta técnica es robusta, rápida y versátil para el monitoreo agrícola \cite{GarciaTejero2015}. Su aplicabilidad se fundamenta en que la temperatura foliar es un valioso indicador del estado fisiológico de la planta, ya que responde a diversos factores de estrés \cite{Pineda2021}.

Una de sus aplicaciones se da en el ámbito agrícola, donde se destaca frente a otros métodos de medición de temperatura. Esto se evidencia en el estudio ``Open-source time-lapse thermal imaging camera for canopy temperature monitoring'', realizado por investigadores de la Universidad Estatal de Michigan \cite{Dong2024}. En este trabajo, los autores analizan diversos métodos para monitorear la temperatura de la superficie foliar, cuyos resultados se sintetizan en la Tabla \ref{tab:comparativo_metodos}.

\FloatBarrier
\begin{table}[h!]
    \centering
    \caption{Comparativo entre métodos de medición de temperatura.}
    \label{tab:comparativo_metodos}
    \renewcommand{\arraystretch}{1.1}
    \small
    \begin{tabular}{|>{\raggedright\arraybackslash}p{2.5cm}|>{\raggedright\arraybackslash}p{4.5cm}|>{\raggedright\arraybackslash}p{4.0cm}|>{\raggedright\arraybackslash}p{4.0cm}|}
        \hline
        \textbf{Método} & \textbf{Descripción} & \textbf{Ventajas} & \textbf{Desventajas} \\
        \hline
        Termómetro Infrarrojo & Utiliza un haz infrarrojo para medir la radiación/energía reflejada de una superficie objetivo. & 
        \begin{itemize}[topsep=0pt, itemsep=0pt, partopsep=0pt, parsep=0pt, leftmargin=1.2em]
            \item Método sin contacto
            \item Amplia disponibilidad en el mercado
            \item Preciso con certificado de calibración
        \end{itemize} & 
        \begin{itemize}[topsep=0pt, itemsep=0pt, partopsep=0pt, parsep=0pt, leftmargin=1.2em]
            \item Problemas con la emisividad
            \item Errores aumentan con la distancia
            \item Limitado a un solo punto en la superficie de la hoja
        \end{itemize} \\
        \hline
        Sonda de Temperatura Táctil & El sensor de temperatura se sujeta a la superficie de la hoja y mide la resistencia térmica usando una fuente de corriente constante y tres cables. &
        \begin{itemize}[topsep=0pt, itemsep=0pt, partopsep=0pt, parsep=0pt, leftmargin=1.2em]
            \item Bajo costo
            \item Alta precisión para mediciones puntuales
        \end{itemize} &
        \begin{itemize}[topsep=0pt, itemsep=0pt, partopsep=0pt, parsep=0pt, leftmargin=1.2em]
            \item Puede alterar el microambiente natural de la planta
            \item Problemas con hojas pequeñas, jóvenes o muy delgadas
            \item Fluctuaciones de temperatura en el cable del sensor
        \end{itemize} \\
        \hline
        Cámara de Imagen Térmica & Utilizada ampliamente en agricultura para monitorear la salud de las plantas, programación de riego, detección de enfermedades, estimación de rendimiento, etc. &
        \begin{itemize}[topsep=0pt, itemsep=0pt, partopsep=0pt, parsep=0pt, leftmargin=1.2em]
            \item Observación continua de imágenes térmicas
            \item Información útil sobre el estrés hídrico y la salud de las plantas
        \end{itemize} &
        \begin{itemize}[topsep=0pt, itemsep=0pt, partopsep=0pt, parsep=0pt, leftmargin=1.2em]
            \item Cámaras térmicas continúas limitadas y costosas
            \item Método más complejo y caro en comparación con otros métodos
        \end{itemize} \\
        \hline
    \end{tabular}
\end{table}
\FloatBarrier

De acuerdo con la tabla anterior, se destacan las ventajas de la cámara de imagen térmica, que permite monitorear la temperatura de las plantas de manera no invasiva. Por lo anterior se propone utilizar esta tecnología para el proyecto. No obstante, debido a los altos costos de las cámaras térmicas, se busca emplear hardware de bajo costo, con la expectativa de obtener resultados que permitan cumplir los objetivos de esta investigación.

% --- CÓDIGO PARA LA FIGURA 1 ---
\begin{figure}[h!]
    \centering
    \caption{Imagen termográfica de una planta con cámara MLX90640.}
    \includegraphics[width=0.7\textwidth]{img/termo.png} 
    \label{fig:termografia_mlx}
\end{figure}

En la agricultura, la IRT se ha consolidado como una herramienta potente para la gestión hídrica. La relación es directa: el estrés por falta de agua provoca un cierre estomático que reduce la transpiración y, en consecuencia, eleva la temperatura de la planta \cite{GarciaTejero2015}. A pesar de su potencial, uno de los principales desafíos para su adopción ha sido el alto costo de los equipos comerciales. Sin embargo, investigaciones recientes en Estados Unidos, lideradas por la Universidad Estatal de Michigan, han demostrado la viabilidad de utilizar hardware de bajo costo, como cámaras térmicas compactas acopladas a microcomputadoras, para monitorear la temperatura del dosel con precisión suficiente para aplicaciones agrícolas, incluyendo el arándano \cite{Dong2024}.

En el contexto latinoamericano, también se ha validado su eficacia. Un estudio conjunto entre el Instituto Federal de Espírito Santo en Brasil y la Universidad de Georgia en Estados Unidos, demostró que la termografía diferencia eficazmente entre plantas de cítricos con distintos niveles de riego en invernadero \cite{Vieira2021}. Asimismo, en Chile, investigadores de la Universidad de Concepción validaron el uso de índices térmicos como el CWSI (Índice de Estrés Hídrico del Cultivo) como un indicador fiable del estado hídrico en frutales \cite{Quezada2020}. Esta tecnología es particularmente relevante para el arándano, un cultivo cuyo sistema radicular superficial lo hace muy susceptible a las fluctuaciones de agua, según lo documentado por el Instituto de Investigaciones Agropecuarias de Chile \cite{Morales2017}. Una gestión hídrica inadecuada en este cultivo impacta negativamente el rendimiento y la calidad de la fruta, haciendo fundamental contar con métodos de detección de estrés.

Este tema ha cobrado relevancia en Colombia, como lo demuestran los avances a nivel de software para el análisis de imágenes infrarrojas realizados por la Universidad del Valle, en su estudio sobre el uso de la termografía en América Latina \cite{Aux2022}. Paralelamente, la industria del arándano en el país ha crecido notablemente, con un aumento promedio anual del 9.1\% desde el año 2000 \cite{Aimeth2018}. Este crecimiento se ha acelerado desde 2019, siendo Boyacá uno de los departamentos líderes en la producción \cite{Blanco2023}. La importancia económica del cultivo es subrayada en un estudio de la Universidad de Cundinamarca, donde se destaca su gran potencial de exportación \cite{Augusto2023}. En este contexto, la Universidad Pedagógica y Tecnológica de Colombia menciona que la variedad Biloxi es la principal sembrada en el Altiplano Cundiboyacense debido a su resistencia, buen calibre y producción \cite{QuintanaReina2020}.

En resumen, los estudios internacionales y regionales demuestran que la termografía es una tecnología con gran potencial para monitorear el estado hídrico de las plantas. Su aplicación en el arándano Biloxi ofrece una herramienta útil para detectar el estrés hídrico y optimizar el riego. Sin embargo, la aplicación de esta tecnología sigue estando limitada debido a los altos costos del hardware especializado. La combinación del crecimiento de la industria del arándano en Colombia y los avances en hardware de bajo costo ofrece una oportunidad clara para desarrollar soluciones tecnológicas que mejoren la producción y protección del cultivo en el país.

\section{Línea de Investigación}
\label{subsec:just_linea_inv}

La adscripción de este proyecto a la línea de investigación \textbf{Aprendizaje, conocimiento, tecnologías, comunicación y digitalización} se justifica plenamente por la naturaleza y los objetivos de la solución propuesta. El sistema de detección de estrés hídrico no es un fin en sí mismo, sino una herramienta integral que aborda sistemáticamente cada uno de los pilares de esta línea:

\begin{description}
    \item [Tecnologías y Digitalización:] 
    El núcleo del proyecto es la \emph{aplicación de tecnología} (hardware de bajo costo como sensores IRT y microcontroladores) y el desarrollo de software para la \emph{digitalización} de un proceso agrícola. Se captura un fenómeno físico y análogo (la temperatura foliar de la planta) y se transforma en datos digitales estructurados y analizables, lo cual es la esencia de la agricultura de precisión y la Industria 4.0.

    \item [Conocimiento:] 
    El sistema no se limita a recolectar datos; su función principal es procesar esa información digital para generar \emph{conocimiento} nuevo y accionable. Transforma lecturas térmicas brutas en un indicador claro y comprensible (un reporte de estado hídrico) que representa el estado fisiológico de la planta. Este es un conocimiento al que el agricultor no podría acceder por simple observación.

    \item [Comunicación:] 
    El resultado de este procesamiento se convierte en un acto de \emph{comunicación} fundamental. El sistema actúa como un puente que ``traduce'' el estado fisiológico del cultivo y \emph{comunica} eficazmente un reporte al agricultor. Se supera así la barrera de la detección visual tardía, estableciendo un nuevo canal de comunicación directo entre la planta y el productor.

    \item [Aprendizaje:] 
    Finalmente, la iniciativa cierra el ciclo del \emph{aprendizaje}. Al recibir esta información (comunicación) derivada del conocimiento (procesamiento), el agricultor \emph{aprende} en tiempo real sobre las necesidades hídricas de su cultivo. Este aprendizaje validado por datos fomenta una cultura de gestión eficiente, optimiza el uso de recursos y permite tomar decisiones informadas para mejorar la productividad.
\end{description}

En conjunto, el proyecto es un ejemplo claro de cómo la tecnología y la digitalización facilitan la generación de conocimiento, mejoran la comunicación y promueven un aprendizaje práctico y continuo en el sector agrícola.

\section{Planteamiento del Problema y Pregunta de Investigación}
\label{sec:planteamiento_problema}

La producción de arándanos en Colombia está en una fase de expansión significativa, impulsada por una alta demanda en los mercados internacionales. La variedad Biloxi se ha destacado como una de las más cultivadas en el altiplano cundiboyacense debido a su adaptabilidad y calidad de fruto \cite{QuintanaReina2020}. La relevancia económica de este cultivo se refleja en el notable crecimiento de las exportaciones. En 2024, estas sumaron US\$3,29 millones, lo que representó una cifra histórica que se elevó 85\,\% frente a 2023 y puso fin a tres años consecutivos de caídas \cite{DANE2025}.

A pesar de su potencial, el arándano es un cultivo particularmente sensible a las condiciones hídricas. Su sistema radicular se caracteriza por ser fibroso, superficial y carente de pelos absorbentes, lo que limita su capacidad para explorar el perfil del suelo en busca de agua \cite{Morales2017}. Esta característica lo hace extremadamente vulnerable al estrés hídrico, tanto por déficit como por exceso, lo que puede provocar una reducción drástica en el crecimiento de la planta, el calibre de los frutos y, en última instancia, el rendimiento total del cultivo. Sin un manejo preciso del riego, los productores enfrentan un riesgo constante de pérdidas económicas significativas.

Fisiológicamente, las plantas responden al déficit de agua cerrando sus estomas para reducir la pérdida de agua por transpiración. La transpiración es un mecanismo de enfriamiento natural; por lo tanto, su disminución provoca un aumento medible en la temperatura de la superficie de las hojas \cite{GarciaTejero2015}. Este cambio térmico es un indicador temprano y fiable del estrés hídrico, a menudo detectable antes de que los síntomas visuales, como la marchitez, sean evidentes \cite{Pineda2021}.

La termografía infrarroja (IRT) es una técnica no invasiva que permite detectar estas variaciones de temperatura y, por consiguiente, monitorear el estado hídrico de los cultivos para optimizar el riego \cite{Dong2024, Vieira2021}. Su aplicación en la agricultura de precisión es una herramienta valiosa para una gestión más sostenible y automatizada. Sin embargo, su adopción se ha visto limitada por el alto costo de los equipos termográficos, lo que los hace inaccesibles para muchos agricultores e investigadores \cite{GarciaTejero2015}.

Considerando la vulnerabilidad del arándano Biloxi al estrés hídrico y la barrera económica de las tecnologías de monitoreo existentes, surge la necesidad de explorar alternativas más asequibles. Esto conduce a la siguiente pregunta de investigación: ¿Cómo se puede desarrollar un sistema que utilice hardware de bajo costo para la detección aproximada del estrés hídrico en plantas de arándano Biloxi mediante termografía?


\section{Objetivo General y Objetivos Específicos}
\subsection{Objetivo General}
Desarrollar un sistema basado en termografía infrarroja (IRT) y hardware de bajo costo para la detección del estrés hídrico en plantas de arándano Biloxi.

\subsection{Objetivos Específicos}
\begin{enumerate}
    \item Identificar y documentar los requisitos funcionales y no funcionales del sistema.
    \item Modelar la arquitectura del sistema mediante la creación de diagramas UML.
    \item Integrar el hardware del módulo termográfico para la recolección de datos.
    \item Desarrollar el software que recolecte los datos del módulo termográfico para su posterior procesamiento.
    \item Evaluar la generación de un reporte de estado hídrico que diferencie entre plantas de arándano Biloxi con riego óptimo y con déficit, a partir de los datos recopilados por el sistema.
\end{enumerate}

\section{ Alcance e Impacto del Proyecto }
\label{sec:impacto}

El presente proyecto busca ser un apoyo a la producción de arándanos a través de la implementación de tecnologías de precisión \emph{asequibles}, como la termografía de bajo costo, para monitorear y gestionar la salud de los cultivos. Esto podría permitir a los agricultores minimizar las pérdidas en las cosechas y, con ello, contribuir a ``satisfacer los desafíos de seguridad alimentaria local, regional y global del siglo XXI'' \cite{Vargas2021}.

La agricultura de precisión, apoyada en tecnologías como la termografía, es fundamental para una producción más sostenible, ya que permite optimizar el uso de recursos críticos como el agua \cite{Pineda2021}. La capacidad de monitorear el estado hídrico de las plantas en tiempo real posibilita una gestión del riego más eficiente, aplicando agua solo cuando y donde es realmente necesario \cite{GarciaTejero2015}. Este enfoque no solo promueve la conservación del recurso hídrico, sino que también contribuye a que los agricultores sean más resilientes y competitivos \cite{Dong2024}.

Este proyecto está alineado con los Objetivos de Desarrollo Sostenible (ODS) de las Naciones Unidas \cite{UNDPsf}. Específicamente, responde al \textbf{ODS 9 (Industria, Innovación e Infraestructura)}, al buscar promover la inversión y la innovación tecnológica en el sector agrícola. También contribuye directamente al \textbf{ODS 12 (Producción y Consumo Responsables)}, al fomentar una gestión eficiente de los recursos naturales mediante la implementación de prácticas agrícolas sostenibles.

En el plano local, la implementación de este proyecto puede tener un impacto significativo en la seguridad alimentaria y la economía regional. Al reducir las pérdidas económicas en la producción, se fortalecen las cadenas de suministro y se mejora la competitividad de un cultivo de alta importancia para el país \cite{Aimeth2018, Augusto2023}. Además, el acceso a tecnologías de precisión de bajo costo ayuda a reducir la brecha digital, democratizando herramientas avanzadas que de otro modo serían inaccesibles para pequeños y medianos productores, lo cual se alinea nuevamente con las metas del ODS 9.



\section{Metodología}
\label{sec:metodologia}

Para el desarrollo de la investigación se empleará una metodología mixta, que integra enfoques cuantitativos y cualitativos para obtener una comprensión más profunda del fenómeno estudiado. Este método permite combinar la recolección y análisis de datos numéricos con la interpretación de observaciones cualitativas, fortaleciendo así la validez de los resultados al compensar las limitaciones inherentes de cada enfoque por separado \cite{HamuiSutton2013}.

En el contexto de este proyecto, el componente cuantitativo se centrará en la recolección y análisis de los datos de temperatura superficial de la canopia, obtenidos a través del módulo de cámara térmica. El componente cualitativo consistirá en el registro y la descripción de los indicadores visuales del estado de la planta, como la turgencia de las hojas y el vigor general, bajo diferentes regímenes de riego. La integración de ambos tipos de datos permitirá establecer una relación más robusta entre las lecturas termográficas y el nivel de estrés hídrico aplicado, comprendiendo los factores que puedan influir en la efectividad del sistema.

Para la construcción del sistema, se utilizarán elementos del marco de trabajo ágil Scrum. Este enfoque se caracteriza por su flexibilidad, adaptabilidad y enfoque en la entrega de valor de manera incremental a lo largo del proyecto \cite{SCRUMstudy2022}. Se organizará el trabajo en ciclos cortos o Sprints, gestionando las tareas a través de un Product Backlog para asegurar un desarrollo organizado y eficiente del prototipo.


\section{Marcos de Referencia}
\subsection{Marco Teórico}
\label{subsec:marco_teorico}

Este marco teórico sienta las bases conceptuales y científicas sobre las que se construye el presente proyecto. Se abordan tres pilares fundamentales: las características del cultivo de arándano variedad Biloxi y su sensibilidad hídrica, los mecanismos fisiológicos del estrés hídrico en las plantas y su detección, y los principios de la termografía infrarroja como herramienta de monitoreo en la agricultura de precisión, con énfasis en soluciones de bajo costo.

\subsubsection{El Cultivo de Arándano (Vaccinium corymbosum L.)}
\label{ssubsec:arandano}

El arándano (Vaccinium corymbosum L.) es un frutal cuya popularidad y demanda han crecido exponencialmente a nivel mundial, impulsando su expansión en regiones con condiciones agroecológicas adecuadas, como el altiplano cundiboyacense en Colombia \cite{Aimeth2018, Augusto2023}. Dentro de las diversas variedades, la Biloxi se ha destacado por su adaptabilidad a climas con bajo requerimiento de frío invernal y por la calidad de su fruto, convirtiéndose en una opción preferente para los productores de la región \cite{QuintanaReina2020}.

Una característica agronómica crucial del arándano es su sistema radicular. Este es típicamente superficial, fibroso y carente de pelos absorbentes, lo que implica una capacidad limitada para explorar el perfil del suelo en busca de agua y nutrientes \cite{Morales2017}. Esta particularidad morfológica hace que el cultivo sea especialmente sensible a las fluctuaciones en la disponibilidad hídrica del suelo. Tanto el déficit como el exceso de agua pueden impactar negativamente el desarrollo vegetativo, el calibre de los frutos y, en consecuencia, el rendimiento general de la cosecha \cite{Morales2017, Salgado2018}. Por ello, una gestión precisa y oportuna del riego es indispensable para asegurar la viabilidad económica y la sostenibilidad de la producción de arándano.

\subsubsection{Estrés Hídrico en Plantas}
\label{ssubsec:estres_hidrico}

El estrés hídrico ocurre cuando la demanda de agua para la transpiración de la planta excede la capacidad de absorción de agua por las raíces, o cuando el contenido de agua en el suelo es insuficiente \cite{Jones2004}. Fisiológicamente, las plantas responden a esta condición mediante una serie de mecanismos adaptativos para conservar agua. Uno de los más importantes y tempranos es el cierre estomático \cite{Rinza2021}. Los estomas son poros microscópicos, principalmente en la superficie de las hojas, que regulan el intercambio gaseoso (CO2 y O2) y la pérdida de agua por transpiración. Al cerrarse, la planta reduce significativamente la pérdida de vapor de agua hacia la atmósfera.

La transpiración, además de su rol en el transporte de agua y nutrientes, funciona como un mecanismo de termorregulación, enfriando la superficie foliar mediante la evaporación del agua \cite{GarciaTejero2015}. Por lo tanto, la reducción de la tasa transpiratoria debido al cierre estomático provoca un aumento medible en la temperatura de la superficie de las hojas \cite{Jones2004, Pineda2021}. Este incremento térmico es un indicador fisiológico directo y temprano del estrés hídrico, a menudo detectable antes de que aparezcan síntomas visuales como la marchitez o la decoloración foliar \cite{Pineda2021, Rinza2021}. La detección precoz de este cambio térmico es clave para intervenir con el riego antes de que el estrés cause daños irreversibles o mermas significativas en la producción.

\subsubsection{Termografía Infrarroja (IRT) en la Agricultura de Precisión}
\label{ssubsec:termografia}

La termografía infrarroja (IRT) es una técnica no invasiva que permite capturar la radiación infrarroja emitida por la superficie de los objetos y convertirla en una imagen visible (termograma), donde cada color o nivel de gris representa una temperatura \cite{Jones2004}. Su aplicación en fisiología vegetal y ecofisiología se basa en la correlación directa entre la temperatura superficial de la hoja y diversos procesos fisiológicos, especialmente la transpiración y el estado hídrico \cite{Jones2004, GarciaTejero2015}.

En la agricultura de precisión, la IRT se ha consolidado como una herramienta valiosa para monitorear el estrés hídrico en diversos cultivos \cite{PobleteEcheverria2023, Vieira2021}. Permite evaluar la variabilidad espacial y temporal del estado hídrico dentro de una parcela, facilitando una gestión del riego más eficiente y localizada \cite{GarciaTejero2015}. Uno de los índices más utilizados derivados de la termografía es el Índice de Estrés Hídrico del Cultivo (CWSI, por sus siglas en inglés), propuesto originalmente por Idso et al. \cite{Idso1981}. El CWSI normaliza la diferencia entre la temperatura del dosel (Tc) y la temperatura del aire (Ta) utilizando referencias de temperaturas foliares en condiciones de máxima transpiración (límite inferior o T\textsubscript{wet}) y mínima transpiración (límite superior o T\textsubscript{dry}), proporcionando un indicador cuantitativo del nivel de estrés (valores entre 0 y 1) \cite{Quezada2020}.

A pesar de su potencial demostrado, la adopción generalizada de la IRT en la agricultura se ha visto históricamente limitada por el alto costo de las cámaras termográficas de grado científico o industrial \cite{GarciaTejero2015}. Sin embargo, avances recientes en la tecnología de sensores han propiciado la aparición de matrices de sensores infrarrojos de bajo costo, como el MLX90640 \cite{Melexis2021}, que, aunque con menor resolución espacial que las cámaras de alta gama, ofrecen una alternativa viable para aplicaciones específicas de monitoreo \cite{Dong2024}. Investigaciones como la de Dong et al. \cite{Dong2024} han explorado y validado el uso de estas tecnologías de bajo costo para el monitoreo de la temperatura del dosel en aplicaciones agrícolas, abriendo nuevas posibilidades para desarrollar herramientas de agricultura de precisión más asequibles y accesibles para un mayor número de productores. Este proyecto se inscribe en esta línea, buscando aprovechar el potencial de la termografía de bajo costo para abordar el desafío del manejo hídrico en el cultivo de arándano Biloxi.

\subsection{Marco Legal}
\label{subsec:marco_legal}

El desarrollo del proyecto se adhiere a la normativa colombiana, cubriendo aspectos clave de propiedad intelectual, protección de datos y seguridad digital.

\subsubsection{Propiedad Intelectual, Derechos de Autor y Licenciamiento}
La gestión de la propiedad intelectual se rige por el \textbf{Acuerdo No. 04 de 2018} de la Universidad de Cundinamarca \cite{UDEC2018}. Conforme a este y a los acuerdos del proyecto:
\begin{itemize}
    \item \textbf{Derechos Morales:} La autoría moral del software y del presente documento corresponde a los estudiantes y a la directora del proyecto, reconociendo su contribución intelectual fundamental \cite{UDEC2018}.
    \item \textbf{Derechos Patrimoniales:} Los derechos patrimoniales sobre el software desarrollado son cedidos a la \textbf{Universidad de Cundinamarca}, conforme a lo estipulado en el marco del proyecto y el estatuto \cite{UDEC2018}.
    \item \textbf{Licencia del Software:} El código fuente del software desarrollado se libera bajo la \textbf{Licencia MIT}, permitiendo su uso, modificación y distribución de forma abierta, fomentando la colaboración y la innovación.
    \item \textbf{Uso Académico del Documento:} Se concede a la Universidad una licencia no exclusiva para el uso académico del presente documento, permitiendo su inclusión en repositorios institucionales para consulta y preservación \cite{UDEC2018}.
\end{itemize}

\subsubsection{Protección de Datos Personales}
El diseño del sistema respeta la \textbf{Ley Estatutaria 1581 de 2012} \cite{Congreso2012}, garantizando la privacidad y seguridad de los datos de usuario. En la fase actual, se aplican buenas prácticas de seguridad y se asegura el ejercicio de los derechos del titular. Se contempla que una futura explotación comercial requeriría cumplir obligaciones adicionales (ej. registro RNBD, política de tratamiento).

\subsubsection{Seguridad Digital}
Las medidas de seguridad del software (autenticación, control de acceso) se alinean con la \textbf{Ley 1273 de 2009} sobre delitos informáticos \cite{Congreso2009}, protegiendo la integridad de los datos y previniendo accesos no autorizados al sistema.   % Capítulo I: Informe de Investigación
% Capítulo II: Documentación del Software
\chapter{DOCUMENTACIÓN SOFTWARE}

% =================================================
% =================================================

\section{Plan de Proyecto}
Texto del plan de proyecto...

% =================================================
% =================================================

\section{Arquitectura del Software}

Describe la estructura global del sistema, mostrando cómo se organizan e interconectan los componentes del frontend y backend.

\subsection{Desarrollo del Frontend}
Documenta la implementación de la interfaz de usuario, detallando las tecnologías utilizadas y el diseño de la experiencia de usuario.

\subsection{Desarrollo del Backend}
Explica la lógica del servidor, la estructura de las APIs, la gestión de bases de datos y la implementación de la lógica de negocio que sustenta la aplicación.

\subsection{Integración Frontend-Backend}
Detalla los mecanismos y protocolos (por ejemplo, REST o WebSockets) que permiten la comunicación y sincronización de datos entre la interfaz y el servidor.

% =================================================
% =================================================

\section{Determinación de Requerimientos}
% --- RF01 ---
\begin{table}[H]
	\centering
	\caption{\textit{Requerimiento Funcional RF01: Registro}}
	\label{tab:rf01}
	\begin{tabular}{@{}ll@{}} % l = left-aligned text, @{} removes side padding
	  \toprule
	  \textbf{Identificador} & RF01 \\
	  \midrule
	  \textbf{Nombre} & Registro \\
	  \textbf{Roles} & Administrador, Usuario \\
	  \textbf{Descripción} & El administrador puede registrarse en el sistema de detección \\ 
						   & proporcionando los datos solicitados en el formulario de registro. \\
						   & Para que un usuario se pueda registrar, debe solicitar el código de \\
						   & acceso proporcionado por el administrador para poder realizar \\
						   & correctamente el registro. \\
	  \bottomrule
	\end{tabular}
  \end{table}
  
  % --- RF02 ---
  \begin{table}[H]
	\centering
	\caption{\textit{Requerimiento Funcional RF02: Inicio de sesión}}
	\label{tab:rf02}
	\begin{tabular}{@{}ll@{}} 
	  \toprule
	  \textbf{Identificador} & RF02 \\
	  \midrule
	  \textbf{Nombre} & Inicio de sesión \\
	  \textbf{Roles} & Administrador, Usuario \\
	  \textbf{Descripción} & Permite a los diferentes roles acceder al sistema de detección con \\
						   & sus credenciales (usuario y contraseña), estas deben ser correctas \\
						   & para su acceso. Al finalizar, se podrá cerrar sesión. En caso tal de \\
						   & olvidar la contraseña, se tendrá la opción para recuperarla. \\
	  \bottomrule
	\end{tabular}
  \end{table}
  
  % --- RF03 ---
  \begin{table}[H]
	\centering
	\caption{\textit{Requerimiento Funcional RF03: CRUD cámara}}
	\label{tab:rf03}
	\begin{tabular}{@{}ll@{}} 
	  \toprule
	  \textbf{Identificador} & RF03 \\
	  \midrule
	  \textbf{Nombre} & CRUD cámara \\
	  \textbf{Roles} & Administrador \\
	  \textbf{Descripción} & Se podrán agregar módulos termográficos al sistema de detección \\
						   & para poder recibir y procesar los datos que estas envíen. Además de \\
						   & visualizar y actualizar el estado de cada módulo termográfico. En \\
						   & caso de ser necesario, se podrá eliminar la cámara del sistema de \\
						   & detección. \\
	  \bottomrule
	\end{tabular}
  \end{table}
  
  % --- RF04 ---
  \begin{table}[H]
	\centering
	\caption{\textit{Requerimiento Funcional RF04: CRUD persona}}
	\label{tab:rf04}
	\begin{tabular}{@{}ll@{}} 
	  \toprule
	  \textbf{Identificador} & RF04 \\
	  \midrule
	  \textbf{Nombre} & CRUD persona \\
	  \textbf{Roles} & Administrador, Usuario \\
	  \textbf{Descripción} & Los distintos roles podrán actualizar sus datos personales o \\
						   & contraseña. También podrán eliminar su cuenta. \\
	  \bottomrule
	\end{tabular}
  \end{table}
  
  % --- RF05 ---
  \begin{table}[H]
	\centering
	\caption{\textit{Requerimiento Funcional RF05: Módulo de mediciones}}
	\label{tab:rf05}
	\begin{tabular}{@{}ll@{}} 
	  \toprule
	  \textbf{Identificador} & RF05 \\
	  \midrule
	  \textbf{Nombre} & Módulo de mediciones \\
	  \textbf{Roles} & Administrador, Usuario \\
	  \textbf{Descripción} & El sistema de detección recopilará y mostrará los datos de las \\
						   & mediciones tomadas por otros sensores por medio de gráficas y un \\
						   & histórico. \\
	  \bottomrule
	\end{tabular}
  \end{table}
  
  % --- RF06 ---
  \begin{table}[H]
	\centering
	\caption{\textit{Requerimiento Funcional RF06: Módulo de procesamiento}}
	\label{tab:rf06}
	\begin{tabular}{@{}ll@{}} 
	  \toprule
	  \textbf{Identificador} & RF06 \\
	  \midrule
	  \textbf{Nombre} & Módulo de procesamiento \\
	  \textbf{Roles} & Administrador, Usuario \\
	  \textbf{Descripción} & El sistema de detección mostrará los datos recopilados por los \\
						   & módulos de cámara por medio de gráficas. Además, se debe mostrar \\
						   & el estado de cada planta y el histórico de datos de todas las plantas. \\
						   & El procesamiento de los datos térmicos indicará el estado de cada \\
						   & planta. Los roles podrán actualizar el estado proporcionado por el \\
						   & sistema de ser necesario. \\
	  \bottomrule
	\end{tabular}
  \end{table}
  
  % --- RF07 ---
  \begin{table}[H]
	\centering
	\caption{\textit{Requerimiento Funcional RF07: Reportes}}
	\label{tab:rf07}
	\begin{tabular}{@{}ll@{}} 
	  \toprule
	  \textbf{Identificador} & RF07 \\
	  \midrule
	  \textbf{Nombre} & Reportes \\
	  \textbf{Roles} & Administrador, Usuario \\
	  \textbf{Descripción} & Los diferentes roles podrán generar reportes sobre el estado de las \\
						   & plantas en formato PDF con base en los datos recopilados por los \\
						   & módulos de cámara y/o por el módulo de procesamiento. Se podrá \\
						   & escoger distintos filtros (una o varias plantas, lapsos de tiempo). \\
	  \bottomrule
	\end{tabular}
  \end{table}
  
  % --- RF08 ---
  \begin{table}[H]
	\centering
	\caption{\textit{Requerimiento Funcional RF08: Notificaciones}}
	\label{tab:rf08}
	\begin{tabular}{@{}ll@{}} 
	  \toprule
	  \textbf{Identificador} & RF08 \\
	  \midrule
	  \textbf{Nombre} & Notificaciones \\
	  \textbf{Roles} & Administrador, Usuario (como receptores) \\
	  \textbf{Descripción} & Se deben enviar notificaciones por correo electrónico a los distintos \\
						   & roles en el cual se puedan alertar sobre cambios de estado en las \\
						   & plantas y enviar notificaciones de seguridad. \\
	  \bottomrule
	\end{tabular}
  \end{table}
  
  % --- RF09 (Nuevo) ---
  \begin{table}[H]
	\centering
	\caption{\textit{Requerimiento Funcional RF09: Gestionar Observaciones}}
	\label{tab:rf09}
	\begin{tabular}{@{}ll@{}} 
	  \toprule
	  \textbf{Identificador} & RF09 \\ 
	  \midrule
	  \textbf{Nombre} & Gestionar Observaciones \\ 
	  \textbf{Roles} & Administrador, Usuario\\ 
	  \textbf{Descripción} & Permite a los usuarios registrar y consultar observaciones \\
						   & cualitativas sobre el estado de las plantas.\\
						   & Se utiliza una plantilla estandarizada para anotar aspectos visuales \\
						   & (color, textura, uniformidad, daños), asignar una calificación \\
						   & subjetiva y añadir notas, complementando los datos cuantitativos \\
						   & para la metodología mixta y el procesamiento de datos. \\
	  \bottomrule
	\end{tabular}
  \end{table}
  
  % --- RNF01 ---
  \begin{table}[H]
	\centering
	\caption{\textit{Requerimiento No Funcional RNF01: Seguridad}}
	\label{tab:rnf01}
	\begin{tabular}{@{}ll@{}} 
	  \toprule
	  \textbf{Identificador} & RNF01 \\ 
	  \midrule
	  \textbf{Nombre} & Seguridad \\ 
	  \textbf{Roles} & N/A (Aplica al Sistema) \\ % Ajustado rol para RNF
	  \textbf{Descripción} & El sistema de detección debe cumplir con los lineamientos y leyes \\
						   & establecidos para la protección, integridad y disponibilidad de los \\
						   & datos (Ley 1581 de 2012). \\
	  \bottomrule
	\end{tabular}
  \end{table}
  
  % --- RNF02 ---
  \begin{table}[H]
	\centering
	\caption{\textit{Requerimiento No Funcional RNF02: Copia de seguridad}}
	\label{tab:rnf02}
	\begin{tabular}{@{}ll@{}} 
	  \toprule
	  \textbf{Identificador} & RNF02 \\ 
	  \midrule
	  \textbf{Nombre} & Copia de seguridad \\ 
	  \textbf{Roles} & N/A (Aplica al Sistema) \\ % Ajustado rol para RNF
	  \textbf{Descripción} & Se debe asegurar un respaldo de los datos en caso de presentarse \\
						   & alguna eventualidad no se vulnere la integridad de los datos. Esta \\
						   & copia de seguridad se debe hacer de forma automática y semanal. \\
	  \bottomrule
	\end{tabular}
  \end{table}

% =================================================
% =================================================

\section{Especificación del Diseño}

\subsection{Modelo de Entidad-Relación (MER)}
\begin{figure}[H]
    \centering
    \caption{Diagrama Entidad-Relación del Sistema.}
    \label{fig:der}
    \includegraphics[width=1\textwidth]{UML/Otros/Diagrama Entidad Relacion.png}
\end{figure}

El Modelo Entidad-Relación (MER), presentado en la Figura \ref{fig:der}, define la estructura lógica de la base de datos diseñada para el sistema, organizando la información necesaria para el funcionamiento de la aplicación. Las tablas se han agrupado lógicamente (y coloreado en el diagrama) según su propósito principal para facilitar su comprensión: Diccionarios de Datos (celeste), Datos de Usuarios (blanco), Datos de Cultivos (naranja), Datos de Dispositivos (morado), Datos de Plantas (verde) y Auditoría (rojo). A continuación, se describe el propósito y las relaciones clave de cada tabla dentro de estos grupos.

\subsubsection*{Grupo 1: Diccionarios de Datos (Celeste)}
Este grupo contiene tablas auxiliares que definen tipos o categorías reutilizables en otras partes del sistema.
\begin{itemize}
    \item \texttt{TableRelation}: Almacena los nombres de las tablas principales de la base de datos. Su propósito es permitir que la tabla \texttt{Status} pueda referenciar a qué tabla pertenece un estado específico, ofreciendo un mecanismo de categorización de estados.
    \item \texttt{Status}: Define un conjunto genérico de estados posibles (ej. 'Activo', 'Inactivo', 'Pendiente') que pueden ser aplicados a diferentes entidades del sistema como invitaciones, plantas, dispositivos u observaciones. Se relaciona con \texttt{TableRelation} para indicar el contexto de cada estado y es referenciada por múltiples tablas (\texttt{CropInvitation}, \texttt{PlantData}, \texttt{DeviceData}, \texttt{DeviceActivation}, \texttt{PlantStateHistory}, \texttt{PlantObservation}) mediante claves foráneas.
\end{itemize}

\subsubsection*{Grupo 2: Datos de Usuarios (Blanco)}
Este grupo gestiona la información y la autenticación de los usuarios del sistema.
\begin{itemize}
    \item \texttt{Person}: Tabla central para los usuarios. Almacena información personal básica (nombre, apellido), credenciales de acceso (email, contraseña cifrada), rol (\texttt{isAdmin}), preferencias de notificación, y timestamps relevantes (creación, actualización, último login, último cambio de contraseña). Se relaciona con \texttt{Crop} para indicar a qué cultivo pertenece el usuario (si no es administrador global). Es referenciada por múltiples tablas para indicar quién realizó una acción (\texttt{Crop}, \texttt{CropInvitation}, \texttt{DeviceData}, \texttt{PlantStateHistory}, \texttt{PlantObservation}, \texttt{RefreshToken}) y por las tablas de auditoría.
    \item \texttt{ChangePassword}: Gestiona las solicitudes de restablecimiento de contraseña, almacenando un token temporal, su fecha de expiración y la referencia al usuario (\texttt{PersonId}) que lo solicitó.
    \item \texttt{RefreshToken}: Almacena los tokens de refresco utilizados para mantener las sesiones de usuario activas de forma segura, junto con información del dispositivo y la IP asociada. Se relaciona con \texttt{Person}.
    \item \texttt{FailedLoginAttempt}: Registra los intentos fallidos de inicio de sesión para monitoreo de seguridad, incluyendo la IP, información del dispositivo y la referencia al usario (\texttt{PersonId}).
\end{itemize}

\subsubsection*{Grupo 3: Datos de Cultivos (Naranja)}
Define la información general sobre los cultivos gestionados.
\begin{itemize}
    \item \texttt{Crop}: Representa un cultivo específico, almacenando su nombre, dirección, ciudad y el identificador del usuario administrador de ese cultivo (\texttt{adminUserId}, FK a \texttt{Person}). Es una entidad central referenciada por \texttt{Person}, \texttt{CropInvitation}, \texttt{PlantData}, \texttt{DeviceData}, \texttt{SensorData}, \texttt{ThermalData} y las tablas de auditoría para contextualizar los datos.
    \item \texttt{CropInvitation}: Gestiona los códigos de invitación que permiten a nuevos usuarios unirse a un cultivo existente. Almacena el código, fechas de validez, estado (\texttt{StatusId}), quién la creó (\texttt{createdBy}, FK a \texttt{Person}), quién la usó (\texttt{usedBy}, FK a \texttt{Person}) y a qué cultivo pertenece (\texttt{CropId}).
\end{itemize}

\subsubsection*{Grupo 4: Datos de Dispositivos (Morado)}
Administra la información de los dispositivos de hardware (cámaras térmicas y otros sensores) utilizados para la recolección de datos.
\begin{itemize}
    \item \texttt{DeviceData}: Registra cada dispositivo físico, incluyendo su nombre, descripción, intervalo de recolección de datos (\texttt{dataCollectionTime}), estado (\texttt{StatusId}), fechas y usuarios de registro/actualización (FKs a \texttt{Person}). Puede estar asociado a un cultivo (\texttt{CropId}) y opcionalmente a una planta específica (\texttt{PlantId}). Es referenciada por \texttt{DeviceLog}, \texttt{DeviceToken} y \texttt{DeviceActivation}.
    \item \texttt{DeviceLog}: Almacena registros (logs) de eventos o errores generados por los dispositivos, referenciando al dispositivo (\texttt{deviceId}) correspondiente.
    \item \texttt{DeviceToken}: Gestiona los tokens de autenticación específicos para que los dispositivos puedan enviar datos de forma segura a la API. Se relaciona con \texttt{DeviceData}.
    \item \texttt{DeviceActivation}: Controla el proceso de activación inicial de un dispositivo mediante un código, almacenando su estado (\texttt{activationStatus}, FK a \texttt{Status}) y fechas relevantes. Se relaciona con \texttt{DeviceData}.
\end{itemize}

\subsubsection*{Grupo 5: Datos de Plantas (Verde)}
Este grupo centraliza toda la información relacionada con las plantas de arándano monitoreadas.
\begin{itemize}
    \item \texttt{PlantData}: Representa cada planta individual dentro de un cultivo. Almacena su nombre identificador, fecha de registro, estado actual (\texttt{StatusId}, ej. 'Saludable', 'No Saludable') y a qué cultivo pertenece (\texttt{CropId}). Es referenciada por \texttt{DeviceData} (opcionalmente), \texttt{PlantStateHistory}, \texttt{SensorData}, \texttt{ThermalData} y \texttt{PlantObservation}.
    \item \texttt{PlantStateHistory}: Mantiene un historial de los cambios de estado de cada planta, registrando cuándo ocurrió el cambio (\texttt{changedAt}), quién lo realizó (\texttt{changedBy}, FK a \texttt{Person}) y cuál fue el nuevo estado (\texttt{StatusId}). Se relaciona con \texttt{PlantData}.
    \item \texttt{SensorData}: Almacena las lecturas periódicas de sensores ambientales asociados a una planta (temperatura, humedad, intensidad lumínica) y opcionalmente datos climáticos de la ciudad. Incluye la fecha de registro (\texttt{recordedAt}) y referencias a la planta (\texttt{PlantId}) y al cultivo (\texttt{CropId}).
    \item \texttt{ThermalData}: Guarda los datos crudos de las imágenes termográficas (en formato JSONB \texttt{thermalImageData}) y opcionalmente una imagen RGB (\texttt{rgbImageData}). Incluye la fecha de registro (\texttt{recordedAt}) y referencias a la planta (\texttt{PlantId}) y al cultivo (\texttt{CropId}).
    \item \texttt{PlantObservation}: Registra observaciones manuales realizadas por los usuarios sobre el estado de una planta, incluyendo descripciones textuales, indicadores visuales (decoloraciones, uniformidad), notas, una calificación subjetiva y referencias a la última toma de datos (\texttt{SensorData} y \texttt{ThermalData})disponibles al momento de la observación. Se relaciona con \texttt{PlantData}, \texttt{Person} (\texttt{createdBy}) y \texttt{Status}.
\end{itemize}

\subsubsection*{Grupo 6: Auditoría (Rojo)}
Este conjunto de tablas implementa un mecanismo detallado de auditoría para rastrear cambios en la base de datos.
\begin{itemize}
    \item Tablas de Auditoría (\texttt{AuditCrop}, 
	\texttt{AuditDataTable}, 
	\texttt{AuditDevice}, 
	\texttt{AuditPerson},
	\\\texttt{AuditSensitiveData}, 
	\texttt{AuditSystemTable}): Cada una de estas tablas está diseñada para registrar modificaciones (inserciones, actualizaciones, eliminaciones) en las tablas principales correspondientes a su nombre o categoría. Almacenan información crucial como la tabla y columna modificada, el ID del registro afectado, la acción realizada ('INSERT', 'UPDATE', 'DELETE'), los valores antiguo y nuevo (cuando aplica), quién realizó el cambio (\texttt{performedBy}, FK a \texttt{Person}), cuándo (\texttt{performedAt}), desde qué IP (\texttt{performedByIp}) e información del dispositivo (\texttt{userAgent}). La tabla \texttt{AuditSensitiveData} se enfoca específicamente en el acceso o cambio de datos sensibles (ej. contraseñas, tokens), mientras que las otras auditan cambios en datos generales. Todas referencian al \texttt{cropId} para contextualizar la auditoría dentro de un cultivo específico.
\end{itemize}

% =================================================
% =================================================

\subsection{Diagramas de Casos de Uso}

Un caso de uso es una unidad coherente de funcionalidad externamente visible proporcionada por un clasificador (denominado sistema) y expresada mediante secuencias de mensajes intercambiados por el sistema y uno o más actores de la unidad del sistema \cite{Rumbaugh2007}. El propósito de un caso de uso es definir una pieza de comportamiento coherente sin revelar la estructura interna del sistema \cite{Rumbaugh2007}.

Esta sección presenta los diagramas de casos de uso que modelan las funcionalidades principales ofrecidas por el sistema. Estos diagramas ilustran, desde una perspectiva de alto nivel, cómo interactúan los diferentes actores principalmente el \texttt{Usuario}, el \texttt{Administrador} y, en ciertos escenarios, el propio \texttt{Sistema} con las funcionalidades clave del sistema (representadas por elipses). Cada diagrama está generalmente asociado a un módulo funcional o a un Requerimiento Funcional (RF) específico identificado en la especificación de requisitos, utilizando la notación estándar UML para representar actores, casos de uso, límites del sistema y relaciones como \texttt{<<Extend>>} o \texttt{<<Include>>}. El objetivo es proporcionar una visión clara del alcance funcional del sistema desde el punto de vista de sus usuarios.



\begin{figure}[H]
    \centering
    \caption{Diagrama de Casos de Uso para la Gestión de Usuarios (RF1, RF2).}
    \label{fig:casos-uso-usuarios} % Opcional: Etiqueta para referencias
    \includegraphics[width=0.8\textwidth]{UML/CasosUso/Diagrama de Casos de Uso RF1 RF2.png}
\end{figure}
La Figura \ref{fig:casos-uso-usuarios} detalla los casos de uso correspondientes al \texttt{Módulo de autenticación} del sistema, cubriendo las funcionalidades de registro e inicio de sesión (RF01 y RF02). Los actores que interactúan con este módulo son el \texttt{Usuario} y el \texttt{Administrador}, ambos representados mediante la generalización \texttt{Persona}. A continuación se describe cada caso de uso:

\subsubsection*{Caso de Uso: Solicitar código}
Permite a un \texttt{Usuario} (actuando como \texttt{Persona}) solicitar un código de acceso. Según RF01, este código es necesario para que un \texttt{Usuario} pueda registrarse y asociarse a un cultivo existente, y debe ser proporcionado previamente por un \texttt{Administrador}.

\subsubsection*{Caso de Uso: Registrar usuario}
Corresponde a la funcionalidad RF01. Permite a \texttt{Persona} crear una nueva cuenta en el sistema. El flujo varía según el rol: un \texttt{Administrador} puede registrarse directamente, mientras que un \texttt{Usuario} necesita ingresar un código de acceso válido (obtenido a través del caso de uso \texttt{Solicitar código}) para completar su registro dentro de un cultivo específico.

\subsubsection*{Caso de Uso: Iniciar sesión}
Representa la funcionalidad principal de RF02, permitiendo a \texttt{Persona} acceder al sistema mediante la validación de sus credenciales (correo electrónico y contraseña). Un inicio de sesión exitoso otorga acceso a las funcionalidades correspondientes al rol del usuario (\texttt{Administrador} o \texttt{Usuario}). Este caso de uso es la base para poder interactuar con el resto del sistema y puede ser extendido por \texttt{Cerrar sesión}.

\subsubsection*{Caso de Uso: Recuperar contraseña}
Forma parte de la funcionalidad RF02. Ofrece a \texttt{Persona} un mecanismo para restablecer su contraseña si la ha olvidado. Típicamente, esto implica un proceso de verificación a través del correo electrónico registrado para garantizar la seguridad.

\subsubsection*{Caso de Uso: Cerrar sesión}
Este caso de uso extiende (`<<Extend>>`) a \texttt{Iniciar sesión}, para completar su funcionamiento. Representa la acción explícita y opcional que realiza \texttt{Persona} para terminar de forma segura su sesión activa dentro de la aplicación, después de haber iniciado sesión y realizado otras tareas.


\begin{figure}[H]
    \centering
    \caption{Diagrama de Casos de Uso para la Gestión de Cámaras (RF3).}
    \label{fig:casos-uso-camaras}
     \includegraphics[width=0.8\textwidth]{UML/CasosUso/Diagrama de Casos de Uso RF3.png}
\end{figure}

La Figura \ref{fig:casos-uso-camaras} describe los casos de uso asociados a la gestión (CRUD - Crear, Leer, Actualizar, Borrar) de los dispositivos de cámara o módulos termográficos, funcionalidad identificada como RF03 y exclusiva para el actor \texttt{Administrador}. El diagrama presenta un caso de uso central y las operaciones específicas que extienden su funcionalidad:

\subsubsection*{Caso de Uso: CRUD Cámaras}
Representa la funcionalidad principal o el punto de acceso para que el \texttt{Administrador} gestione los módulos termográficos registrados en el sistema. Este caso de uso, descrito en RF03, se ve extendida (`<<Extend>>`) por operaciones más específicas como registrar o consultar cámaras.

\subsubsection*{Caso de Uso: Registrar cámara}
Extiende (`<<Extend>>`) la funcionalidad de \texttt{CRUD Cámaras}. Permite al \texttt{Administrador} añadir un nuevo módulo termográfico al sistema (operación Create). Esto incluye la configuración inicial del dispositivo dentro de la plataforma.

\subsubsection*{Caso de Uso: Consultar cámara}
Extiende (`<<Extend>>`) la funcionalidad de \texttt{CRUD Cámaras}. Permite al \texttt{Administrador} buscar y visualizar la información detallada y el estado actual de los módulos termográficos ya registrados en el sistema (operación Read). Este caso de uso sirve como punto de partida para otras acciones opcionales.

\subsubsection*{Caso de Uso: Editar cámara}
Extiende (`<<Extend>>`) la funcionalidad de \texttt{Consultar cámara}. Una vez que el \textit{Administrador} ha consultado los detalles de una cámara específica, tiene la opción de modificar su configuración, nombre, descripción o actualizar su estado dentro del sistema (operación Update).

\subsubsection*{Caso de Uso: Eliminar cámara}
Extiende (`<<Extend>>`) la funcionalidad de \texttt{Consultar cámara}. Después de consultar o seleccionar una cámara, el \texttt{Administrador} puede optar por eliminar permanentemente el registro de ese dispositivo del sistema (operación Delete), usualmente si el dispositivo se da de baja o ya no se utiliza.


\begin{figure}[H]
    \centering
    \caption{Diagrama de Casos de Uso para la Gestión de Perfiles (RF4).}
    \label{fig:casos-uso-perfiles}
    \includegraphics[width=0.8\textwidth]{UML/CasosUso/Diagrama de Casos de Uso RF4.png}
\end{figure}

La Figura \ref{fig:casos-uso-perfiles} detalla los casos de uso relacionados con la gestión del perfil de usuario dentro del sistema, funcionalidad descrita en RF04. Estas operaciones pueden ser realizadas tanto por el \texttt{Usuario} como por el \texttt{Administrador}, representados por la generalización \texttt{Persona}, sobre la información de su propia cuenta. El diagrama se centra en el módulo o funcionalidad \texttt{CRUD Persona}:

\subsubsection*{Caso de Uso: CRUD Persona}
Este caso de uso actúa como el punto de entrada general para que \texttt{Persona} administre la información asociada a su perfil en el sistema, tal como se indica en RF04. La funcionalidad principal que extiende (`<<Extend>>`) esta gestión es \texttt{Consultar perfil}.

\subsubsection*{Caso de Uso: Consultar perfil}
Extiende (`<<Extend>>`) la funcionalidad de \texttt{CRUD Persona}. Permite a \texttt{Persona} visualizar sus propios datos de perfil registrados en la aplicación (operación de Leer). Esta consulta es, generalmente, el paso previo necesario para poder realizar modificaciones o eliminar la cuenta.

\subsubsection*{Caso de Uso: Editar perfil}
Extiende (`<<Extend>>`) la funcionalidad de \texttt{Consultar perfil}. Una vez que \texttt{Persona} visualiza su perfil, este caso de uso le permite modificar sus datos personales registrados, como nombre, apellido, correo electrónico o preferencias de notificación (operación de Actualizar datos), de acuerdo con RF04.

\subsubsection*{Caso de Uso: Eliminar perfil}
Extiende (`<<Extend>>`) la funcionalidad de \texttt{Consultar perfil}. Habilita a \texttt{Persona} para solicitar la eliminación permanente de su cuenta y datos asociados del sistema (operación de Borrar), como lo permite RF04. Esta acción se realiza típicamente desde la vista del perfil del usuario.

\subsubsection*{Caso de Uso: Cambiar contraseña}
Extiende (`<<Extend>>`) la funcionalidad de \texttt{Consultar perfil}. Permite a \texttt{Persona} iniciar el proceso para actualizar su contraseña de acceso al sistema (operación de Actualizar contraseña), como se menciona en RF04. Usualmente, esta opción está disponible dentro de la sección de gestión o consulta del perfil.


\begin{figure}[H]
    \centering
    \caption{Diagrama de Casos de Uso para el Módulo de Mediciones (RF5).}
    \label{fig:casos-uso-mediciones}
    \includegraphics[width=0.8\textwidth]{UML/CasosUso/Diagrama de Casos de Uso RF5.png}
\end{figure}

La Figura \ref{fig:casos-uso-mediciones} presenta los casos de uso asociados al \texttt{Modulo de mediciones} del sistema. Este módulo, accesible por los actores \texttt{Usuario} y \texttt{Administrador} (generalizados como \texttt{Persona}), funciona como el panel de control principal o \textit{Dashboard} tras el inicio de sesión. A continuación, se describen los casos de uso involucrados:

\subsubsection*{Caso de Uso: Panel de control}
Este caso de uso representa la pantalla principal que visualiza \texttt{Persona} al interactuar con el módulo de mediciones. Su función primordial, relacionada con RF05, es mostrar de forma consolidada los datos clave recopilados por los sensores y cámaras, típicamente mediante gráficas de las últimas 24 horas y los últimos valores registrados. Proporciona una visión general del estado del cultivo y sirve como punto central desde el cual se puede acceder (`<<Extend>>`) a otras funcionalidades específicas de gestión.

\subsubsection*{Caso de Uso: CRUD Plantas}
Extiende (`<<Extend>>`) la funcionalidad del \texttt{Panel de control}. Permite a \texttt{Persona} gestionar las plantas registradas en el sistema: añadir nuevas plantas, consultar su información, editar sus detalles o eliminarlas (Crear, Leer, Actualizar, Borrar). Esta es una funcionalidad esencial para administrar las entidades principales monitoreadas (\texttt{PlantData}).

\subsubsection*{Caso de Uso: CRUD Cámaras}
Extiende (`<<Extend>>`) la funcionalidad del \texttt{Panel de control}, proporcionando un acceso a la gestión de los dispositivos (módulos termográficos y sensores). Es importante resaltar que, aunque el \texttt{Panel de control} es accesible por \texttt{Persona}, la funcionalidad específica de \texttt{CRUD Cámaras} está restringida al rol \texttt{Administrador}, tal como se definió en RF03. Permite al \texttt{Administrador} realizar las operaciones de Crear, Leer, Actualizar y Borrar sobre los dispositivos desde este panel.

\subsubsection*{Caso de Uso: Módulo Observaciones}
Extiende (`<<Extend>>`) la funcionalidad del \texttt{Panel de control}. Representa la funcionalidad añadida para gestionar las observaciones cualitativas realizadas sobre las plantas. Permite a \texttt{Persona} registrar y consultar notas descriptivas sobre aspectos visuales (decoloraciones, uniformidad, notas sobre hojas/tallos) o calificaciones subjetivas del estado de la planta. Esta funcionalidad es clave para cumplir con la metodología de investigación y permitir una mejor compresión de los datos recopilados.


\begin{figure}[H]
    \centering
    \caption{Diagrama de Casos de Uso para la Gestión de Plantas (RF6).}
    \label{fig:casos-uso-plantas}
    \includegraphics[width=0.8\textwidth]{UML/CasosUso/Diagrama de Casos de Uso RF6.png}
\end{figure}

La Figura \ref{fig:casos-uso-plantas} ilustra los casos de uso pertenecientes al \texttt{Modulo de procesamiento}, cuya funcionalidad principal (descrita en RF06) es la gestión de las plantas y la visualización de su estado y datos procesados. Los actores involucrados son el \texttt{Usuario} y el \texttt{Administrador} (generalizados como \texttt{Persona}), además de un actor externo, el \texttt{Sistema de analisis}.

\subsubsection*{Caso de Uso: Panel de control}
Reutilizado del diagrama anterior (Figura \ref{fig:casos-uso-mediciones}), sirve como punto de entrada general para \texttt{Persona}, desde donde se puede acceder (`<<Extend>>`) a la funcionalidad específica de gestión de plantas (\texttt{CRUD Planta}).

\subsubsection*{Caso de Uso: CRUD Planta}
Actúa como el caso de uso central para la administración del ciclo de vida de las plantas dentro de este módulo. Es accedido desde el \texttt{Panel de control} y engloba las operaciones fundamentales sobre las plantas, siendo extendido (`<<Extend>>`) por acciones más específicas como \texttt{Registrar planta} y \texttt{Consultar planta}.

\subsubsection*{Caso de Uso: Registrar planta}
Extiende (`<<Extend>>`) la funcionalidad de \texttt{CRUD Planta}. Permite a \texttt{Persona} añadir una nueva planta al sistema (operación Create), registrando su información inicial para comenzar el monitoreo.

\subsubsection*{Caso de Uso: Consultar planta}
Extiende (`<<Extend>>`) la funcionalidad de \texttt{CRUD Planta}. Permite a \texttt{Persona} visualizar (Leer) la información detallada de una planta específica. Principalmente, según RF06, mostrar su estado actual (resultado del procesamiento de datos). Este caso de uso también es utilizado por el \texttt{Sistema de analisis} externo, para consultar el estado o datos procesados de las plantas. Sirve como base para otras operaciones extendidas.

\subsubsection*{Caso de Uso: Editar planta}
Extiende (`<<Extend>>`) la funcionalidad de \texttt{Consultar planta}. Permite a \texttt{Persona} modificar (Actualizar) la información asociada a una planta existente. Como se especifica en RF06, esto incluye la capacidad de los usuarios para actualizar manualmente el estado asignado a la planta si lo consideran necesario tras una revisión. Este caso de uso también es utilizado por el \texttt{Sistema de analisis} externo, para actualizar el estado de una planta después de procesar los datos obtenidos de los sensores y cámaras.

\subsubsection*{Caso de Uso: Eliminar planta}
Extiende (`<<Extend>>`) la funcionalidad de \texttt{Consultar planta}. Otorga a \texttt{Persona} la capacidad de eliminar (Borrar) el registro de una planta del sistema, por ejemplo, si la planta física es retirada del cultivo.

\subsubsection*{Caso de Uso: Ver historial de estados}
Extiende (`<<Extend>>`) la funcionalidad de \texttt{Consultar planta}. Permite a \texttt{Persona} acceder y visualizar el registro histórico de los cambios de estado que ha tenido una planta a lo largo del tiempo, funcionalidad explícitamente mencionada en RF06 para el seguimiento de la condición de las plantas.


\begin{figure}[H]
    \centering
    \caption{Diagrama de Casos de Uso para la Generación de Reportes (RF7).}
    \label{fig:casos-uso-reportes}
     \includegraphics[width=0.8\textwidth]{UML/CasosUso/Diagrama de Casos de Uso RF7.png}
\end{figure}

La Figura \ref{fig:casos-uso-reportes} presenta los casos de uso correspondientes a la generación de \texttt{Reportes} del sistema, funcionalidad descrita en RF07. Tanto el \texttt{Usuario} como el \texttt{Administrador} (generalizados como \texttt{Persona}) pueden acceder a estas funcionalidades para obtener informes sobre el estado de las plantas y otros módulos del sistema.

\subsubsection*{Caso de Uso: Generar reporte}
Este es el caso de uso principal que inicia \texttt{Persona} para solicitar la creación de un informe. Principalmente, el sistema compila la información relevante sobre el estado de las plantas basándose en los datos recopilados y procesados. El resultado es un reporte consolidado que se genera en formato PDF.

\subsubsection*{Caso de Uso: Filtrar datos}
Extiende (`<<Extend>>`) la funcionalidad de \texttt{Generar reporte}. Proporciona a \texttt{Persona} la opción de aplicar criterios específicos para refinar el contenido del informe antes de su generación final. Por ejemplo, filtrar por planta(s) específica(s) o por lapsos de tiempo.

\subsubsection*{Caso de Uso: Descargar Reporte}
Extiende (`<<Extend>>`) la funcionalidad de \texttt{Filtrar datos} (y, por lo tanto, la de \texttt{Generar reporte}). Una vez que el reporte ha sido generado (y posiblemente filtrado), este caso de uso permite a \texttt{Persona} descargar el archivo resultante (en formato PDF) a su dispositivo local para su consulta o almacenamiento.

\subsubsection*{Caso de Uso: Adjuntar Reporte}
Extiende (`<<Extend>>`) la funcionalidad de \texttt{Filtrar datos}. Representa una opción disponible después de generar (y filtrar) el reporte. Permite de adjuntar el reporte generado por medio de un correo electrónico del usuario autenticado.


\begin{figure}[H]
    \centering
    \caption{Diagrama de Casos de Uso para las Notificaciones (RF8).}
    \label{fig:casos-uso-notificaciones}
    \includegraphics[width=0.8\textwidth]{UML/CasosUso/Diagrama de Casos de Uso RF8.png}
\end{figure}

La Figura \ref{fig:casos-uso-notificaciones} ilustra los casos de uso del \texttt{Modulo de Notificaciones}, que corresponden a la funcionalidad descrita en RF08. En este módulo, el actor principal que inicia las acciones es el propio \texttt{Sistema}, indicando que estas notificaciones son procesos automatizados. Estas alertas se envían por correo electrónico a los roles \texttt{Administrador} y \texttt{Usuario} del sistema.

\subsubsection*{Caso de Uso: Notificar cambio de estado en planta}
Este caso de uso es ejecutado automáticamente por el \texttt{Sistema}. Se activa cuando se produce un cambio significativo en el estado registrado de una planta (por ejemplo, al pasar de 'Saludable' a 'No Saludable' o viceversa, basado en el procesamiento de datos o una actualización manual). El objetivo es alertar proactivamente a los usuarios relevantes (\texttt{Administrador}, \texttt{Usuario}) por correo electrónico sobre la nueva condición de la planta, facilitando una respuesta rápida.

\subsubsection*{Caso de Uso: Notificar intentos fallidos de inicio de sesión}
Iniciado también por el \texttt{Sistema}, este caso de uso forma parte de las "notificaciones de seguridad" mencionadas en RF08. Se activa cuando el sistema detecta una actividad potencialmente sospechosa, como múltiples intentos fallidos de inicio de sesión asociados a una cuenta de usuario. El propósito es informar al usuario afectado, mediante correo electrónico, sobre estos intentos para que pueda verificar la seguridad de su cuenta y tomar acciones si es necesario (ej. cambiar contraseña).


\begin{figure}[H]
    \centering
    \caption{Diagrama de Casos de Uso para el Módulo de Observaciones (RF9).}
    \label{fig:casos-uso-observaciones}
    \includegraphics[width=0.8\textwidth]{UML/CasosUso/Diagrama de Casos de Uso RF9.png}
\end{figure}

La Figura \ref{fig:casos-uso-observaciones} describe los casos de uso para el \texttt{Modulo de Observaciones}. Esta funcionalidad, descrita en RF09, es esencial para la metodología mixta del proyecto, permitiendo la recolección de datos cualitativos sobre las plantas. Los actores \texttt{Usuario} y \texttt{Administrador} (generalizados como \texttt{Persona}) interactúan con este módulo.

\subsubsection*{Caso de Uso: Modulo de Observación}
Este caso de uso representa el punto de entrada principal para que \texttt{Persona} interactúe con las funcionalidades de registro y consulta de observaciones cualitativas de las plantas. Sirve como interfaz para acceder a las operaciones específicas que extienden (`<<Extend>>`) su funcionalidad.

\subsubsection*{Caso de Uso: Registrar Observación}
Extiende (`<<Extend>>`) la funcionalidad del \texttt{Modulo de Observación}. Permite a \texttt{Persona} registrar una nueva observación cualitativa sobre una planta específica, utilizando la plantilla definida en el diseño experimental. Esto incluye seleccionar el estado general visual, describir cambios, anotar aspectos específicos de color/textura y hojas/tallos, asignar una calificación subjetiva y opcionalmente notas adicionales. Estos datos son cruciales para complementar los datos cuantitativos y tener una mejor comprensión de los datos.

\subsubsection*{Caso de Uso: Consultar Observación}
Extiende (`<<Extend>>`) la funcionalidad del \texttt{Modulo de Observación}. Permite a \texttt{Persona} buscar y visualizar las observaciones cualitativas previamente registradas para una planta. Esto facilita el seguimiento de la evolución visual descrita en el diseño experimental y la comparación con los datos cuantitativos (termografía, sensores).


% =================================================
% =================================================

\subsection{Diagramas de Secuencia}
Un diagrama de secuencia muestra un conjunto de mensajes ordenados en una secuencia temporal \cite{Rumbaugh2007}. Cada rol se muestra como una línea de vida es decir, una línea vertical que representa al rol a lo largo del tiempo a través de la interacción completa \cite{Rumbaugh2007}. Los mensajes se muestran con flechas entre líneas de vida \cite{Rumbaugh2007}.

\subsection*{Líneas de Vida Principales}

Los diagramas de secuencia presentados en este documento ilustran las interacciones entre diferentes componentes del sistema para realizar casos de uso específicos. Debido al considerable número de escenarios detallados, se presenta a continuación una descripción general de las líneas de vida (\textit{lifelines}) que aparecen de forma recurrente, representando los actores y las capas arquitectónicas principales del sistema:

\begin{itemize}
    \item \texttt{:Persona}: Representa al usuario final que interactúa con el sistema a través de la interfaz gráfica. Dependiendo del contexto, puede ser un \texttt{Usuario} o un \texttt{Administrador}. Es el iniciador de las secuencias asociadas a funcionalidades interactivas.

    \item \texttt{:Vista}: Simboliza la interfaz de usuario cliente con la que interactúa \texttt{:Persona}. Esta línea de vida representa la aplicación ejecutándose en diferentes plataformas (web, movil u otras). Sus responsabilidades incluyen presentar información, capturar datos del usuario, realizar validaciones iniciales y enviar solicitudes (vía \texttt{HTTPS}) a la API backend, así como mostrar las respuestas recibidas.

    \item \texttt{:Infraestructura}: Representa la capa más externa de la API backend, siguiendo los principios de la Arquitectura Cebolla (\textit{Onion Architecture}). Esta capa actúa como la fachada de la API, recibiendo las solicitudes HTTP desde \texttt{:Vista}. Es responsable de manejar los controladores (endpoints), la configuración, la comunicación con elementos externos como la base de datos \texttt{PostgreSQL} (a través de \texttt{ODBC} o un ORM), la gestión de logs, la autenticación/autorización a nivel de API, y otros aspectos transversales (\textit{cross-cutting concerns}). Traduce las solicitudes del exterior hacia las capas internas y las respuestas de las capas internas hacia el exterior.

    \item \texttt{:Aplicacion}: Representa la capa de lógica de aplicación en la Arquitectura Cebolla. Contiene la orquestación de los casos de uso del sistema. Recibe las solicitudes ya procesadas por la capa de \texttt{:Infraestructura}, invoca la lógica de negocio necesaria en la capa de \texttt{:Dominio}, coordina las transacciones y prepara los datos de respuesta para la capa de \texttt{:Infraestructura}. Implementa la lógica específica de la aplicación que no pertenece estrictamente al dominio ni a la infraestructura.

    \item \texttt{:Dominio}: Simboliza el núcleo del sistema, la capa más interna y central de la Arquitectura Cebolla. Alberga las entidades de negocio (ej. Planta, Cultivo, Usuario), los objetos de valor, las reglas de negocio fundamentales, los eventos de dominio y, crucialmente, las interfaces de los repositorios (abstracciones para el acceso a datos). Procesa la lógica de negocio pura, respondiendo a las invocaciones de la capa de \texttt{:Aplicacion} y es completamente independiente de las tecnologías de UI, base de datos o infraestructura externa. La persistencia real de los datos (definida por las interfaces de repositorio) se implementa típicamente en la capa de \texttt{:Infraestructura}.
\end{itemize}

Generalmente, el flujo de una solicitud iniciada por el usuario sigue la secuencia \texttt{:Persona} $\rightarrow$ \texttt{:Vista} $\rightarrow$ \texttt{:Infraestructura} $\rightarrow$ \texttt{:Aplicacion} $\rightarrow$ \texttt{:Dominio}. La capa de \texttt{:Dominio} (o \texttt{:Aplicacion} a través de ella) utiliza las interfaces de repositorio, cuya implementación en \texttt{:Infraestructura} interactúa con la base de datos. La respuesta viaja de vuelta siguiendo el camino inverso: \texttt{:Dominio} $\rightarrow$ \texttt{:Aplicacion} $\rightarrow$ \texttt{:Infraestructura} $\rightarrow$ \texttt{:Vista} $\rightarrow$ \texttt{:Persona}. Esta estructura promueve la separación de responsabilidades y la mantenibilidad del sistema.

Es importante señalar que aquellos diagramas que introducen líneas de vida con roles particulares no cubiertos en la descripción general (como la línea de vida \texttt{:Camara} detallada en la Figura\ref{fig:seq_activar_camara}) o aquellos que representan patrones de interacción fundamentales y reutilizables (como el flujo base para envío de formularios mostrado en la Figura\ref{fig:seq_base_formulario}) incluyen una descripción textual específica adjunta. 
Para los diagramas de secuencia restantes, se entiende que siguen el patrón general de interacción entre las capas \texttt{:Vista}, \texttt{:Infraestructura}, \texttt{:Aplicacion} y \texttt{:Dominio}, utilizando las responsabilidades asignadas a cada línea de vida según se describió previamente. 

\begin{figure}[H]
    \centering
    \caption{Diagrama de secuencia base: Envío de formulario.}
    \includegraphics[width=0.8\textwidth]{UML/Secuencia/Diagrama de Secuencia Base Envio Formulario.png}
	\label{fig:seq_base_formulario}
\end{figure}
\subsubsection*{Líneas de Vida Involucradas}
Las líneas de vida principales en este diagrama base son:
\begin{itemize}
    \item \texttt{:Persona}: Representa al usuario (\texttt{Usuario} o \texttt{Administrador}) que interactúa con la interfaz gráfica del sistema. Es quien inicia la acción, ingresa los datos y toma la decisión final de confirmar o cancelar el envío.
    \item \texttt{:Vista}: Representa el componente de la interfaz de usuario (la pantalla o ventana específica) que contiene el formulario. Recibe los datos ingresados, gestiona el proceso de envío inicial y maneja el diálogo de confirmación con el usuario.
\end{itemize}

\subsubsection*{Descripción del Flujo Genérico}
La secuencia describe el siguiente flujo general:
\begin{enumerate}
    \item El proceso comienza cuando \texttt{:Persona} accede a un formulario (1: \texttt{Ingresa a un formulario}) e introduce los datos requeridos (2: \texttt{Ingresar datos del formulario}).
    \item \texttt{:Persona} inicia la acción de envío (3: \texttt{Enviar formulario}) hacia la \texttt{:Vista}.
    \item La \texttt{:Vista} realiza un procesamiento inicial (4: \texttt{Procesar solicitud}), que podría incluir validaciones del lado del cliente.
    \item La \texttt{:Vista} solicita una confirmación explícita al usuario mostrando una ventana emergente (5: \texttt{Mostrar ventana emergente de confirmación}).
    \item Se presenta un fragmento alternativo (\texttt{alt}) basado en la respuesta de \texttt{:Persona}:
    \begin{itemize}
        \item \textbf{Si \texttt{[Confirma acción]}:} \texttt{:Persona} confirma el envío (6). La \texttt{:Vista} procede con el procesamiento definitivo de la solicitud (7), cierra la ventana emergente (8) y permite continuar la secuencia (9), lo que usualmente implica una redirección o un mensaje de éxito. \textit{(Nota: El paso 7 es donde, en diagramas específicos, se detallaría la comunicación con controladores, servicios y base de datos)}.
        \item \textbf{Si \texttt{[Cancela acción]}:} \texttt{:Persona} cancela el envío (10). La \texttt{:Vista} procesa la cancelación (11), cierra la ventana emergente (12) y vuelve a mostrar el formulario (13), permitiendo al usuario corregir datos o abandonar la tarea.
    \end{itemize}
\end{enumerate}
Este diagrama establece la interacción fundamental usuario-interfaz para operaciones de formulario, haciendo énfasis en el paso de confirmación antes del procesamiento final.


\begin{figure}[H]
    \centering
    \caption{Diagrama de Secuencia para el Registro (RF1.0).}
 \includegraphics[width=0.8\textwidth]{UML/Secuencia/Diagrama de Secuencia RF1.0 Registro.png}
\end{figure}


\begin{figure}[H]
    \centering
    \caption{Diagrama de Secuencia para Solicitar Código (RF1.1).}
    \includegraphics[width=0.8\textwidth]{UML/Secuencia/Diagrama de Secuencia RF1.1 Solicitar Código.png}
\end{figure}


\begin{figure}[H]
	\centering
	\caption{Diagrama de Secuencia para Iniciar Sesión (RF2.0).}
	\includegraphics[width=0.8\textwidth]{UML/Secuencia/Diagrama de Secuencia RF2.0 Iniciar Sesión.png}
\end{figure}


\begin{figure}[H]
	\centering
	\caption{Diagrama de Secuencia para Cerrar Sesión (RF2.1).}
 \includegraphics[width=0.8\textwidth]{UML/Secuencia/Diagrama de Secuencia RF2.1 Cerrar Sesión.png}
\end{figure}


\begin{figure}[H]
	\centering
		\caption{Diagrama de Secuencia para Recuperar Contraseña (RF2.2).}
	\includegraphics[width=0.8\textwidth]{UML/Secuencia/Diagrama de Secuencia RF2.2 Recuperar Contraseña.png}
\end{figure}


\begin{figure}[H]
	\centering
		\caption{Diagrama de Secuencia para Crear Cámara (RF3.1).}
	\includegraphics[width=0.8\textwidth]{UML/Secuencia/Diagrama de Secuencia RF3.1 Crear Cámara.png}
\end{figure}


\begin{figure}[H]
	\centering
		\caption{Diagrama de Secuencia para Activar Cámara (RF3.1.1).}
	\includegraphics[width=0.63\textwidth]{UML/Secuencia/Diagrama de Secuencia RF3.1.1 Activar Cámara.png}
	\label{fig:seq_activar_camara}
\end{figure}

\subsubsection*{Línea de Vida :Camara}
La responsabilidad principal de la línea de vida \texttt{:Camara} es interactuar con la API del sistema. Encapsula la lógica autónoma del dispositivo físico, manejando su activación, configuración, el ciclo periódico de toma y envío de datos (ambientales y de imagen), y la gestión básica de errores de comunicación con la API. Su comportamiento se divide en dos fases principales:

1.  \textbf{Fase de Activación:}
    \begin{itemize}
        \item Al iniciar y verificar la conexión de red, la \texttt{:Camara} envía un código de activación único a la API.
        \item Espera una respuesta de la API. Si la activación es exitosa, recibe la configuración operativa (ej. intervalo de muestreo) que fue definida por el \texttt{Administrador} durante el registro del dispositivo en el sistema.
        \item Almacena esta configuración recibida de forma persistente (memoria no volátil) para guiar su funcionamiento futuro.
        \item Si el proceso de activación falla (ej. código inválido, API no responde correctamente), la \texttt{:Camara} entra en un estado de error y detiene el proceso, sin pasar a la fase operativa.
    \end{itemize}

2.  \textbf{Fase Operativa (Ciclo de Trabajo):}
    \begin{itemize}
        \item Una vez activada, la \texttt{:Camara} opera en un ciclo continuo basado en el intervalo de tiempo definido en su configuración almacenada.
        \item Al inicio de cada ciclo, verifica la disponibilidad de la API.
        \item \textbf{Si la API está disponible:} Procede a recolectar los datos de los sensores ambientales (temperatura, humedad, etc.) y los envía a la API. Seguidamente, captura las imágenes (termográfica y RGB) y las envía también a la API. Después de enviar los datos, entra en un modo de bajo consumo o reposo hasta que el temporizador del ciclo indique el inicio del siguiente.
        \item \textbf{Si la API no está disponible:} La \texttt{:Camara} omite la recolección y envío de datos para ese ciclo. Entra en un periodo de espera antes de volver a intentar la verificación de la API en el siguiente ciclo programado.
    \end{itemize}


\begin{figure}[H]
	\centering
		\caption{Diagrama de Secuencia para Consultar Cámara (RF3.2).}
	\includegraphics[width=0.8\textwidth]{UML/Secuencia/Diagrama de Secuencia RF3.2 Consultar Cámara.png}
\end{figure}


\begin{figure}[H]
	\centering
	\caption{Diagrama de Secuencia para Editar Cámara (RF3.3).}
 \includegraphics[width=0.8\textwidth]{UML/Secuencia/Diagrama de Secuencia RF3.3 Editar Cámara.png}
\end{figure}


\begin{figure}[H]
	\centering
		\caption{Diagrama de Secuencia para Eliminar Cámara (RF3.4).}
	\includegraphics[width=0.8\textwidth]{UML/Secuencia/Diagrama de Secuencia RF3.4 Eliminar Cámara.png}
\end{figure}


\begin{figure}[H]
	\centering
	\caption{Diagrama de Secuencia para Consultar Perfil (RF4.1).}
 \includegraphics[width=0.8\textwidth]{UML/Secuencia/Diagrama de Secuencia RF4.1 Consultar Perfil.png}
\end{figure}


\begin{figure}[H]
	\centering
		\caption{Diagrama de Secuencia para Editar Perfil (RF4.2).}
	\includegraphics[width=0.8\textwidth]{UML/Secuencia/Diagrama de Secuencia RF4.2 Editar Perfil.png}
\end{figure}


\begin{figure}[H]
	\centering
	\caption{Diagrama de Secuencia para Eliminar Perfil (RF4.3).}
 \includegraphics[width=0.8\textwidth]{UML/Secuencia/Diagrama de Secuencia RF4.3 Eliminar Perfil.png}
\end{figure}


\begin{figure}[H]
	\centering
		\caption{Diagrama de Secuencia para Cambiar Contraseña (RF4.4).}
	\includegraphics[width=0.8\textwidth]{UML/Secuencia/Diagrama de Secuencia RF4.4 Cambiar Contraseña.png}
\end{figure}


\begin{figure}[H]
	\centering
		\caption{Diagrama de Secuencia para Agregar Integrante de Cultivo (RF4.5).}
\includegraphics[width=0.8\textwidth]{UML/Secuencia/Diagrama de Secuencia RF4.5 Agregar Integrante Cultivo.png}
\end{figure}

\begin{figure}[H]
	\centering
	\caption{Diagrama de Secuencia para Eliminar Integrante de Cultivo (RF4.6).}
 \includegraphics[width=0.8\textwidth]{UML/Secuencia/Diagrama de Secuencia RF4.6 Eliminar Integrante Cultivo.png}
\end{figure}


\begin{figure}[H]
	\centering
		\caption{Diagrama de Secuencia para el Módulo de Mediciones (RF5.0).}
	\includegraphics[width=0.8\textwidth]{UML/Secuencia/Diagrama de Secuencia RF5.0 Módulo de mediciones.png}
\end{figure}


\begin{figure}[H]
	\centering
	\caption{Diagrama de Secuencia para Crear Planta (RF6.1).}
 \includegraphics[width=0.8\textwidth]{UML/Secuencia/Diagrama de Secuencia RF6.1 Crear Planta.png}
\end{figure}


\begin{figure}[H]
	\centering
		\caption{Diagrama de Secuencia para Consultar Planta (RF6.2).}
	\includegraphics[width=0.8\textwidth]{UML/Secuencia/Diagrama de Secuencia RF6.2 Consultar Planta.png}
\end{figure}


\begin{figure}[H]
	\centering
	\caption{Diagrama de Secuencia para Editar Planta (RF6.3).}
 \includegraphics[width=0.8\textwidth]{UML/Secuencia/Diagrama de Secuencia RF6.3 Editar Planta.png}
\end{figure}


\begin{figure}[H]
	\centering
	\caption{Diagrama de Secuencia para Eliminar Planta (RF6.4).}
 \includegraphics[width=0.8\textwidth]{UML/Secuencia/Diagrama de Secuencia RF6.4 Eliminar Planta.png}
\end{figure}


\begin{figure}[H]
	\centering
		\caption{Diagrama de Secuencia para Generar Reporte (RF7.0).}
	\includegraphics[width=0.8\textwidth]{UML/Secuencia/Diagrama de Secuencia RF7.0 Generar Reporte.png}
\end{figure}


\begin{figure}[H]
	\centering
	\caption{Diagrama de Secuencia para Descargar Reporte (RF7.1).}
 \includegraphics[width=0.8\textwidth]{UML/Secuencia/Diagrama de Secuencia RF7.1 Descargar Reporte.png}
\end{figure}


\begin{figure}[H]
	\centering
		\caption{Diagrama de Secuencia para Adjuntar Reporte (RF7.2).}
	\includegraphics[width=0.8\textwidth]{UML/Secuencia/Diagrama de Secuencia RF7.2 Adjuntar Reporte.png}
\end{figure}


\begin{figure}[H]
	\centering
	\caption{Diagrama de Secuencia para Notificar Planta (RF8.1).}
 \includegraphics[width=0.8\textwidth]{UML/Secuencia/Diagrama de Secuencia RF8.1 Notificar Planta.png}
\end{figure}


\begin{figure}[H]
	\centering
	\caption{Diagrama de Secuencia para Notificar Seguridad (RF8.2).}
 \includegraphics[width=0.8\textwidth]{UML/Secuencia/Diagrama de Secuencia RF8.2 Notificar Seguridad.png}
\end{figure}


\begin{figure}[H]
	\centering
	\caption{Diagrama de Secuencia para Crear Observación (RF9.1).}
 \includegraphics[width=0.8\textwidth]{UML/Secuencia/Diagrama de Secuencia RF9.1 Crear Observación.png}
\end{figure}


\begin{figure}[H]
	\centering
	\caption{Diagrama de Secuencia para Consultar Observación (RF9.1).}
 \includegraphics[width=0.8\textwidth]{UML/Secuencia/Diagrama de Secuencia RF9.2 Consultar Observación.png}
\end{figure}

% =================================================
% =================================================

\subsection{Diagramas de Actividades}

Una actividad muestra el flujo de control entre las actividades computacionales involucradas en la realización de un cálculo o un flujo de trabajo \cite{Rumbaugh2007}. Una acción es un paso computacional primitivo y un nodo de actividad es un grupo de acciones o subactividades \cite{Rumbaugh2007}. Una actividad describe tanto el cómputo secuencial como el concurrente \cite{Rumbaugh2007}.

Los diagramas de actividad presentados a continuación modelan los flujos de trabajo (\textit{workflows}) asociados a procesos o funcionalidades clave del sistema. Estos diagramas utilizan particiones verticales, comúnmente conocidas como calles o \textit{swimlanes}, para asignar claramente la responsabilidad de cada acción a un participante específico del proceso. En la mayoría de los diagramas de este documento, se utilizan dos calles principales:

\begin{itemize}
    \item \textbf{\texttt{Usuario}}: Esta calle agrupa todas aquellas actividades que son ejecutadas directamente por la persona que interactúa con la interfaz gráfica del sistema. Representa las acciones del usuario final, ya sea que actúe con el rol de \texttt{Usuario} o de \texttt{Administrador}. Ejemplos típicos de actividades en esta calle incluyen: ingresar datos en un formulario, seleccionar opciones, iniciar una acción (como presionar un botón) y visualizar mensajes o resultados mostrados por la aplicación.

    \item \textbf{\texttt{Sistema}}: Esta calle engloba todas las actividades y procesos que son realizados internamente por la aplicación software (backend y/o frontend). Representa las operaciones automáticas del sistema, tales como: procesar la información enviada por el usuario, validar datos según reglas de negocio, ejecutar algoritmos, consultar o actualizar información en la base de datos, determinar estados, generar respuestas y preparar la información que se devolverá a la interfaz para ser visualizada por el usuario.
\end{itemize}

La separación de actividades entre estas dos calles principales (\texttt{Usuario} y \texttt{Sistema}) permite visualizar de manera clara la interacción entre el usuario humano y la lógica interna de la aplicación a lo largo de un flujo de trabajo específico. Otros diagramas podrían incluir calles adicionales si participan otros actores o sistemas externos específicos en un proceso particular.

\begin{figure}[H]
    \centering
    \caption{Diagrama de Actividad para el Registro (RF1.0).}
    \includegraphics[width=0.7\textwidth]{UML/Actividad/Diagrama de Actividad RF1.0 Registro.png}
\end{figure}


\begin{figure}[H]
    \centering
    \caption{Diagrama de Actividad para Solicitar Código (RF1.1).}
 \includegraphics[width=0.6\textwidth]{UML/Actividad/Diagrama de Actividad RF1.1 Solicitar Código.png}
\end{figure}


\begin{figure}[H]
	\centering
	\caption{Diagrama de Actividad para Iniciar Sesión (RF2.0).}
 \includegraphics[width=0.7\textwidth]{UML/Actividad/Diagrama de Actividad RF2.0 Iniciar Sesión.png}
\end{figure}


\begin{figure}[H]
	\centering
		\caption{Diagrama de Actividad para Cerrar Sesión (RF2.1).}
	\includegraphics[width=0.65\textwidth]{UML/Actividad/Diagrama de Actividad RF2.1 Cerrar Sesión.png}
\end{figure}


\begin{figure}[H]
	\centering
	\caption{Diagrama de Actividad para Recuperar Contraseña (RF2.2).}
 \includegraphics[width=0.45\textwidth]{UML/Actividad/Diagrama de Actividad RF2.2 Recuperar Contraseña.png}
\end{figure}


\begin{figure}[H]
	\centering
	\caption{Diagrama de Actividad para Crear Cámara (RF3.1).}
 \includegraphics[width=0.5\textwidth]{UML/Actividad/Diagrama de Actividad RF3.1 Crear Cámara.png}
\end{figure}


\begin{figure}[H]
	\centering
		\caption{Diagrama de Actividad para Activar Cámara (RF3.1.1).}
	\includegraphics[width=0.7\textwidth]{UML/Actividad/Diagrama de Actividad RF3.1.1 Activar Cámara.png}
	\label{fig:act_activar_hardware}
\end{figure}

El diagrama de actividad de la Figura~\ref{fig:act_activar_hardware} muestra el flujo de trabajo para la activación e inicio de operación del dispositivo físico. Conforme a la descripción general proporcionada al inicio de esta sección, las calles \texttt{Administrador} y \texttt{Sistema} representan las acciones del usuario y del backend, respectivamente. La calle \texttt{Hardware}, corresponden a la lógica ejecutada por el firmware del propio dispositivo.
Este flujo describe cómo el hardware gestiona su activación inicial y luego opera en un ciclo de verificación, recolección, envío y espera, interactuando con el \texttt{Sistema} cuando es necesario y gestionando su consumo de energía.


\begin{figure}[H]
	\centering
	\caption{Diagrama de Actividad para Consultar Cámara (RF3.2).}
 \includegraphics[width=0.65\textwidth]{UML/Actividad/Diagrama de Actividad RF3.2 Consultar Cámara.png}
\end{figure}


\begin{figure}[H]
	\centering
	\caption{Diagrama de Actividad para Editar Cámara (RF3.3).}
 \includegraphics[width=0.4\textwidth]{UML/Actividad/Diagrama de Actividad RF3.3 Editar Cámara.png}
\end{figure}


\begin{figure}[H]
	\centering
		\caption{Diagrama de Actividad para Eliminar Cámara (RF3.4).}
\includegraphics[width=0.4\textwidth]{UML/Actividad/Diagrama de Actividad RF3.4 Eliminar Cámara.png}
\end{figure}


\begin{figure}[H]
	\centering
		\caption{Diagrama de Actividad para Consultar Perfil (RF4.1).}
	\includegraphics[width=0.6\textwidth]{UML/Actividad/Diagrama de Actividad RF4.1 Consultar Perfil.png}
\end{figure}


\begin{figure}[H]
	\centering
	\caption{Diagrama de Actividad para Editar Perfil (RF4.2).}
 \includegraphics[width=0.45\textwidth]{UML/Actividad/Diagrama de Actividad RF4.2 Editar Perfil.png}
\end{figure}


\begin{figure}[H]
	\centering
	\caption{Diagrama de Actividad para Eliminar Perfil (RF4.3).}
 \includegraphics[width=0.45\textwidth]{UML/Actividad/Diagrama de Actividad RF4.3 Eliminar Perfil.png}
\end{figure}


\begin{figure}[H]
	\centering
		\caption{Diagrama de Actividad para Cambiar Contraseña (RF4.4).}
	\includegraphics[width=0.5\textwidth]{UML/Actividad/Diagrama de Actividad RF4.4 Cambiar Contraseña.png}
\end{figure}


\begin{figure}[H]
	\centering
	\caption{Diagrama de Actividad para Agregar Integrante de Cultivo (RF4.5).}
 \includegraphics[width=0.5\textwidth]{UML/Actividad/Diagrama de Actividad RF4.5 Agregar Integrante Cultivo.png}
\end{figure}


\begin{figure}[H]
	\centering
	\caption{Diagrama de Actividad para Eliminar Integrante de Cultivo (RF4.6).}
 \includegraphics[width=0.45\textwidth]{UML/Actividad/Diagrama de Actividad RF4.6 Eliminar Integrante Cultivo.png}
\end{figure}

\begin{figure}[H]
	\centering
	\caption{Diagrama de Actividad para el Módulo de Mediciones (RF5.0).}
 \includegraphics[width=0.8\textwidth]{UML/Actividad/Diagrama de Actividad RF5.0 Módulo de mediciones.png}
\end{figure}


\begin{figure}[H]
	\centering
	\caption{Diagrama de Actividad para Crear Planta (RF6.1).}
 \includegraphics[width=0.5\textwidth]{UML/Actividad/Diagrama de Actividad RF6.1 Crear Planta.png}
\end{figure}


\begin{figure}[H]
	\centering
	\caption{Diagrama de Actividad para Consultar Planta (RF6.2).}
 \includegraphics[width=0.65\textwidth]{UML/Actividad/Diagrama de Actividad RF6.2 Consultar Planta.png}
\end{figure}


\begin{figure}[H]
	\centering
		\caption{Diagrama de Actividad para Editar Planta (RF6.3).}
	\includegraphics[width=0.4\textwidth]{UML/Actividad/Diagrama de Actividad RF6.3 Editar Planta.png}
\end{figure}


\begin{figure}[H]
	\centering
	\caption{Diagrama de Actividad para Eliminar Planta (RF6.4).}
 \includegraphics[width=0.4\textwidth]{UML/Actividad/Diagrama de Actividad RF6.4 Eliminar Planta.png}
\end{figure}


\begin{figure}[H]
	\centering
		\caption{Diagrama de Actividad para Generar Reporte (RF7.0).}
	\includegraphics[width=0.65\textwidth]{UML/Actividad/Diagrama de Actividad RF7.0 Generar Reporte.png}
\end{figure}


\begin{figure}[H]
	\centering
		\caption{Diagrama de Actividad para Descargar Reporte (RF7.1).}
\includegraphics[width=0.65\textwidth]{UML/Actividad/Diagrama de Actividad RF7.1 Descargar Reporte.png}
\end{figure}


\begin{figure}[H]
	\centering
		\caption{Diagrama de Actividad para Adjuntar Reporte (RF7.2).}
\includegraphics[width=0.5\textwidth]{UML/Actividad/Diagrama de Actividad RF7.2 Adjuntar Reporte.png}
\end{figure}


\begin{figure}[H]
	\centering
	\caption{Diagrama de Actividad para Notificar Planta (RF8.1).}
 \includegraphics[width=0.5\textwidth]{UML/Actividad/Diagrama de Actividad RF8.1 Notificar Planta.png}
\end{figure}


\begin{figure}[H]
	\centering
		\caption{Diagrama de Actividad para Notificar Seguridad (RF8.2).}
	\includegraphics[width=0.8\textwidth]{UML/Actividad/Diagrama de Actividad RF8.2 Notificar Seguridad.png}
\end{figure}


\begin{figure}[H]
	\centering
	\caption{Diagrama de Actividad para Crear Observación (RF9.1).}
 \includegraphics[width=0.5\textwidth]{UML/Actividad/Diagrama de Actividad RF9.1 Crear Observación.png}
\end{figure}


\begin{figure}[H]
	\centering
		\caption{Diagrama de Actividad para Consultar Observación (RF9.2).}
	\includegraphics[width=0.65\textwidth]{UML/Actividad/Diagrama de Actividad RF9.2 Consultar Observación.png}
\end{figure}

% =================================================
% =================================================

\subsection{Diagrama de Clases}

Una clase es la descripción de un concepto del dominio de la aplicación o del dominio de la solución \cite{Rumbaugh2007}. Las clases son el centro alrededor del cual se organiza la vista de clases \cite{Rumbaugh2007}. La vista estática se muestra en los diagramas de clases \cite{Rumbaugh2007}.

En esta sección se presenta el diagrama de clases principal del sistema. Se describirán las tablas (clases) más importantes, sus atributos, las relaciones (asociaciones, generalizaciones) entre ellas y cómo estas estructuras en conjunto definen el funcionamiento general y la organización de los datos del sistema.

% =================================================
% =================================================

\subsection{Diagrama de Despliegue}

La vista de despliegue muestra la disposición física de los nodos, que son recursos computacionales de tiempo de ejecución, como computadoras u otros dispositivos \cite{Rumbaugh2007}. Durante la ejecución, los nodos pueden contener artefactos, que son entidades físicas como archivos \cite{Rumbaugh2007}.

\begin{figure}[H]
    \centering
    \caption{Diagrama de Despliegue del Sistema.}
    \label{fig:despliegue}
    \includegraphics[width=0.8\textwidth]{UML/Otros/Diagrama de Despliegue.png}
\end{figure}

El Diagrama de Despliegue, mostrado en la Figura \ref{fig:despliegue}, representa la arquitectura física del sistema y cómo sus componentes de software se distribuyen en diferentes nodos de hardware y servidores. 
Se identifican cuatro nodos principales: 
\begin{itemize}
    \item \texttt{Frontend}: Representa el entorno cliente donde se ejecuta la interfaz de usuario. Despliega un componente \texttt{Uno Platform} y genera artefactos para diversas plataformas (\texttt{.wasm}, \texttt{.apk}, \texttt{.exe}).
    \item \texttt{<<Servidor de aplicación>>}: Alojado en la plataforma \texttt{Somee}, donde se despliega el componente principal de backend: la \texttt{API Net Core 8}.
    \item \texttt{<<Servidor Base de Datos>>}: Alojado en \texttt{HelioHost}, que ejecuta el sistema gestor de base de datos \texttt{PostgreSQL 13.20}.
    \item \texttt{<<Hardware>> Dispositivo sensores}: Representa el dispositivo físico encargado de la toma de datos y ejecuta su propio \texttt{Firmware}.
\end{itemize}
La comunicación entre el \texttt{Frontend} y la \texttt{API}, así como entre el \texttt{Dispositivo sensores} y la \texttt{API}, se establece a través del protocolo seguro \texttt{HTTPS}. La \texttt{API} interactúa con la base de datos mediante \texttt{ODBC}. El diagrama ilustra una arquitectura distribuida cliente-servidor, donde tanto los usuarios finales como los dispositivos de hardware se comunican con un servicio central (API) que gestiona la lógica y la persistencia de los datos.

% =================================================
% =================================================

\section{Diseño de los Casos de Prueba}
Texto sobre el diseño de casos de prueba utilizando SonarQube.

% =================================================
% =================================================

\section{Estimación de Recursos}
Texto sobre la estimación de recursos utilizando el método de puntos de función o puntos de casos de uso.

\section{Resultados de la Implementación del Software}
Texto sobre el resultado de implementar el software.

\section{Conclusiones y Recomendaciones del software}
Discusión, conclusiones y recomendaciones sobre el software y su integración.
         % Capítulo II: Documentación del Software
% Capítulo IV: Desarrollo y Caracterización del Hardware
\chapter{DESARROLLO Y CARACTERIZACIÓN DEL HARDWARE}
\label{chap:hardware} % Etiqueta opcional para el capítulo

\section{Introducción}
\label{sec:hardware_intro}
Aunque la termografía infrarroja (IRT) es una técnica eficaz para el monitoreo agrícola \cite{ErazoAux2022, Hernanda2024, Quezada2020}, su adopción ha sido limitada por el alto costo de los equipos tradicionales, generando una brecha tecnológica \cite{Dong2024, Yun2023}. Los avances recientes en la fabricación de sensores térmicos de bajo costo, como el MLX90640, y microcontroladores potentes, ofrecen la posibilidad de desarrollar sistemas embebidos asequibles que superen esta barrera \cite{Dong2024}. No obstante, la viabilidad de esta tecnología emergente depende críticamente de su fiabilidad metrológica. Componentes de bajo costo pueden presentar limitaciones técnicas, como la deriva térmica y una alta sensibilidad a las condiciones ambientales, lo que exige procesos rigurosos de calibración y validación para asegurar la confiabilidad de los datos \cite{Acorsi2020, Jiao2022, Yun2023}.

Establecer la precisión, repetibilidad y fiabilidad de un sistema basado en estos componentes es un paso fundamental previo a su recomendación y despliegue en campo. Por lo tanto, el objetivo de este capítulo es describir en detalle el proceso de desarrollo, integración y caracterización del hardware del prototipo de monitoreo termográfico. Se abordará la selección y justificación de cada componente electrónico, el diseño e integración del sistema y la metodología experimental empleada para su validación y caracterización metrológica, con el fin de establecer su fiabilidad como instrumento de medición en aplicaciones agrícolas.

\section{Descripción y Selección de Componentes}
\label{sec:hardware_componentes}
El diseño del sistema se basó en la integración de componentes electrónicos de bajo costo, alta disponibilidad, bajo consumo energético y tamaño reducido. Estos criterios son clave para el desarrollo de soluciones escalables en agricultura de precisión \cite{Dong2024, SanchezSutil2021}. A continuación, se detalla cada componente seleccionado y la justificación técnica de su elección.

    \subsection{Microcontrolador ESP32-S3}
    \label{subsec:hardware_esp32s3}
    El componente central del prototipo es el microcontrolador ESP32-S3 de Espressif Systems \cite{ESP32S3Data}. Se trata de un sistema en chip (SoC) de bajo consumo energético, basado en una arquitectura de 32 bits. El dispositivo integra un microprocesador Xtensa® 32-bit LX7 de doble núcleo, capaz de operar a frecuencias de hasta 240 MHz. Además, proporciona conectividad inalámbrica dual, incorporando Wi-Fi de 2.4 GHz (compatible con IEEE 802.11b/g/n) y Bluetooth® 5 (LE).

    La selección de este microcontrolador se justifica por sus características idóneas para aplicaciones de Internet de las Cosas (IoT) \cite{SanchezSutil2021}, su bajo costo y un conjunto de periféricos integrados que fueron cruciales para este proyecto. Entre estos destacan una interfaz de cámara (DVP 8-bit/16-bit) y un controlador de host SD/MMC. Se optó por una placa de desarrollo basada en la variante N16R8 (16MB de Flash y 8MB de PSRAM), cuyo alto rendimiento y capacidad de memoria son suficientes para las tareas de adquisición, procesamiento de imágenes y ejecución de algoritmos futuros. Adicionalmente, la placa de desarrollo seleccionada incluye un LED RGB integrado de alta luminosidad, empleado para proveer retroalimentación visual de los estados del sistema, el cual es controlado mediante el periférico LED PWM del microcontrolador.

    A pesar de sus ventajas, se debe reportar una limitación encontrada durante la implementación. Las placas de desarrollo ESP32-S3 adquiridas presentaron problemas con sus antenas Wi-Fi integradas en la PCB. Se observó que la conectividad era nula o intermitente, presuntamente debido a una impedancia inadecuada de la antena. Para solventar este inconveniente y asegurar la conectividad Wi-Fi, fue necesario soldar una antena externa al puerto correspondiente en la placa, una modificación que estabilizó la comunicación con la red.

    \subsection{Sensor Térmico MLX90640}
    \label{subsec:hardware_mlx90640}
    Para la medición de la temperatura superficial se seleccionó el sensor de imagen térmica MLX90640 de Melexis \cite{MLX90640Data}. Este componente es un arreglo (matriz) de termopilas infrarrojas completamente calibrado de fábrica, con una resolución de 32x24 píxeles, lo que resulta en un total de 768 puntos de medición independientes. Cada píxel puede medir la temperatura de objetos en un rango de -40\degree C a 300\degree C. El sensor opera con un voltaje de alimentación nominal de 3.3V y consume menos de 23mA durante su funcionamiento. La comunicación con el microcontrolador se realiza a través de una interfaz digital I\textsuperscript{2}C compatible con el modo FM+ (hasta 1MHz).

    Una característica importante del MLX90640 es su disponibilidad en diferentes variantes según el campo de visión (Field of View - FOV). Existen opciones con un FOV amplio de 110\degree x 75\degree{} (variante BAA) y una opción más estrecha de 55\degree x 35\degree{} (variante BAB). Para este estudio, se seleccionó la variante BAB (MLX90640ESF-BAB), cuyo campo de visión más reducido es adecuado para enfocar el dosel de plantas individuales de arándano a distancias cortas. El sensor presenta una sensibilidad térmica, expresada como la Diferencia de Temperatura Equivalente a Ruido (Noise Equivalent Temperature Difference - NETD), de 0.1K RMS a una tasa de refresco de 1Hz. La tasa de refresco es programable por el usuario y puede variar entre 0.5Hz y 64Hz.

    La elección de este sensor se justifica principalmente por su bajo costo en comparación con las cámaras térmicas industriales tradicionales, y su capacidad para proporcionar datos matriciales de temperatura. Esta característica supera las limitaciones de los termómetros infrarrojos de un solo punto, permitiendo capturar la distribución espacial de la temperatura en el objetivo, lo cual es fundamental para analizar el dosel del cultivo \cite{Dong2024}. 

    Para la implementación de los prototipos, se utilizaron módulos basados en el chip MLX90640 provenientes de los fabricantes WaveShare y SeenGreat. Es importante notar que, aunque el chip base es el mismo (Melexis MLX90640), pueden existir diferencias menores en los componentes pasivos o el diseño de la placa entre módulos de distintos fabricantes. Las especificaciones y recomendaciones del fabricante del chip, Melexis, indican la necesidad de desacoplo adecuado de la fuente de alimentación mediante capacitores cercanos a los pines Vdd y Vss para minimizar el ruido.

    \subsection{Sensor Ambiental BME280 / DHT22}
    \label{subsec:hardware_sensores_amb}
    Para contextualizar las mediciones térmicas y posibilitar el cálculo de índices como el CWSI (Índice de Estrés Hídrico del Cultivo), es fundamental registrar de manera precisa la temperatura del aire (Ta) y la humedad relativa (HR) \cite{Almeida2024, Katimbo2022}. Con este fin, se evaluaron dos tipos de sensores ambientales digitales de bajo costo durante el desarrollo del prototipo.

    Inicialmente, se utilizó el sensor DHT22 (también conocido como AM2302). Este sensor utiliza un elemento capacitivo polimérico para la detección de humedad y un termistor para la temperatura. Opera con un voltaje de alimentación entre 3.3V y 6V DC y proporciona una señal digital calibrada a través de un protocolo de comunicación de bus único. El fabricante especifica una precisión típica de $\pm2$\% RH para la humedad (máximo $\pm5$\% RH) y $<\pm0.5$\degree C para la temperatura. Su bajo costo y facilidad de integración lo hicieron una opción atractiva en las etapas iniciales del diseño.

    Posteriormente, y como alternativa, se integró el sensor BME280 de Bosch Sensortec. Este es un sensor ambiental digital de tipo microelectromecánico (MEMS) que mide no solo Ta y HR, sino también presión atmosférica. Opera con un voltaje de alimentación principal ($V_{DD}$) de 1.71V a 3.6V y un voltaje de interfaz ($V_{DDIO}$) de 1.2V a 3.6V. La comunicación se realiza a través de interfaces digitales I\textsuperscript{2}C (hasta 3.4 MHz) o SPI (hasta 10 MHz). El BME280 ofrece una precisión típica de $\pm3$\% RH para la humedad y $\pm1.0$\degree C para la temperatura en el rango de 0 a 65\degree C.

    Durante la fase de caracterización experimental, se realizó una evaluación comparativa directa entre ambos tipos de sensores. Los prototipos 1 y 2 fueron equipados con el sensor BME280, mientras que el prototipo 3 utilizó un sensor DHT22 que había estado en uso previo. Los resultados, detallados más adelante en la sección \ref{subsec:hardware_resultados_amb}, mostraron un comportamiento casi idéntico y alta concordancia entre los dos sensores BME280. En contraste, el sensor DHT22 exhibió un sesgo sistemático significativo y una mayor inestabilidad en sus mediciones. Esta discrepancia se atribuyó tanto a la tecnología de polímero capacitivo del DHT22, susceptible a degradación y autocalentamiento \cite{SanchezSutil2021}, como al posible desgaste por uso previo del sensor específico en el prototipo 3.

    Basado en esta evidencia experimental, que demostró la superior fiabilidad y consistencia del BME280 en las condiciones de prueba, se tomó la decisión de utilizar exclusivamente el sensor BME280 en los prototipos finales destinados a mediciones fiables. La integración se realizó mediante la interfaz I\textsuperscript{2}C.

    \subsection{Sensor de Luminosidad BH1750}
    \label{subsec:hardware_bh1750}
    Para registrar la iluminancia ambiental, se integró al prototipo el sensor digital de luz ambiental BH1750FVI de ROHM Semiconductor. Este componente funciona como un luxómetro, convirtiendo la intensidad de la luz ambiental incidente en un valor digital. La comunicación con el microcontrolador ESP32-S3 se realiza a través de la interfaz de bus I\textsuperscript{2}C, compatible con velocidades de hasta 400 kHz \cite{BH1750Data}.

    El BH1750FVI se caracteriza por una respuesta espectral diseñada para aproximarse a la del ojo humano, lo que lo hace adecuado para medir la iluminancia percibida. Presenta una dependencia reducida del tipo de fuente de luz (incandescente, fluorescente, LED, luz solar) y una influencia muy baja de la radiación infrarroja. El sensor ofrece una salida digital de 16 bits, capaz de medir en un amplio rango, típicamente de 1 a 65535 lux. Además, incorpora funciones para rechazar el ruido lumínico de 50Hz/60Hz y permite ajustar la sensibilidad para compensar la influencia de ventanas ópticas o extender el rango de detección. Opera con un voltaje de alimentación (Vcc) entre 2.4V y 3.6V y tiene un bajo consumo de corriente, típicamente 120 µA durante la medición y cercano a 0.01 µA en modo de bajo consumo (power down).

    La selección de este sensor se justifica por la necesidad de proveer datos contextuales sobre las condiciones de iluminación ambiental en las que se realizan las mediciones termográficas. Durante las pruebas experimentales, el sensor BH1750FVI demostró un funcionamiento nominal y coherente, proporcionando lecturas de iluminancia consistentes con el entorno.

    \subsection{Cámara RGB OV2640}
    \label{subsec:hardware_ov2640}
    Adicionalmente, el prototipo integró un sensor de imagen CMOS a color OV2640 de OmniVision \cite{OV2640Data}. Este sensor, con un tamaño de 1/4 de pulgada, ofrece una resolución nativa de 2 megapíxeles (1600x1200 píxeles, UXGA). Fue seleccionado por su bajo costo, pequeño tamaño y compatibilidad con la interfaz de cámara DVP (Digital Video Port) del microcontrolador ESP32-S3.

    El OV2640 incluye un bloque de procesamiento de señal de imagen (ISP) integrado y soporta múltiples formatos de salida, como Raw RGB, RGB565, YUV422/420 y YCbCr422, además de formatos comprimidos como JPEG. Puede operar a diferentes velocidades de cuadro según la resolución seleccionada, alcanzando hasta 15 fps en UXGA/SXGA, 30 fps en SVGA y 60 fps en CIF. El control del sensor y la configuración de sus parámetros se realizan a través de la interfaz SCCB (Serial Camera Control Bus), compatible con el protocolo I\textsuperscript{2}C. Funciona con voltajes de alimentación separados para el núcleo (1.3V), la parte analógica (2.5-3.0V) y la interfaz de E/S (1.7V a 3.3V).

    El propósito principal de incluir esta cámara en el prototipo fue proveer un contexto visual de la escena que se está midiendo con el sensor térmico. Las imágenes RGB capturadas sirven como referencia visual de la planta o el área bajo observación, complementando los datos térmicos. Durante las pruebas, la cámara OV2640 demostró un funcionamiento nominal, capturando imágenes a color de la escena de forma coherente con la configuración establecida.

    \subsection{Regulador de Voltaje LM2596}
    \label{subsec:hardware_lm2596}
    Para la gestión de la alimentación del sistema, se seleccionó el regulador de voltaje reductor (step-down o buck) LM2596 de Texas Instruments. Este circuito integrado monolítico pertenece a la familia SIMPLE SWITCHER® y está diseñado para proporcionar todas las funciones activas necesarias para un regulador conmutado, simplificando el diseño de fuentes de alimentación \cite{LM2596Data}.

    La función principal de este componente en el prototipo es convertir el voltaje de entrada de 12V DC, proveniente de una fuente de alimentación externa, a una salida regulada y estable de 3.3V DC. Este voltaje es el requerido para alimentar de forma segura tanto al microcontrolador ESP32-S3 como a los diversos sensores integrados (térmico, ambiental, luminosidad y cámara RGB).

    La elección del LM2596S se justifica por varias de sus características clave:
    \begin{itemize}
        \item \textbf{Regulación a 3.3V:} Proporciona directamente el voltaje operativo requerido por los componentes principales del sistema, con una tolerancia garantizada de $\pm4\%$ sobre las condiciones de línea y carga especificadas.
        \item \textbf{Capacidad de Corriente:} Es capaz de manejar una corriente de carga de salida de hasta 3A, lo cual es más que suficiente para el consumo total del prototipo, incluso considerando picos de consumo durante la transmisión Wi-Fi o el procesamiento intensivo.
        \item \textbf{Amplio Rango de Entrada:} Admite un voltaje de entrada de hasta 40V, proporcionando flexibilidad y robustez frente a posibles variaciones en la fuente de alimentación de 12V.
        \item \textbf{Alta Eficiencia:} Como regulador conmutado que opera a una frecuencia fija de 150 kHz, ofrece una alta eficiencia (típicamente 73\% con entrada de 12V y carga de 3A), minimizando la disipación de calor en comparación con reguladores lineales.
        \item \textbf{Funciones de Protección:} Incorpora características de autoprotección esenciales, como limitación de corriente interna (con reducción de frecuencia en dos etapas). Estas protecciones ayudan a prevenir daños al regulador y a los componentes alimentados en caso de sobrecargas, cortocircuitos o sobrecalentamiento, contribuyendo a la robustez general del sistema.
    \end{itemize}

    Para la implementación, se siguió el circuito de aplicación típico recomendado por el fabricante, utilizando un módulo comercial basado en el LM2596S (versión de montaje superficial TO-263) que ya incluía los componentes externos necesarios (inductor, diodo, capacitores) en una pequeña placa PCB.

\section{Diseño e Integración del Sistema}
\label{sec:hardware_integracion}
Esta sección describe cómo se ensamblaron los componentes electrónicos seleccionados y cómo se desarrolló el software embebido (firmware) para controlar el sistema y gestionar los datos.

\subsection{Diseño Electrónico y Conexiones Físicas}
\label{subsec:hardware_diseno_electronico}
El ensamblaje físico de los prototipos se realizó utilizando placas de prototipado perforadas (protoboards) montadas sobre una base de madera, como se observa en las Figuras \ref{fig:prototipo_final_1} y \ref{fig:prototipo_inicial}. Esta metodología permitió una rápida iteración y modificación durante la fase de desarrollo y prueba. Todos los componentes (microcontrolador ESP32-S3, sensores MLX90640, BME280, BH1750, OV2640 y el módulo regulador LM2596) se interconectaron mediante cables DuPont sobre la protoboard.

\begin{figure}[h!]
    \centering
    \caption{Ensamblaje interno de los prototipos finales utilizando protoboard.}
    \includegraphics[width=0.8\textwidth]{img/prototipo2.jpg}
    \label{fig:prototipo_final_1}
\end{figure}

\begin{figure}[h!]
    \centering
    \caption{Ensamblaje interno del prototipo inicial.}
    \includegraphics[width=0.6\textwidth]{img/prototipo1.jpg}
    \label{fig:prototipo_inicial}
\end{figure}

La alimentación general del sistema proviene de una fuente externa de 12V DC, conectada a un terminal de tornillo. Desde allí, se alimenta el módulo regulador LM2596, que a su vez proporciona los 3.3V necesarios para el ESP32-S3 y los demás sensores. Se incluyeron pequeños ventiladores (fans) en la carcasa para facilitar la circulación de aire y mitigar posibles acumulaciones de calor generadas por los componentes electrónicos, aunque su efectividad real requeriría una evaluación térmica detallada.

Las conexiones de datos entre el ESP32-S3 y los sensores se realizaron siguiendo las interfaces requeridas por cada componente:
\begin{itemize}
    \item \textbf{Sensores MLX90640, BME280 y BH1750:} Se conectaron al bus I\textsuperscript{2}C del ESP32-S3, compartiendo las líneas SDA (Datos Serie) y SCL (Reloj Serie).
    \item \textbf{Cámara OV2640:} Se conectó utilizando la interfaz de cámara DVP (Digital Video Port) paralela del ESP32-S3, que incluye líneas de datos (típicamente 8), reloj de píxel (PCLK), sincronización horizontal (HREF) y sincronización vertical (VSYNC), además de una interfaz I\textsuperscript{2}C separada (SCCB) para la configuración de la cámara.
    \item \textbf{Módulo Lector SD:} Se utilizó la interfaz SPI del ESP32-S3 para la comunicación con la tarjeta SD, empleando las líneas MOSI, MISO, SCK y CS correspondientes.
\end{itemize}
El conjunto electrónico se alojó dentro de cajas plásticas (color naranja para los prototipos finales), proporcionando protección física a los componentes. Se realizaron perforaciones en las cajas para permitir el paso de cables (alimentación, antena externa) y asegurar la exposición de los sensores al ambiente (especialmente el térmico y ambiental).

\subsection{Desarrollo e Integración del Firmware}
\label{subsec:hardware_firmware}
El firmware para el microcontrolador ESP32-S3 se desarrolló utilizando el entorno de desarrollo integrado (IDE) Visual Studio Code con la extensión PlatformIO. PlatformIO facilita la gestión de proyectos embebidos, incluyendo la compilación, carga del firmware y manejo de dependencias \cite{PlatformIO}. El framework de desarrollo empleado fue Arduino Core para ESP32 \cite{ArduinoESP32}, que proporciona un conjunto de librerías y funciones para interactuar con el hardware del microcontrolador de forma simplificada.

La arquitectura del firmware se organizó de manera modular, separando responsabilidades en diferentes librerías y archivos fuente, como se evidencia en la estructura del proyecto proporcionada. La carpeta `src` contiene la lógica principal de la aplicación (`main.cpp`), controladores de ciclo (`CycleController`) y tareas específicas para la adquisición de datos ambientales (`EnvironmentTasks`) y de imágenes (`ImageTasks`), así como la inicialización del sistema (`SystemInit`).

La carpeta `lib` alberga un conjunto de librerías personalizadas desarrolladas para este proyecto, encapsulando la interacción con cada sensor (`BH1750Sensor`, `BME280Sensor`, `MLX90640Sensor`, `OV2640Sensor`) y gestionando funcionalidades clave del sistema:
\begin{itemize}
    \item \textbf{Conectividad:} `WifiManager` para la gestión de la conexión a redes Wi-Fi.
    \item \textbf{Comunicación:} `API` y `MultipartDataSender` para el envío de datos a un servidor backend.
    \item \textbf{Interfaz Web:} `WebPortal` para ofrecer una interfaz de configuración y monitoreo accesible vía navegador, utilizando `ESPAsyncWebServer`.
    \item \textbf{Almacenamiento:} `ConfigManager` para leer/escribir la configuración del dispositivo usando el sistema de archivos `LittleFS` \cite{LittleFS}, y `SDManager` para interactuar con la tarjeta SD.
    \item \textbf{Gestión del Tiempo:} `TimeManager` para obtener y gestionar la hora del sistema, posiblemente mediante NTP (Network Time Protocol).
    \item \textbf{Utilidades:} `LEDStatus` para controlar el LED RGB integrado, `ErrorLogger` para el registro de errores, y `EnvironmentDataJSON` para formatear los datos recolectados en JSON \cite{ArduinoJson}.
\end{itemize}

El desarrollo se apoyó en librerías de terceros gestionadas por PlatformIO, incluyendo `Adafruit MLX90640` \cite{AdafruitMLX90640}, `BH1750` \cite{BH1750Lib}, `Adafruit BME280 Library` \cite{AdafruitBME280}, `DallasTemperature` \cite{DallasTemp} y `OneWire` \cite{OneWire} (sugiriendo pruebas con sensores DS18B20 no incluidos en el prototipo final), `ESPAsyncWebServer` \cite{ESPAsyncWebServer} con `AsyncTCP` \cite{AsyncTCP}, y `ArduinoJson` \cite{ArduinoJson} para la manipulación eficiente de datos JSON. La comunicación con los sensores I\textsuperscript{2}C (MLX90640, BME280, BH1750) se implementó utilizando la librería `Wire` estándar del framework Arduino \cite{ WireLib}.

El flujo principal del firmware, orquestado en `main.cpp` y `CycleController`, involucra la inicialización de los componentes (`SystemInit`), la conexión a Wi-Fi, la sincronización horaria, y la ejecución periódica de tareas para leer los sensores ambientales y térmicos, capturar imágenes RGB, almacenar datos localmente (SD) y enviarlos al servidor backend a través de la API configurada. El portal web permite al usuario configurar parámetros como las credenciales Wi-Fi, la URL del servidor API y la frecuencia de muestreo.

\section{Metodología de Caracterización y Validación}
\label{sec:hardware_metodologia}
Para establecer la viabilidad y fiabilidad del prototipo como instrumento de medición en aplicaciones agrícolas, se diseñó e implementó una metodología de caracterización y validación experimental. Este proceso es indispensable para cuantificar la precisión del sistema y corregir posibles derivas instrumentales, un aspecto crítico al trabajar con sensores de bajo costo que son inherentemente más sensibles a las condiciones ambientales y al desgaste \cite{Jiao2022, Yun2023}. La metodología se estructuró en la definición de un sistema de referencia trazable y la ejecución de procedimientos experimentales específicos en entornos controlados.

\subsection{Sistema de Referencia Utilizado}
\label{subsec:hardware_sistema_referencia}
Se estableció un sistema de referencia para realizar las pruebas de calibración y evaluación de precisión del sensor térmico MLX90640. Este sistema permitió generar y medir temperaturas conocidas y estables que sirvieron como patrón de comparación para las lecturas del prototipo. Los componentes del sistema de referencia fueron (ver Figuras \ref{fig:cuerpo_gris} y \ref{fig:controlador_pid}):
\begin{itemize}
    \item \textbf{Cuerpo Gris de Emisividad Conocida:} Se construyó un cuerpo gris utilizando una lámina de aluminio de $10 \times 15 \times 1$ cm, pintada con una capa de pintura negra mate resistente a altas temperaturas. Esta configuración asegura una emisividad alta y conocida, estimada en 0.96, lo cual es fundamental para mediciones termográficas precisas. Este cuerpo gris actuó como el objetivo cuya temperatura superficial sería medida por los prototipos. Para calentarlo de manera controlada, se colocó sobre una estufa eléctrica de resistencia.
    \item \textbf{Controlador de Temperatura PID:} Se empleó un controlador de temperatura Proporcional-Integral-Derivativo (PID), modelo REX-C100 (RKC Instrument), para mantener la temperatura del cuerpo gris en un valor estable y predefinido durante las pruebas de calibración. El controlador ajustaba la potencia suministrada a la estufa eléctrica basándose en la retroalimentación de temperatura. Durante las pruebas con referencia estable, se configuraron temperaturas objetivo (SV - Set Value) de $170^{\circ}$C y $190^{\circ}$C.
    \item \textbf{Termocupla Tipo K:} Para obtener la medición de referencia de la temperatura superficial del cuerpo gris con alta precisión, se utilizó una termocupla tipo K conectada directamente al controlador PID. Las termocuplas de este tipo ofrecen una precisión del orden de $\pm0.3^{\circ}$C en el rango de trabajo \cite{Acorsi2020}, sirviendo como el valor "verdadero" contra el cual se compararon las mediciones del sensor MLX90640. La punta de la termocupla se fijó en contacto directo con la superficie pintada del cuerpo gris.
\end{itemize}

% Figura Cuerpo Gris
\begin{figure}[h!]
    \centering
    \caption{Cuerpo gris (lámina de aluminio pintada de negro mate) sobre estufa eléctrica, con termocupla tipo K insertada para medición de temperatura de referencia.}
    \includegraphics[width=0.7\textwidth]{img/cuerpogris.jpg}
    \label{fig:cuerpo_gris}
\end{figure}

% Figura Controlador PID
\begin{figure}[h!]
    \centering
    \caption{Controlador de temperatura PID REX-C100 utilizado para estabilizar la temperatura del cuerpo gris. Muestra la temperatura medida (PV) y la temperatura objetivo (SV).}
    \includegraphics[width=0.5\textwidth]{img/pid.jpg}
    \label{fig:controlador_pid}
\end{figure}

\subsection{Procedimientos Experimentales}
\label{subsec:hardware_procedimientos}
Se ejecutaron dos tipos principales de pruebas experimentales: pruebas de calibración cuantitativa y una prueba de contraste térmico. Todas las pruebas se realizaron en entornos controlados para minimizar la interferencia de variables externas como corrientes de aire o fluctuaciones bruscas de la temperatura ambiente. Los entornos utilizados fueron un microinvernadero (condiciones cercanas a la aplicación final) y una habitación cerrada sin ventilación.

\textbf{Pruebas de Calibración:} El objetivo principal fue evaluar la precisión y estabilidad de los sensores térmicos MLX90640 bajo diferentes condiciones ambientales y de temperatura objetivo. Se utilizó el sistema de referencia (cuerpo gris calentado y controlado por PID) como fuente de temperatura estable y conocida. Se realizaron cuatro configuraciones de prueba distintas:
\begin{enumerate}
    \item \textit{Prueba en espacio controlado con referencia inestable:} Se evaluó la respuesta inicial de los sensores a una fuente de calor menos controlada (aproximadamente $190^{\circ}$C) durante el día en una habitación cerrada. Esto permitió observar la dinámica inicial y posibles derivas rápidas.
    \item \textit{Prueba en invernadero con referencia estable (diurna):} Se analizó la precisión en el microinvernadero durante el día, usando una temperatura de referencia estable de $170^{\circ}$C. Esta prueba buscaba simular condiciones de operación diurnas, incluyendo la posible influencia de la radiación solar indirecta en el entorno del invernadero.
    \item \textit{Prueba en invernadero con referencia estable (nocturna):} Se repitió la prueba anterior durante la noche, manteniendo la referencia estable a $170^{\circ}$C, para evaluar el desempeño en ausencia de radiación solar y con temperaturas ambientales potencialmente diferentes.
    \item \textit{Prueba en espacio controlado con referencia estable:} Se realizaron mediciones con una temperatura de referencia estable de aproximadamente $190^{\circ}$C durante la noche en la habitación cerrada. Esta prueba permitió confirmar el comportamiento observado a alta temperatura sin las posibles perturbaciones ambientales presentes en el microinvernadero.
\end{enumerate}

\textbf{Verificación Posterior a 30°C:} Adicionalmente, transcurridos aproximadamente dos meses desde la caracterización inicial, se realizó una prueba de calibración complementaria. El objetivo fue verificar la consistencia y posible deriva de los prototipos 1 y 2 (equipados con sensores BME280 y MLX90640 nuevos al inicio) tras un periodo de uso intermitente, y evaluar su comportamiento en un rango de temperatura más cercano a las condiciones ambientales típicas ($30^{\circ}$C). Esta prueba se ejecutó utilizando el sistema de referencia con el cuerpo gris estabilizado a $30^{\circ}$C mediante el controlador PID. Las mediciones se llevaron a cabo al atardecer en la habitación cerrada, replicando las condiciones de las pruebas nocturnas anteriores para minimizar la influencia de factores externos.

\textbf{Prueba de Contraste Térmico:} Esta prueba se diseñó específicamente para comparar de forma directa y simultánea el comportamiento de los tres prototipos ensamblados. El procedimiento consistió en:
\begin{enumerate}
    \item Colocar frente a los tres prototipos un objetivo caliente (un vaso con agua recién hervida, iniciando a $\sim60^{\circ}$C) y un objetivo frío (un vaso con agua y hielo, a $\sim11^{\circ}$C), sobre un fondo a temperatura ambiente.
    \item Registrar secuencias de imágenes térmicas simultáneamente con los tres prototipos durante aproximadamente 20 minutos. Este tiempo fue suficiente para observar la dinámica de enfriamiento del objeto caliente y evaluar la estabilidad relativa de las mediciones entre los prototipos.
    \item Extraer de cada imagen térmica capturada la temperatura máxima (correspondiente al objetivo caliente), la temperatura mínima (objetivo frío) y la temperatura promedio de toda la escena (768 píxeles).
\end{enumerate}
Esta prueba permitió evaluar la capacidad de los prototipos para medir un rango amplio de temperaturas y detectar posibles sesgos o derivas relativas entre ellos.

\textbf{Adquisición y Preprocesamiento de Datos:} El firmware de los prototipos, desarrollado en PlatformIO, se encargó de leer los datos brutos de los sensores vía I\textsuperscript{2}C y enviarlos a un aplicativo web para su almacenamiento. La frecuencia de muestreo del sensor térmico se configuró en 0.5 Hz. Para el análisis posterior, realizado mediante scripts de Python, se implementó un paso de preprocesamiento consistente en promediar 6 capturas tomadas en 15 segundos para mejorar la estabilidad de los datos. A partir de los datos preprocesados, se calcularon las temperaturas máxima, mínima y promedio para cada cuadro térmico.

\section{Resultados y Discusión}
\label{sec:hardware_resultados}
En esta sección se presentan los hallazgos derivados de la caracterización experimental de los sensores integrados en los prototipos. Los datos obtenidos permiten validar el desempeño comparativo del sistema en condiciones controladas, contrastando el rendimiento de diferentes tecnologías de sensores y evaluando el impacto del uso previo en su fiabilidad. Estos resultados se discutirán en el contexto de los objetivos de la investigación, estableciendo la viabilidad del hardware propuesto como una herramienta de bajo costo para aplicaciones en agricultura de precisión.

\subsection{Desempeño de Sensores Ambientales}
\label{subsec:hardware_resultados_amb}
Se realizó una evaluación comparativa del rendimiento de los sensores ambientales integrados en los prototipos, enfocándose en la consistencia y fiabilidad de las mediciones de temperatura del aire (Ta) y humedad relativa (HR).

La comparación inicial se centró en el desempeño de los sensores BME280 (instalados en los prototipos 1 y 2) frente a un sensor DHT22 previamente utilizado (instalado en el prototipo 3). Las mediciones registradas a lo largo de varias horas, presentadas en la Figura \ref{fig:comparacion_ambiental_dht_bme}, revelaron diferencias significativas. Los dos prototipos equipados con sensores BME280 mostraron un comportamiento casi idéntico y una alta concordancia entre sus lecturas, tanto de temperatura como de humedad. En marcado contraste, el prototipo 3 con el sensor DHT22 exhibió un sesgo sistemático pronunciado y una mayor inestabilidad. Cuantitativamente, el sensor DHT22 registró una temperatura media aproximadamente $4^{\circ}$C superior y una humedad promedio 21 puntos porcentuales inferior a las mediciones de los BME280. Además, su desviación estándar fue casi el doble, confirmando su menor fiabilidad y mayor variabilidad en las condiciones de prueba. Estos resultados, atribuibles tanto a la tecnología de polímero capacitivo del DHT22 (susceptible a degradación y autocalentamiento \cite{SanchezSutil2021}) como al posible desgaste por uso previo del sensor específico, justificaron la selección final del BME280 para los prototipos.

% Figura Comparacion Ambiental DHT vs BME
\begin{figure}[h!]
    \centering
    \caption{Comparación de mediciones de temperatura y humedad relativa entre los sensores BME280 (Cam 1 y Cam 2) y el sensor DHT22 (Cam 3) a lo largo del tiempo.}
    \includegraphics[width=\textwidth]{img/comparacion_ambiental.png}
    \label{fig:comparacion_ambiental_dht_bme}
\end{figure}

Para evaluar la respuesta dinámica y la consistencia de los sensores BME280 bajo cambios ambientales controlados, se realizó una prueba de contraste térmico adicional, cuyos resultados se presentan en la Figura \ref{fig:contraste_ambiental}. Durante esta prueba, los tres prototipos (equipados con BME280) fueron expuestos secuencialmente a un estímulo frío (zona azulada), en este caso una bolsa de hielo, y luego a un estímulo caliente (zona rojiza), en este caso un ventilador caliente. Se observa que los tres sensores respondieron coherentemente a los estímulos: la temperatura medida disminuyó durante la exposición al frío y aumentó marcadamente durante la exposición al calor, retornando gradualmente a las condiciones ambientales al cesar los estímulos. De manera similar, la humedad relativa mostró un aumento durante la fase fría y una disminución abrupta durante la fase caliente. Los prototipos 1 y 2 muestran nuevamente una alta concordancia. El prototipo 3, aunque sigue la misma tendencia, presenta un ligero desfase (offset) respecto a los otros dos, esto debido posiblmente al uso intensivo que tuve anteriormente. No obstante, la respuesta dinámica a los cambios fue similar en los tres casos, validando la capacidad de los sensores BME280 para detectar y seguir variaciones ambientales.

% Figura Contraste Ambiental BME280
\begin{figure}[h!]
    \centering
    \caption{Respuesta de los sensores de temperatura y humedad BME280 (Cams 1, 2 y 3) a estímulos controlados de frío y calor.}
    \includegraphics[width=\textwidth]{img/contraste_ambiental.png} 
    \label{fig:contraste_ambiental}
\end{figure}

Finalmente, se evaluó el desempeño de los sensores de luminosidad BH1750 integrados en los tres prototipos. La Figura \ref{fig:comparacion_luminosidad} muestra las lecturas de iluminancia (en Lux) registradas a lo largo de una tarde. Se observa que los tres sensores siguieron la tendencia general esperada, con valores altos durante las horas de luz diurna, fluctuaciones y una disminución progresiva hacia el atardecer. Si bien los tres sensores muestran un comportamiento similar en tendencia, se aprecian algunas diferencias en la magnitud y una mayor variabilidad aparente en el prototipo 3 en ciertos intervalos. En general, los sensores BH1750 proporcionaron mediciones coherentes del nivel de luz ambiental, cumpliendo su función de proveer datos contextuales.

% Figura Comparacion Luminosidad BH1750
\begin{figure}[h!]
    \centering
    \includegraphics[width=\textwidth]{img/comparacion_luminosidad.png}
    \caption{Comparación de las mediciones de luminosidad (Lux) registradas por los sensores BH1750 de los tres prototipos a lo largo del tiempo.}
    \label{fig:comparacion_luminosidad}
\end{figure}

\subsection{Desempeño del Sensor Térmico}
\label{subsec:hardware_resultados_termico}
La caracterización de los sensores de imagen térmica MLX90640 fue un paso crucial para evaluar su precisión y fiabilidad en la medición de temperaturas superficiales. Se emplearon dos procedimientos experimentales principales: una prueba de contraste térmico cualitativa y una serie de pruebas de calibración cuantitativa contra un sistema de referencia.

\textbf{Prueba de Contraste Térmico:}
En esta prueba, se evaluó la capacidad de los tres prototipos (Cam 1 y Cam 2 con sensores SeenGreat, Cam 3 con sensor WaveShare, siendo este último el de uso previo) para medir simultáneamente un objetivo caliente ($\sim$40\degree C a 55\degree C) y uno frío ($\sim$5\degree C a 12\degree C). Los resultados se muestran en la Figura \ref{fig:contraste_termico}. Los tres prototipos lograron identificar y seguir la dinámica de enfriamiento del objetivo caliente (gráfica superior) y la relativa estabilidad del objetivo frío (gráfica central). Sin embargo, se observó un comportamiento diferencial notable en el prototipo 3. Este mostró un sesgo positivo consistente tanto en la medición de la temperatura mínima (objetivo frío) como en la temperatura promedio de toda la escena (gráfica inferior), registrando valores sistemáticamente más altos que los prototipos 1 y 2. Las mediciones del objetivo caliente fueron más cercanas entre los tres dispositivos, aunque la Cam 3 también tendió a registrar temperaturas ligeramente superiores al inicio.

% Figura Contraste Térmico
\begin{figure}[h!]
    \centering
    \caption{Prueba de contraste térmico: comparación de temperaturas máxima (vaso caliente), mínima (vaso frío) y promedio de la escena registradas por los tres prototipos (Cam 1, Cam 2: SeenGreat; Cam 3: WaveShare con uso previo).}
    \includegraphics[width=\textwidth]{img/contraste_termico.png}
    \label{fig:contraste_termico}
\end{figure}

\textbf{Pruebas de Calibración Cuantitativa:}
Para una evaluación de la precisión, se compararon las lecturas de temperatura máxima de cada prototipo contra la temperatura medida por la termocupla de referencia acoplada al cuerpo gris. Se emplearon varias métricas de error:
\begin{itemize}
    \item \textbf{Error Medio (ME)} o \textbf{Error Promedio}: Calculado como la media de las diferencias entre la lectura del sensor y la temperatura de referencia ($T_{medida} - T_{referencia}$). Indica el sesgo promedio del sensor (si tiende a medir por encima o por debajo del valor real).
    \item \textbf{Error Absoluto Medio (MAE)}: Calculado como la media de los valores absolutos de las diferencias ($|T_{medida} - T_{referencia}|$). Representa la magnitud promedio del error, sin importar su signo.
    \item \textbf{Raíz del Error Cuadrático Medio (RMSE)}: Calculado como la raíz cuadrada de la media de las diferencias al cuadrado ($\sqrt{\frac{1}{N}\sum(T_{medida} - T_{referencia})^2}$). Es una medida de la magnitud general del error, dando más peso a los errores grandes. Un RMSE bajo indica alta precisión y exactitud combinadas.
\end{itemize}

Los resultados de las pruebas de calibración bajo diferentes condiciones (presentados visualmente en la Figura \ref{fig:calibracion1}) confirmaron el desempeño inferior del prototipo 3 (equipado con el sensor MLX90640 de uso previo). En la prueba diurna en invernadero a $170^{\circ}$C, los prototipos 1 y 2 mantuvieron un RMSE bajo ($2.33^{\circ}$C y $3.11^{\circ}$C, respectivamente), indicando una buena precisión. En contraste, el prototipo 3 mostró un RMSE extremadamente alto de $20.83^{\circ}$C, con un error promedio (sesgo) negativo de $-17.28^{\circ}$C. Esta marcada diferencia subraya el impacto crítico del desgaste o uso intensivo previo en la fiabilidad de estos sensores de bajo costo. La desviación estándar ('std' en la salida del script) de las mediciones del prototipo 3 también fue consistentemente mayor, indicando una menor estabilidad.

A temperaturas de referencia más altas ($\sim190^{\circ}$C en espacio controlado), los prototipos 1 y 2 exhibieron un sesgo negativo consistente (ME de $-7.49^{\circ}$C y $-12.66^{\circ}$C), resultando en un RMSE mayor ($7.78^{\circ}$C y $12.85^{\circ}$C) que a $170^{\circ}$C. Sin embargo, este error seguía siendo considerablemente menor y más predecible que el del prototipo 3, cuyo sesgo negativo alcanzó $-17.59^{\circ}$C con un RMSE de $17.80^{\circ}$C. La menor precisión a temperaturas muy altas fue consistente entre los sensores nuevos, sugiriendo una característica intrínseca que podría ser corregida mediante calibración por software si fuese necesario para la aplicación específica. Las pruebas nocturnas a $170^{\circ}$C arrojaron resultados similares, con RMSEs bajos para los prototipos 1 y 2 ($4.07^{\circ}$C y $2.21^{\circ}$C) y un error mayor para el prototipo 3 ($7.81^{\circ}$C).

% Figura Calibración 1 (Sesiones Tarde/Noche)
\begin{figure}[h!]
    \centering
    \caption{Resultados de las pruebas de calibración iniciales. Comparación de temperaturas máxima, promedio y mínima entre los tres prototipos durante sesiones de tarde (izquierda) y noche (derecha), con indicación de las temperaturas de referencia.}
    \includegraphics[width=\textwidth]{img/calibracion1.png}
    \label{fig:calibracion1}
\end{figure}

\textbf{Verificación Posterior a 30°C:}
La prueba de calibración realizada aproximadamente dos meses después, utilizando los prototipos 1 y 2, sirvió para verificar su comportamiento tras un periodo de uso y en rangos de temperatura más amplios, incluyendo una referencia de $30^{\circ}$C. Los resultados (mostrados gráficamente en la Figura \ref{fig:calibracion2}) indicaron una alta consistencia entre las mediciones de Cam 1 y Cam 2 en los tres puntos de referencia ($30^{\circ}$C, $170^{\circ}$C y $190^{\circ}$C). A $30^{\circ}$C, ambos prototipos mostraron mediciones muy cercanas a la referencia, validando su buen desempeño en temperaturas ambientales típicas. A $170^{\circ}$C y $190^{\circ}$C, se mantuvieron los patrones observados previamente: buena precisión a $170^{\circ}$C y un sesgo negativo similar entre ambos a $190^{\circ}$C. Esta consistencia a lo largo del tiempo y en diferentes rangos sugiere una estabilidad razonable de los sensores (cuando son nuevos y se manejan adecuadamente) y refuerza la validez de los prototipos 1 y 2.

% Figura Calibración 2 (Verificación posterior)
\begin{figure}[h!]
    \centering
    \caption{Resultados de la prueba de verificación posterior (~2 meses después). Comparación de temperaturas máxima, promedio y mínima entre los prototipos 1 y 2 a temperaturas de referencia de 30°C, 170°C y 190°C.}
    \includegraphics[width=\textwidth]{img/calibracion2.png}
    \label{fig:calibracion2}
\end{figure}

La caracterización térmica demostró cuantitativamente que el sensor MLX90640 previamente utilizado (Prototipo 3) presentaba una degradación significativa en su rendimiento, haciéndolo inadecuado para mediciones fiables. Por el contrario, los sensores nuevos (Prototipos 1 y 2) mostraron una precisión aceptable (RMSE < 3.2\degree C a 170\degree C) y un comportamiento consistente, validando su idoneidad para la arquitectura del sistema propuesto.

\subsection{Validación del Sistema Integrado}
\label{subsec:hardware_validacion_integrada}
La evaluación individual de los componentes clave proporciona la base para validar el desempeño del sistema de monitoreo termográfico en su conjunto. Los resultados presentados en las subsecciones anteriores (\ref{subsec:hardware_resultados_amb} y \ref{subsec:hardware_resultados_termico}) confirman la idoneidad de la arquitectura de hardware seleccionada para los prototipos finales (configuración de Cam 1 y Cam 2).

La combinación del microcontrolador ESP32-S3, el sensor ambiental microelectromecánico BME280 y los sensores térmicos MLX90640 (nuevos) demostró constituir un sistema de monitoreo estable y funcionalmente viable. Los sensores BME280 proporcionaron mediciones consistentes y fiables de temperatura y humedad ambiental, superando claramente al sensor DHT22 evaluado inicialmente. Los sensores térmicos MLX90640, cuando eran nuevos, mostraron una precisión aceptable para aplicaciones agrícolas, especialmente en rangos de temperatura cercanos a las condiciones ambientales y hasta 170°C (RMSE < 3.2°C). Aunque se observó un sesgo a temperaturas más elevadas (~190°C), este fue consistente y predecible, sugiriendo la posibilidad de corrección por software.

La integración física mediante protoboard y la conexión a través de las interfaces digitales estándar (I\textsuperscript{2}C para MLX90640, BME280, BH1750; DVP y SCCB/I\textsuperscript{2}C para OV2640; SPI para tarjeta SD) funcionaron según lo esperado, permitiendo al firmware desarrollado gestionar la adquisición de datos de todos los periféricos. Los componentes complementarios, como el sensor de luminosidad BH1750 y la cámara RGB OV2640, también operaron nominalmente, proveyendo el contexto ambiental y visual deseado. El regulador LM2596 suministró la alimentación de 3.3V de manera estable.

Un hallazgo fundamental de esta caracterización es la confirmación cuantitativa del impacto negativo del uso previo intensivo en la fiabilidad, particularmente evidente en el hardware del prototipo 3. Este resultado subraya la importancia crítica de utilizar componentes calibrados o al menos probados y verificar periódicamente su estado para asegurar la calidad de las mediciones en despliegues a largo plazo con tecnología de bajo costo.

Por lo que, la caracterización experimental valida que la arquitectura de hardware propuesta, implementada en los prototipos 1 y 2 con componentes seleccionados y verificados, constituye un sistema integrado fiable y funcionalmente adecuado para la adquisición de datos termográficos y ambientales en el contexto agrícola. Esta validación es el fundamento técnico que permite posicionar al prototipo como una herramienta potencialmente útil y accesible para la monitorización del estrés hídrico y la toma de decisiones agronómicas.

\subsection{Limitaciones y Recomendaciones para Trabajo Futuro}
\label{subsec:hardware_limitaciones_futuro}
Si bien la caracterización realizada validó la funcionalidad básica y la viabilidad del prototipo de hardware, es importante reconocer las limitaciones inherentes a este estudio y proponer líneas de mejora y trabajo futuro centradas específicamente en el desarrollo del sistema físico.

Una limitación principal es que la validación del desempeño se realizó primordialmente en condiciones controladas (habitación cerrada y microinvernadero). Para consolidar la aplicabilidad real del sistema, es indispensable realizar evaluaciones exhaustivas a campo abierto. Esto expondrá el hardware a factores ambientales no controlados como la velocidad del viento (que puede afectar las lecturas de sensores de baja masa térmica \cite{GimenezGallego2021}), la lluvia, el polvo y la radiación solar directa, los cuales podrían impactar tanto la precisión de las mediciones como la durabilidad de los componentes. Podría ser necesario diseñar e implementar carcasas protectoras o filtros adicionales para mitigar estos efectos.

Asimismo, la calibración térmica, aunque efectiva para detectar inconsistencias y validar el rango operativo básico, se concentró en puntos de referencia a temperaturas relativamente altas o muy específicas. Futuros trabajos deberían incluir una caracterización más exhaustiva con puntos de referencia estables y precisos a lo largo de todo el espectro de temperaturas de interés agrícola (e.g., 10°C a 40°C), utilizando un sistema de calibración más exacto. Esto permitiría desarrollar modelos de corrección más robustos, posiblemente implementados directamente en el firmware, y realizar una calibración radiométrica más completa para mitigar la sensibilidad del sensor a influencias externas.

De cara al futuro, se proponen las siguientes mejoras y líneas de desarrollo para el hardware:
\begin{itemize}
    \item \textbf{Diseño Modular y Conectores:} El prototipado en protoboard, si bien útil para el desarrollo, no es robusto para despliegues a largo plazo. Se recomienda diseñar una placa de circuito impreso (PCB) a medida que integre los componentes principales. Además, implementar un sistema de conexión modular para los sensores (utilizando conectores estandarizados en lugar de conexiones directas o soldadas) facilitaría enormemente el reemplazo o la actualización de sensores individuales. Esto es particularmente relevante dado el impacto observado del desgaste en los componentes de bajo costo, permitiendo un mantenimiento más sencillo y económico.
    \item \textbf{Prototipos Robustos y Carcasas a Medida:} Para mejorar la protección contra las condiciones ambientales de campo, se debe desarrollar una carcasa más robusta y específica para el dispositivo. El uso de impresión 3D permitiría diseñar e iterar rápidamente carcasas a medida que aseguren la protección de la electrónica, faciliten el montaje en campo (e.g., en postes o estructuras de soporte) y garanticen la exposición adecuada de los sensores al ambiente, minimizando al mismo tiempo la interferencia térmica de la propia carcasa.
    \item \textbf{Adaptación para Plataformas Móviles (Drones):} Explorar la adaptación del diseño de hardware para su integración en vehículos aéreos no tripulados (UAVs o drones). Esto requeriría miniaturización adicional, optimización del consumo energético y posibles modificaciones en la óptica o el sistema de adquisición para mediciones aéreas, permitiendo cubrir áreas de cultivo más extensas.
    \item \textbf{Sistema de Calibración Mejorado:} Desarrollar un sistema de calibración de cuerpo negro más preciso y versátil que el utilizado en este estudio, capaz de generar y mantener temperaturas de referencia estables en un rango más amplio y relevante para la agricultura, y bajo diferentes condiciones de humedad y flujo de aire controlados.
\end{itemize}

La implementación de estas mejoras en el hardware contribuiría significativamente a aumentar la robustez, fiabilidad, facilidad de mantenimiento y aplicabilidad del sistema de monitoreo termográfico de bajo costo, acercándolo a una solución práctica y escalable para la agricultura de precisión.         % Capítulo III: Desarrollo y Caracterización del Hardware
% Capítulo V: Estudio Experimental en Plantas de Arándano Biloxi
\chapter{ESTUDIO EXPERIMENTAL}

\section{Introducción y Objetivos del Estudio Experimental}
Se presenta el contexto del cultivo de arándano biloxi y la relevancia de detectar tempranamente Botrytis cinerea. Se definen los objetivos específicos del estudio experimental.

\section{Preparación del Hongo \textit{Botrytis cinerea}}
Descripción del protocolo para el manejo, cultivo y preparación del hongo, detallando medidas de seguridad y procedimientos para garantizar la viabilidad del patógeno.

\section{Diseño Experimental}
\begin{itemize}
    \item \textbf{Procedimiento de Infección:} Descripción de cómo se realizará la inoculación en las plantas.
    \item \textbf{Planificación de la Toma de Datos:} Definición del cronograma, frecuencia y condiciones bajo las cuales se efectuarán las mediciones.
    \item \textbf{Técnicas Utilizadas:} Detalle de las metodologías (incluyendo termografía) y herramientas empleadas para la captura de datos.
\end{itemize}

\section{Recolección y Análisis de Datos}
Metodología para la recopilación de datos experimentales, métodos estadísticos aplicados para el análisis comparativo entre plantas infectadas y de control, y presentación de resultados preliminares.

\section{Resultados y Discusión}
Presentación detallada de los hallazgos experimentales, análisis crítico de los resultados y comparación con los objetivos planteados en el estudio. Se discutirán las implicaciones de la detección temprana y su potencial para mejorar el manejo del cultivo.      % Capítulo IV: Estudio Experimental en Plantas de Arándano Biloxi
\chapter{Resultados y Conclusiones Finales}
\label{chap:conclusiones}

Este capítulo final presenta una síntesis de los resultados obtenidos a lo largo del proyecto, ofreciendo una discusión crítica sobre el cumplimiento de los objetivos, las implicaciones de los hallazgos y las futuras líneas de investigación. Se busca dar una respuesta clara a la pregunta de investigación y consolidar el aporte de este trabajo al campo de la agricultura de precisión.

\section{Respuesta a la Pregunta de Investigación}
\label{sec:respuesta_pregunta}

La pregunta central que guio este proyecto fue: \textbf{¿Es posible desarrollar un sistema de bajo costo, basado en termografía infrarroja e inteligencia artificial, que permita detectar de manera temprana el estrés hídrico en plantas de arándano variedad Biloxi para optimizar la gestión del riego?}

La respuesta, a la luz de los resultados obtenidos, es afirmativa. El desarrollo y validación del sistema integrado por hardware, software y un estudio experimental han demostrado que es factible construir e implementar una solución tecnológica de bajo costo capaz de monitorear indicadores fisiológicos asociados al estrés hídrico.

\begin{itemize}
    \item \textbf{Viabilidad Tecnológica:} Se diseñó, ensambló y validó metrológicamente un prototipo de hardware funcional utilizando componentes asequibles como el microcontrolador ESP32-S3 y el sensor térmico MLX90640. La caracterización del sensor demostró una precisión aceptable para aplicaciones agrícolas, confirmando que la barrera del alto costo de los equipos comerciales puede ser superada.
    
    \item \textbf{Detección Temprana:} El estudio experimental, aunque concebido como una prueba de concepto, evidenció que el sistema es suficientemente sensible para registrar los cambios en la temperatura foliar de la planta de arándano en respuesta a la restricción de riego. El incremento sostenido del Índice de Estrés Hídrico del Cultivo (CWSI) durante la fase de sequía y su rápido descenso tras la rehidratación, constituye una prueba empírica de que el sistema detecta la respuesta fisiológica de la planta antes de que los signos de marchitamiento sean visualmente evidentes.
\end{itemize}

Por lo tanto, el proyecto demuestra exitosamente que la combinación de termografía de bajo costo y análisis de datos es una estrategia viable y prometedora para la detección temprana del estrés hídrico en el cultivo de arándano.

\section{Discusión General}
\label{sec:discusion_general}

A continuación, se presenta un análisis crítico de la consecución de los objetivos planteados, las fortalezas y limitaciones del sistema, y su impacto potencial.

\subsection{Consecución de los Objetivos Planteados}
El proyecto cumplió satisfactoriamente con los objetivos específicos propuestos:

\begin{enumerate}
    \item \textbf{Diseñar y construir un prototipo de hardware:} Se logró el diseño y ensamblaje de un dispositivo funcional, robusto y de bajo costo. La validación metrológica confirmó su fiabilidad como instrumento de medición, siendo este uno de los principales aportes del proyecto.
    
    \item \textbf{Desarrollar un software de procesamiento:} Se implementó el firmware necesario para la adquisición continua de datos térmicos y ambientales. Adicionalmente, se desarrollaron los scripts para la segmentación de imágenes y el cálculo del CWSI, permitiendo transformar los datos crudos en un indicador agronómico valioso.
    
    \item \textbf{Implementar un modelo de inteligencia artificial:} Aunque el estudio experimental se centró en la prueba de concepto y la validación del hardware, el diseño del sistema contempla la integración futura de modelos de IA. La recolección de un conjunto de datos etiquetados (planta con y sin estrés) durante 55 días sienta las bases para el entrenamiento de algoritmos de clasificación, cumpliendo con la fase inicial de este objetivo.
    
    \item \textbf{Crear una aplicación web:} Se desarrolló una plataforma web que permite la visualización de los datos recopilados por el hardware, proporcionando una interfaz intuitiva para el usuario final. Esto completa el ciclo desde la captura del dato en campo hasta su presentación para la toma de decisiones.
\end{enumerate}

\subsection{Fortalezas y Limitaciones del Sistema}

\subsubsection{Fortalezas}
\begin{itemize}
    \item \textbf{Bajo Costo:} La principal fortaleza del sistema es su asequibilidad. Al utilizar componentes de hardware de código abierto y ampliamente disponibles, el costo del prototipo es una fracción del de las cámaras termográficas comerciales, lo que democratiza el acceso a esta tecnología.
    \item \textbf{Monitoreo Continuo y No Invasivo:} A diferencia de los métodos tradicionales (e.g., bomba de Scholander), el sistema permite un monitoreo constante sin afectar a la planta, proporcionando una visión dinámica de su estado hídrico a lo largo del día y en diferentes condiciones climáticas.
    \item \textbf{Sistema Integral:} El proyecto no se limita al hardware, sino que ofrece una solución completa que abarca la captura de datos, el procesamiento, el almacenamiento y la visualización, lo que aumenta su valor práctico para el usuario final.
\end{itemize}

\subsubsection{Limitaciones}
\begin{itemize}
    \item \textbf{Alcance del Experimento:} La prueba de concepto se realizó con un solo individuo, lo que impide una validación estadística robusta y la generalización de los resultados. Factores como la variabilidad entre plantas y las diferentes condiciones microclimáticas de un cultivo real no fueron evaluados.
    \item \textbf{Resolución del Sensor Térmico:} La resolución de 32x24 píxeles del sensor MLX90640, aunque suficiente para esta prueba de concepto, puede ser una limitación para analizar plantas a mayor distancia o para obtener detalles finos de la distribución de temperatura en el dosel.
    \item \textbf{Dependencia de la Conectividad:} Aunque cuenta con almacenamiento local en una tarjeta SD, la funcionalidad en tiempo real del sistema depende de una conexión Wi-Fi estable para la transmisión de datos al servidor, lo cual puede ser un desafío en zonas rurales.
\end{itemize}

\subsection{Impacto Potencial}
La implementación de este sistema en un entorno real tiene el potencial de generar un impacto significativo en la agricultura del arándano. Al proporcionar a los agricultores una herramienta precisa y asequible para programar el riego basada en las necesidades reales de la planta, se pueden lograr beneficios como:
\begin{itemize}
    \item \textbf{Optimización del Uso del Agua:} Reducción del consumo de agua al evitar el riego innecesario o excesivo.
    \item \textbf{Mejora del Rendimiento y Calidad del Cultivo:} Evitar el estrés hídrico, que afecta negativamente el tamaño y la calidad de la fruta.
    \item \textbf{Sostenibilidad Agrícola:} Promover prácticas agrícolas más sostenibles y resilientes al cambio climático.
\end{itemize}

\section{Recomendaciones y Trabajo Futuro}
\label{sec:recomendaciones}

A partir de los resultados y las limitaciones identificadas, se proponen las siguientes líneas de trabajo futuro:

\begin{itemize}
    \item \textbf{Escalamiento del Estudio Experimental:} Realizar experimentos con un mayor número de plantas, incluyendo réplicas y grupos de control. Evaluar el sistema en condiciones de campo reales para analizar su comportamiento frente a la variabilidad climática y del cultivo.
    
    \item \textbf{Integración de Inteligencia Artificial:} Utilizar el conjunto de datos recopilado para entrenar y validar modelos de aprendizaje automático (e.g., SVM, Redes Neuronales) que puedan clasificar automáticamente el estado hídrico de las plantas (e.g., "Normal", "Estrés Leve", "Estrés Severo") y generar alertas automáticas.
    
    \item \textbf{Mejora del Hardware:} Explorar la integración de sensores térmicos de mayor resolución a medida que su costo disminuya. Incorporar una cámara en el espectro visible para correlacionar los datos térmicos con imágenes RGB y aplicar técnicas de visión por computador para una mejor segmentación del dosel.
    
    \item \textbf{Calibración Específica del CWSI:} Desarrollar una línea base del CWSI específica para el arándano var. Biloxi en las condiciones climáticas de la sabana de Bogotá, lo que aumentaría la precisión del índice.
    
    \item \textbf{Desarrollo de una Red de Sensores:} Ampliar el sistema para crear una red de nodos de monitoreo distribuidos en un cultivo, permitiendo generar mapas de estrés hídrico y una gestión del riego por zonas.
\end{itemize}

\section{Conclusiones Finales}
\label{sec:conclusiones_finales}

Este trabajo de grado ha cumplido exitosamente su objetivo principal al demostrar que es posible desarrollar un sistema funcional y de bajo costo para la detección de estrés hídrico en plantas de arándano Biloxi mediante termografía infrarroja.

Se ha diseñado y validado un prototipo de hardware que representa una alternativa asequible a los equipos comerciales. A través de un estudio experimental controlado, se comprobó que el sistema es capaz de detectar los cambios fisiológicos en la planta asociados a la falta de agua, validando la prueba de concepto.

El proyecto no solo aporta una solución tecnológica, sino que también contribuye con un valioso conjunto de datos de 55 días de monitoreo continuo, que servirá como base para el desarrollo de futuros modelos de inteligencia artificial. A pesar de sus limitaciones, este trabajo sienta las bases para futuras investigaciones y desarrollos que podrían tener un impacto positivo en la sostenibilidad y eficiencia de la producción de arándanos en Colombia, facilitando la adopción de prácticas de agricultura de precisión en un sector clave de la economía nacional.     % Capítulo V: Resultados Integrados y Conclusiones Globales

% Bibliografía
\addcontentsline{toc}{chapter}{Bibliografía}
\bibliography{bibliografia}

% Fin del Documento
\end{document}