% Capítulo I: Informe de Investigación
\doublespacing
\chapter{INFORME DE INVESTIGACIÓN}

\section{Estado del Arte}
Texto del estado del arte...

\section{Línea de Investigación}
Texto de la línea de investigación...

\section{Planteamiento del Problema y Pregunta de Investigación}
La producción de arándanos está en aumento debido a su alta demanda, siendo la variedad Biloxi la que más se cultiva en el altiplano cundiboyacense según datos de Quintana (2020). Sin embargo, la planta requiere condiciones específicas de temperatura para dar fruto. La falta de estas condiciones la hace susceptible a enfermedades, representando un desafío significativo para los productores. Sin un manejo oportuno, las enfermedades pueden causar la muerte de las plantas, afectando el cultivo y provocando pérdidas económicas. Esto subraya la importancia de una gestión adecuada, reflejada en las exportaciones de frutos del género Vaccinium en Colombia, que alcanzaron los 2,2 millones de dólares, según la Asociación Nacional de Comercio Exterior (ANALDEX, 2022).

\bigskip
Respecto a las enfermedades en plantas (fitopatologías), Quintana afirma que la Botrytis Cinerea es una de las más comunes en el arándano. La Botryotinia Fuckeliana (fase asexuada: Botrytis Cinerea), comúnmente conocida como Botrytis o moho gris, es una enfermedad fúngica que afecta una amplia variedad de plantas, incluyendo la planta de arándano Biloxi (Quintana, 2020). Esta enfermedad es particularmente destructiva en condiciones de alta humedad y temperaturas moderadas, promoviendo la formación de esporas. Los síntomas típicos de la Botrytis incluyen manchas marrones en hojas, flores y frutos, que eventualmente se cubren de un moho gris característico. En las hojas, causa lesiones de color café que comienzan generalmente por el centro de la lámina y se extienden hacia los bordes, produciendo una necrosis extensiva. En condiciones de alta humedad, sobre las lesiones de las hojas se desarrollan las estructuras reproductivas del patógeno (conidióforos y conidios), que dan un aspecto plomizo (grisáceo o plateado) en los tejidos (Morales, et al. 2017).

\bigskip
Una técnica no invasiva ampliamente utilizada en agricultura para monitorear la salud de las plantas es la termografía infrarroja (IRT), permitiendo además gestionar el riego, detectar enfermedades y estimar la producción (Dong et al., 2024; Aux et al., 2022). Esta herramienta es fundamental para avanzar hacia una agricultura más automatizada, precisa y sostenible, permitiendo supervisar el estrés térmico en cultivos y analizar el impacto de patógenos en la transpiración de las plantas. Aunque la investigación demuestra que la IRT está superando sus limitaciones y se está convirtiendo en un método robusto, confiable y económico para determinar el estado hídrico de las plantas y detectar el estrés (Pineda et al., 2020), su alto costo actual restringe su uso principalmente a grupos de investigación y empresas especializadas (García Tejero et al., 2015).

\bigskip
Teniendo en cuenta las afectaciones de la Botrytis a los cultivos de arándano, es crucial explorar métodos no invasivos como la termografía. Si la termografía permite hacer una detección temprana de esta enfermedad se podría mejorar la gestión de la salud del cultivo y reducir las pérdidas. Lo cual plantea la pregunta, ¿Cómo desarrollar un sistema para aproximar la detección temprana de Botrytis Cinerea en plantas de arándanos Biloxi mediante hardware de bajo costo?


\section{Objetivo General y Objetivos Específicos}
\subsection{Objetivo General}
Desarrollar un sistema de detección aproximada de Botrytis Cinerea en plantas de arándano Biloxi utilizando termografía infrarroja (IRT) mediante hardware de bajo costo.

\subsection{Objetivos Específicos}
\begin{enumerate}
    \item Identificar y documentar los requisitos funcionales y no funcionales del sistema.
    \item Modelar la arquitectura del sistema mediante la creación de diagramas UML.
    \item Integrar el hardware del módulo termográfico para la recolección de datos.
    \item Desarrollar el software que recolecte los datos del módulo termográfico para su posterior procesamiento.
    \item Analizar los datos obtenidos en plantas infectadas por Botrytis Cinerea y plantas saludables para determinar la precisión del sistema desarrollado.
\end{enumerate}

\section{ Alcance e Impacto del Proyecto }
El presente proyecto busca ser un apoyo a la producción de arándanos a través de la implementación de tecnologías de precisión, como la termografía, para monitorear y gestionar la salud de los cultivos. Esto podría permitir a los agricultores minimizar las perdidas en los cultivos y poder “satisfacer los desafíos de seguridad alimentaria local, regional y global del siglo XXI” (Vargas Q. \& Best S., 2021).

\bigskip
La agricultura tecnificada y precisa es más sostenible porque permite utilizar recursos de manera óptima, como la utilización de fitosanitarios solo donde sea necesario. Según Manuel Pérez-Ruiz, director del máster en Agricultura Digital e Innovación Agroalimentaria de la Universidad de Sevilla, esta forma de agricultura también ayuda a que los agricultores sean más competitivos y no abandonen su territorio (Communications, 2024).
Este proyecto está alineado con los Objetivos de Desarrollo Sostenible (ODS), específicamente con el ODS 9, que busca promover la inversión en infraestructura y la innovación para impulsar el crecimiento económico y el desarrollo sostenible. También contribuye a reducir la huella ecológica mediante la gestión eficiente de los recursos naturales compartidos y la implementación de prácticas agrícolas sostenibles, lo que se relaciona con el ODS 12.

\bigskip
La implementación de este proyecto puede tener un impacto significativo en la seguridad alimentaria y la economía local. Al reducir las pérdidas económicas en la producción de cultivos y al promover la eficiencia energética, podemos crear cadenas de producción y suministro más eficientes. Además, el acceso a tecnologías de precisión puede ayudar a reducir la brecha digital entre países desarrollados y en desarrollo, lo que se relaciona con el ODS 9.


\section{Metodología}
\subsection{Metodología de Investigación}
Para llevar a cabo esta investigación se utilizará la metodología de investigación mixta, la cual combina la perspectiva cuantitativa y cualitativa, lo que permite dar profundidad al análisis y comprender mejor los procesos. Esto implica recopilar, analizar e interpretar datos tanto cualitativos como cuantitativos para obtener una visión más completa de la problemática a estudiar. Esta metodología busca compensar las limitaciones de cada enfoque al mismo tiempo que fortalece la validez de la interpretación de los resultados (Hamui-Sutton, 2013).

\bigskip
La metodología mixta en este proyecto se empleará con el fin de combinar la recopilación y análisis de datos cuantitativos, como las lecturas de temperatura obtenidas a través del módulo de cámara térmica, con la exploración cualitativa de los resultados, como la descripción de las características en las plantas de arándano biloxi. Además de los datos termográficos, también se tomarán en cuenta otras variables importantes para el crecimiento de la planta y el hongo, buscando así tener una visión más completa de los factores que influyen en su desarrollo.Esto permitirá obtener una vista más amplia de la problemática a estudiar, comprendiendo así la relación entre las lecturas termográficas y la presencia de Botrytis Cinerea, además de aquellos factores que puedan influir en la precisión del sistema.


\subsection{Metodología de Desarrollo}
Para el desarrollo del sistema, se adoptarán elementos del marco de trabajo ágil Scrum, un enfoque de desarrollo iterativo y colaborativo diseñado para fomentar la adaptabilidad y la entrega continua de valor. Este método, reconocido por su flexibilidad y eficiencia, se basa en ciclos cortos de trabajo llamados Sprints, que permiten planificar, ejecutar y revisar tareas de manera organizada y dinámica (SCRUMstudy™, 2022).

\bigskip
Durante el proyecto, se trabajará en Sprints semanales, en los cuales se evaluarán los avances del Product Backlog previamente definido. Esta estructura facilita la retroalimentación constante, priorización de tareas y ajustes según las necesidades del proyecto, asegurando un desarrollo estructurado y orientado a cumplir los objetivos establecidos. La implementación de Scrum permitirá un desarrollo funcional que cumpla con las expectativas de precisión y eficacia requeridas.

\section{Marcos de Referencia}
\subsection{Marco Teórico}
Texto del marco teórico...

\subsection{Marco Legal}
Texto del marco legal...