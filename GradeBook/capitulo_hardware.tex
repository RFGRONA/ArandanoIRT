% Capítulo IV: Desarrollo y Caracterización del Hardware
\chapter{DOCUMENTACIÓN DEL HARDWARE}

\section{Introducción}
En este capítulo se presenta la importancia del hardware en el sistema de detección temprana de Botrytis cinerea, justificando la selección de cada componente y su contribución al funcionamiento global del sistema.

\section{Objetivos}
\begin{itemize}
    \item Describir los componentes seleccionados para el sistema de detección.
    \item Detallar la metodología utilizada para la caracterización y validación de cada componente.
    \item Presentar los resultados obtenidos en la integración del hardware y su rendimiento en condiciones experimentales.
\end{itemize}

\section{Descripción de Componentes}
\subsection{Microcontrolador ESP32-S3-WROOM-1 N16R8}
Descripción del microcontrolador y sus características principales.

\subsection{Cámara termográfica MLX90640}
Descripción del sensor de termografía, sus especificaciones y función en la detección.

\subsection{Sensor de luz BH1750}
Detalle del sensor de luminosidad y su relevancia en el monitoreo ambiental.

\subsection{Sensor de humedad y temperatura DHT22}
Explicación del sensor de temperatura y humedad, resaltando su precisión y rango de medición.

\subsection{Cámara RGB OV2640}
Breve descripción de la cámara adicional y sus aplicaciones en el proyecto.

\subsection{Regulador de voltaje LM2596}
Descripción del regulador de voltaje y su importancia para garantizar la estabilidad del sistema.

\section{Metodología de Caracterización}
Esta sección describe el proceso sistemático para evaluar y validar el desempeño de cada componente y su integración en el sistema final.

\subsection{Evaluación y verificación de componentes}
\begin{itemize}
    \item \textbf{Definición de Parámetros de Evaluación:} Establecer los parámetros (por ejemplo, precisión, tiempo de respuesta y estabilidad) a medir para cada sensor y módulo.
    \item \textbf{Diseño de Protocolos de Prueba:} Elaborar procedimientos detallados para realizar pruebas de calibración y verificación en condiciones controladas.
    \item \textbf{Implementación de Ensayos Experimentales:} Ejecutar las pruebas en laboratorio, documentando condiciones ambientales, configuraciones y resultados obtenidos.
    \item \textbf{Análisis de Resultados:} Comparar los datos obtenidos con las especificaciones del fabricante y los requerimientos del proyecto.
\end{itemize}

\subsection{Configuración e Integración del Firmware}
\begin{itemize}
    \item \textbf{Diseño del Esquema de Integración:} Elaborar diagramas que muestren la conexión física entre el ESP32 y cada componente, indicando rutas de comunicación (I2C, SPI, etc.).
    \item \textbf{Configuración en Platform.io IDE:} Documentar el proceso de configuración, instalación de librerías, asignación de pines y gestión de interrupciones o tiempos de muestreo.
    \item \textbf{Pruebas de Integración:} Realizar pruebas para verificar la comunicación y correcta transmisión de datos entre el hardware y el firmware.
    \item \textbf{Documentación de la Integración:} Registrar todos los pasos y resultados obtenidos para facilitar futuras revisiones o ajustes.
\end{itemize}

\subsection{Validación y Análisis de Resultados}
En esta sección se evalúa el rendimiento global del sistema integrado mediante la comparación de datos experimentales controlados con los objetivos del proyecto. Se presentan datos, gráficos y análisis que demuestran el desempeño de cada componente y del sistema en conjunto, contrastándolos con los parámetros de referencia y registrando hallazgos para futuras revisiones.

\section{Implementación del Sistema Integrado}
Se describe la implementación práctica, incluyendo la integración final del hardware con el firmware, la calibración de sensores y la ejecución de pruebas en condiciones reales para ajustar y optimizar el sistema.

\section{Resultados}
Interpretación de los resultados finales obtenidos, identificando fortalezas y áreas de mejora, y estableciendo la relación entre el rendimiento del hardware y los objetivos del proyecto.
